\documentclass[]{article}
\usepackage{lmodern}
\usepackage{amssymb,amsmath}
\usepackage{ifxetex,ifluatex}
\usepackage{fixltx2e} % provides \textsubscript
\ifnum 0\ifxetex 1\fi\ifluatex 1\fi=0 % if pdftex
  \usepackage[T1]{fontenc}
  \usepackage[utf8]{inputenc}
\else % if luatex or xelatex
  \ifxetex
    \usepackage{mathspec}
  \else
    \usepackage{fontspec}
  \fi
  \defaultfontfeatures{Ligatures=TeX,Scale=MatchLowercase}
\fi
% use upquote if available, for straight quotes in verbatim environments
\IfFileExists{upquote.sty}{\usepackage{upquote}}{}
% use microtype if available
\IfFileExists{microtype.sty}{%
\usepackage{microtype}
\UseMicrotypeSet[protrusion]{basicmath} % disable protrusion for tt fonts
}{}
\usepackage[margin=1in]{geometry}
\usepackage{hyperref}
\hypersetup{unicode=true,
            pdftitle={Homework 8},
            pdfauthor={Emorie Beck},
            pdfborder={0 0 0},
            breaklinks=true}
\urlstyle{same}  % don't use monospace font for urls
\usepackage{color}
\usepackage{fancyvrb}
\newcommand{\VerbBar}{|}
\newcommand{\VERB}{\Verb[commandchars=\\\{\}]}
\DefineVerbatimEnvironment{Highlighting}{Verbatim}{commandchars=\\\{\}}
% Add ',fontsize=\small' for more characters per line
\usepackage{framed}
\definecolor{shadecolor}{RGB}{248,248,248}
\newenvironment{Shaded}{\begin{snugshade}}{\end{snugshade}}
\newcommand{\KeywordTok}[1]{\textcolor[rgb]{0.13,0.29,0.53}{\textbf{#1}}}
\newcommand{\DataTypeTok}[1]{\textcolor[rgb]{0.13,0.29,0.53}{#1}}
\newcommand{\DecValTok}[1]{\textcolor[rgb]{0.00,0.00,0.81}{#1}}
\newcommand{\BaseNTok}[1]{\textcolor[rgb]{0.00,0.00,0.81}{#1}}
\newcommand{\FloatTok}[1]{\textcolor[rgb]{0.00,0.00,0.81}{#1}}
\newcommand{\ConstantTok}[1]{\textcolor[rgb]{0.00,0.00,0.00}{#1}}
\newcommand{\CharTok}[1]{\textcolor[rgb]{0.31,0.60,0.02}{#1}}
\newcommand{\SpecialCharTok}[1]{\textcolor[rgb]{0.00,0.00,0.00}{#1}}
\newcommand{\StringTok}[1]{\textcolor[rgb]{0.31,0.60,0.02}{#1}}
\newcommand{\VerbatimStringTok}[1]{\textcolor[rgb]{0.31,0.60,0.02}{#1}}
\newcommand{\SpecialStringTok}[1]{\textcolor[rgb]{0.31,0.60,0.02}{#1}}
\newcommand{\ImportTok}[1]{#1}
\newcommand{\CommentTok}[1]{\textcolor[rgb]{0.56,0.35,0.01}{\textit{#1}}}
\newcommand{\DocumentationTok}[1]{\textcolor[rgb]{0.56,0.35,0.01}{\textbf{\textit{#1}}}}
\newcommand{\AnnotationTok}[1]{\textcolor[rgb]{0.56,0.35,0.01}{\textbf{\textit{#1}}}}
\newcommand{\CommentVarTok}[1]{\textcolor[rgb]{0.56,0.35,0.01}{\textbf{\textit{#1}}}}
\newcommand{\OtherTok}[1]{\textcolor[rgb]{0.56,0.35,0.01}{#1}}
\newcommand{\FunctionTok}[1]{\textcolor[rgb]{0.00,0.00,0.00}{#1}}
\newcommand{\VariableTok}[1]{\textcolor[rgb]{0.00,0.00,0.00}{#1}}
\newcommand{\ControlFlowTok}[1]{\textcolor[rgb]{0.13,0.29,0.53}{\textbf{#1}}}
\newcommand{\OperatorTok}[1]{\textcolor[rgb]{0.81,0.36,0.00}{\textbf{#1}}}
\newcommand{\BuiltInTok}[1]{#1}
\newcommand{\ExtensionTok}[1]{#1}
\newcommand{\PreprocessorTok}[1]{\textcolor[rgb]{0.56,0.35,0.01}{\textit{#1}}}
\newcommand{\AttributeTok}[1]{\textcolor[rgb]{0.77,0.63,0.00}{#1}}
\newcommand{\RegionMarkerTok}[1]{#1}
\newcommand{\InformationTok}[1]{\textcolor[rgb]{0.56,0.35,0.01}{\textbf{\textit{#1}}}}
\newcommand{\WarningTok}[1]{\textcolor[rgb]{0.56,0.35,0.01}{\textbf{\textit{#1}}}}
\newcommand{\AlertTok}[1]{\textcolor[rgb]{0.94,0.16,0.16}{#1}}
\newcommand{\ErrorTok}[1]{\textcolor[rgb]{0.64,0.00,0.00}{\textbf{#1}}}
\newcommand{\NormalTok}[1]{#1}
\usepackage{graphicx,grffile}
\makeatletter
\def\maxwidth{\ifdim\Gin@nat@width>\linewidth\linewidth\else\Gin@nat@width\fi}
\def\maxheight{\ifdim\Gin@nat@height>\textheight\textheight\else\Gin@nat@height\fi}
\makeatother
% Scale images if necessary, so that they will not overflow the page
% margins by default, and it is still possible to overwrite the defaults
% using explicit options in \includegraphics[width, height, ...]{}
\setkeys{Gin}{width=\maxwidth,height=\maxheight,keepaspectratio}
\IfFileExists{parskip.sty}{%
\usepackage{parskip}
}{% else
\setlength{\parindent}{0pt}
\setlength{\parskip}{6pt plus 2pt minus 1pt}
}
\setlength{\emergencystretch}{3em}  % prevent overfull lines
\providecommand{\tightlist}{%
  \setlength{\itemsep}{0pt}\setlength{\parskip}{0pt}}
\setcounter{secnumdepth}{0}
% Redefines (sub)paragraphs to behave more like sections
\ifx\paragraph\undefined\else
\let\oldparagraph\paragraph
\renewcommand{\paragraph}[1]{\oldparagraph{#1}\mbox{}}
\fi
\ifx\subparagraph\undefined\else
\let\oldsubparagraph\subparagraph
\renewcommand{\subparagraph}[1]{\oldsubparagraph{#1}\mbox{}}
\fi

%%% Use protect on footnotes to avoid problems with footnotes in titles
\let\rmarkdownfootnote\footnote%
\def\footnote{\protect\rmarkdownfootnote}

%%% Change title format to be more compact
\usepackage{titling}

% Create subtitle command for use in maketitle
\newcommand{\subtitle}[1]{
  \posttitle{
    \begin{center}\large#1\end{center}
    }
}

\setlength{\droptitle}{-2em}
  \title{Homework 8}
  \pretitle{\vspace{\droptitle}\centering\huge}
  \posttitle{\par}
\subtitle{Psych 5068}
  \author{Emorie Beck}
  \preauthor{\centering\large\emph}
  \postauthor{\par}
  \predate{\centering\large\emph}
  \postdate{\par}
  \date{\today}

\usepackage{fancyhdr}
\usepackage{array}
\usepackage{longtable}
\usepackage{lscape}
\newcommand{\blandscape}{\begin{landscape}}
\newcommand{\elandscape}{\end{landscape}}
\usepackage{dcolumn}
\usepackage{bbm}
\usepackage{threeparttable}
\usepackage{booktabs}
\usepackage{expex}
\usepackage{rotating, graphicx}
\usepackage{tabulary}
\usepackage{algorithm}
\usepackage{multirow}
\usepackage{colortbl}
\usepackage{longtable}
\usepackage{array}
\usepackage{multirow}
\usepackage[table]{xcolor}
\usepackage{wrapfig}
\usepackage{float}
\usepackage{pdflscape}
\usepackage{tabu}
\usepackage{threeparttable}
\usepackage{booktabs}
\usepackage{longtable}
\usepackage{array}
\usepackage{multirow}
\usepackage[table]{xcolor}
\usepackage{wrapfig}
\usepackage{float}
\usepackage{colortbl}
\usepackage{pdflscape}
\usepackage{tabu}
\usepackage{threeparttable}
\usepackage[normalem]{ulem}

\begin{document}
\maketitle

{
\setcounter{tocdepth}{2}
\tableofcontents
}
\section{Workspace}\label{workspace}

\subsection{Packages}\label{packages}

\subsection{Data}\label{data}

\begin{Shaded}
\begin{Highlighting}[]
\KeywordTok{source}\NormalTok{(}\StringTok{"https://raw.githubusercontent.com/emoriebeck/homeworks/master/table_fun.R"}\NormalTok{)}
\NormalTok{data_url <-}\StringTok{ "https://raw.githubusercontent.com/emoriebeck/homeworks/master/homework8/iq(2).csv"}
\NormalTok{dat      <-}\StringTok{ }\NormalTok{data_url }\OperatorTok\StringTok{ }\NormalTok{read.csv }\OperatorTok\StringTok{ }\NormalTok{tbl_df }
\end{Highlighting}
\end{Shaded}

\section{Question 1}\label{question-1}

First, create a grand mean centered version of the family size variable.
Name it fam\_size\_GMC.

\begin{Shaded}
\begin{Highlighting}[]
\NormalTok{dat <-}\StringTok{ }\NormalTok{dat }\OperatorTok
\StringTok{  }\KeywordTok{mutate}\NormalTok{(}\DataTypeTok{fam_size_GMC =}\NormalTok{ fam_size }\OperatorTok{-}\StringTok{ }\KeywordTok{mean}\NormalTok{(fam_size, }\DataTypeTok{na.rm =}\NormalTok{ T))}
\end{Highlighting}
\end{Shaded}

\section{Question 2}\label{question-2}

\begin{enumerate}
\def\labelenumi{\arabic{enumi}.}
\setcounter{enumi}{1}
\tightlist
\item
  Begin wiht the following model (Model 1), which treats all
  coefficients as random. Is there any evidence for birth order or
  family size effects in this analysis?
\end{enumerate}

\(score = \pi_{0jk}*verbal + \pi_{1jk}*spatial\)

\(pi_{0jk} = \beta_{00k} + \beta_{01k}*order + r_{00k}\)\\
\(pi_{1jk} = \beta_{10k} + \beta_{11k}*order + r_{10k}\)

\(\beta_{00k} = \gamma_{000} + \gamma_{001}*famsize + u_{00k}\)\\
\(\beta_{01k} = \gamma_{010} + \gamma_{011}*famsize + u_{01k}\)\\
\(\beta_{10k} = \gamma_{100} + \gamma_{101}*famsize + u_{10k}\)\\
\(\beta_{11k} = \gamma_{110} + \gamma_{111}*famsize + u_{11k}\)

\begin{Shaded}
\begin{Highlighting}[]
\NormalTok{Model_}\DecValTok{1}\NormalTok{ <-}\StringTok{ }\KeywordTok{lmer}\NormalTok{(score }\OperatorTok{~}\StringTok{ }
\StringTok{                }\OperatorTok{-}\DecValTok{1} \OperatorTok{+}\StringTok{ }\NormalTok{verbal }\OperatorTok{+}\StringTok{ }\NormalTok{spatial }\OperatorTok{+}\StringTok{ }\CommentTok{# level 1}
\StringTok{                }\NormalTok{order}\OperatorTok{:}\NormalTok{verbal }\OperatorTok{+}\StringTok{ }\NormalTok{order}\OperatorTok{:}\NormalTok{spatial }\OperatorTok{+}\StringTok{  }\CommentTok{# level 2}
\StringTok{                }\NormalTok{fam_size_GMC}\OperatorTok{:}\NormalTok{verbal }\OperatorTok{+}\StringTok{ }\NormalTok{fam_size_GMC}\OperatorTok{:}\NormalTok{spatial }\OperatorTok{+}\StringTok{ }\CommentTok{# level 3}
\StringTok{                }\NormalTok{fam_size_GMC}\OperatorTok{:}\NormalTok{order}\OperatorTok{:}\NormalTok{verbal }\OperatorTok{+}\StringTok{ }\NormalTok{fam_size_GMC}\OperatorTok{:}\NormalTok{order}\OperatorTok{:}\NormalTok{spatial }\OperatorTok{+}\StringTok{ }\CommentTok{# level 3}
\StringTok{                }\NormalTok{(}\OperatorTok{-}\DecValTok{1} \OperatorTok{+}\StringTok{ }\NormalTok{verbal }\OperatorTok{+}\StringTok{ }\NormalTok{spatial}\OperatorTok{|}\NormalTok{child_unique) }\OperatorTok{+}
\StringTok{                }\NormalTok{(}\OperatorTok{-}\DecValTok{1} \OperatorTok{+}\StringTok{ }\NormalTok{verbal }\OperatorTok{+}\StringTok{ }\NormalTok{spatial }\OperatorTok{+}\StringTok{ }\NormalTok{order}\OperatorTok{:}\NormalTok{verbal }\OperatorTok{+}\StringTok{ }\NormalTok{order}\OperatorTok{:}\NormalTok{spatial}\OperatorTok{|}\NormalTok{family),}
                \DataTypeTok{data=}\NormalTok{dat) }
\NormalTok{tab1    <-}\StringTok{ }\KeywordTok{table_fun}\NormalTok{(Model_}\DecValTok{1}\NormalTok{)}

\NormalTok{tab1 }\OperatorTok\StringTok{ }\KeywordTok{select}\NormalTok{(}\OperatorTok{-}\NormalTok{type) }\OperatorTok
\StringTok{  }\CommentTok{# mutate(term = str_replace_all(term, "\textbackslash{}\textbackslash{}_", "\textbackslash{}\textbackslash{}\textbackslash{}\textbackslash{}_")) %>%}
\StringTok{  }\KeywordTok{kable}\NormalTok{(., }\StringTok{"latex"}\NormalTok{, }\DataTypeTok{booktabs =}\NormalTok{ T, }\DataTypeTok{escape =}\NormalTok{ F,}
        \DataTypeTok{col.names =} \KeywordTok{c}\NormalTok{(}\StringTok{"Term"}\NormalTok{, }\StringTok{"b"}\NormalTok{, }\StringTok{"CI"}\NormalTok{),}
        \DataTypeTok{caption =} \StringTok{"Question 2: Model 1"}\NormalTok{) }\OperatorTok
\StringTok{  }\KeywordTok{kable_styling}\NormalTok{(}\DataTypeTok{full_width =}\NormalTok{ F) }\OperatorTok
\StringTok{  }\KeywordTok{column_spec}\NormalTok{(}\DecValTok{2}\OperatorTok{:}\DecValTok{3}\NormalTok{, }\DataTypeTok{width =} \StringTok{"2cm"}\NormalTok{) }\OperatorTok
\StringTok{  }\KeywordTok{group_rows}\NormalTok{(}\StringTok{"Fixed Parts"}\NormalTok{,}\DecValTok{1}\NormalTok{,}\DecValTok{8}\NormalTok{) }\OperatorTok
\StringTok{  }\KeywordTok{group_rows}\NormalTok{(}\StringTok{"Random Parts"}\NormalTok{,}\DecValTok{9}\NormalTok{,}\DecValTok{10}\NormalTok{) }\OperatorTok
\StringTok{  }\KeywordTok{group_rows}\NormalTok{(}\StringTok{"Model Terms"}\NormalTok{,}\DecValTok{11}\NormalTok{,}\DecValTok{12}\NormalTok{) }\OperatorTok
\StringTok{  }\KeywordTok{add_header_above}\NormalTok{(}\KeywordTok{c}\NormalTok{(}\StringTok{" "}\NormalTok{ =}\StringTok{ }\DecValTok{1}\NormalTok{, }\StringTok{"Score"}\NormalTok{ =}\StringTok{ }\DecValTok{2}\NormalTok{))}
\end{Highlighting}
\end{Shaded}

\begin{table}

\caption{\label{tab:q2}Question 2: Model 1}
\centering
\begin{tabular}[t]{l>{\raggedright\arraybackslash}p{2cm}>{\raggedright\arraybackslash}p{2cm}}
\toprule
\multicolumn{1}{c}{ } & \multicolumn{2}{c}{Score} \\
\cmidrule(l{2pt}r{2pt}){2-3}
Term & b & CI\\
\midrule
\addlinespace[0.3em]
\multicolumn{3}{l}{\textbf{Fixed Parts}}\\
\hspace{1em}verbal & 30.31 & [30.02, 31.22]\\
\hspace{1em}spatial & 30.27 & [29.77, 30.81]\\
\hspace{1em}verbal:order & -0.93 & [-2.56, -0.83]\\
\hspace{1em}spatial:order & -0.38 & [-1.03, 0.10]\\
\hspace{1em}verbal:fam\_size\_GMC & 0.01 & [-0.32, 0.40]\\
\hspace{1em}spatial:fam\_size\_GMC & -0.06 & [-0.12, 0.74]\\
\hspace{1em}verbal:order:fam\_size\_GMC & -0.00 & [-0.76, 0.27]\\
\hspace{1em}spatial:order:fam\_size\_GMC & 0.14 & [-0.45, 0.34]\\
\addlinespace[0.3em]
\multicolumn{3}{l}{\textbf{Random Parts}}\\
\hspace{1em}$\tau_{00}$ & 9.05 & [7.08, 9.78]\\
\hspace{1em}$\tau_{11}$ & 8.68 & [7.73, 10.40]\\
$R^2_m$ & 0.00 & \\
$R^2_c$ & 0.72 & \\
\bottomrule
\end{tabular}
\end{table}

Based on the model, there is no evidence for an effect of family size.

\section{Question 3}\label{question-3}

Modify the first model (call the modification Model 2) in a way that
will provide an omnibus test for family size effect when the two models
are compared. (Hint: The models should differ by 4 degrees of freedom).
What does this model comparison tell you about the presence of family
size effects in this sample?

\begin{Shaded}
\begin{Highlighting}[]
\NormalTok{Model_}\DecValTok{2}\NormalTok{ <-}\StringTok{ }\KeywordTok{lmer}\NormalTok{(score }\OperatorTok{~}\StringTok{ }
\StringTok{                }\OperatorTok{-}\DecValTok{1} \OperatorTok{+}\StringTok{ }\NormalTok{verbal }\OperatorTok{+}\StringTok{ }\NormalTok{spatial }\OperatorTok{+}\StringTok{         }\CommentTok{# level 1}
\StringTok{                }\NormalTok{order}\OperatorTok{:}\NormalTok{verbal }\OperatorTok{+}\StringTok{ }\NormalTok{order}\OperatorTok{:}\NormalTok{spatial }\OperatorTok{+}\StringTok{  }\CommentTok{# level 2}
\StringTok{                }\NormalTok{(}\OperatorTok{-}\DecValTok{1} \OperatorTok{+}\StringTok{ }\NormalTok{verbal }\OperatorTok{+}\StringTok{ }\NormalTok{spatial}\OperatorTok{|}\NormalTok{child_unique) }\OperatorTok{+}
\StringTok{                }\NormalTok{(}\OperatorTok{-}\DecValTok{1} \OperatorTok{+}\StringTok{ }\NormalTok{verbal }\OperatorTok{+}\StringTok{ }\NormalTok{spatial }\OperatorTok{+}\StringTok{ }\NormalTok{order}\OperatorTok{:}\NormalTok{verbal }\OperatorTok{+}\StringTok{ }\NormalTok{order}\OperatorTok{:}\NormalTok{spatial}\OperatorTok{|}\NormalTok{family),}
                \DataTypeTok{data=}\NormalTok{dat) }
\NormalTok{tab2    <-}\StringTok{ }\KeywordTok{table_fun}\NormalTok{(Model_}\DecValTok{2}\NormalTok{)}

\NormalTok{tab2 }\OperatorTok\StringTok{ }\KeywordTok{select}\NormalTok{(}\OperatorTok{-}\NormalTok{type) }\OperatorTok
\StringTok{  }\CommentTok{# mutate(term = str_replace_all(term, "\textbackslash{}\textbackslash{}_", "\textbackslash{}\textbackslash{}\textbackslash{}\textbackslash{}_")) %>%}
\StringTok{  }\KeywordTok{kable}\NormalTok{(., }\StringTok{"latex"}\NormalTok{, }\DataTypeTok{booktabs =}\NormalTok{ T, }\DataTypeTok{escape =}\NormalTok{ F,}
        \DataTypeTok{col.names =} \KeywordTok{c}\NormalTok{(}\StringTok{"Term"}\NormalTok{, }\StringTok{"b"}\NormalTok{, }\StringTok{"CI"}\NormalTok{),}
        \DataTypeTok{caption =} \StringTok{"Question 3: Model 2"}\NormalTok{) }\OperatorTok
\StringTok{  }\KeywordTok{kable_styling}\NormalTok{(}\DataTypeTok{full_width =}\NormalTok{ F) }\OperatorTok
\StringTok{  }\KeywordTok{column_spec}\NormalTok{(}\DecValTok{2}\OperatorTok{:}\DecValTok{3}\NormalTok{, }\DataTypeTok{width =} \StringTok{"2cm"}\NormalTok{) }\OperatorTok
\StringTok{  }\KeywordTok{group_rows}\NormalTok{(}\StringTok{"Fixed Parts"}\NormalTok{,}\DecValTok{1}\NormalTok{,}\DecValTok{4}\NormalTok{) }\OperatorTok
\StringTok{  }\KeywordTok{group_rows}\NormalTok{(}\StringTok{"Random Parts"}\NormalTok{,}\DecValTok{5}\NormalTok{,}\DecValTok{6}\NormalTok{) }\OperatorTok
\StringTok{  }\KeywordTok{group_rows}\NormalTok{(}\StringTok{"Model Terms"}\NormalTok{,}\DecValTok{7}\NormalTok{,}\DecValTok{8}\NormalTok{) }\OperatorTok
\StringTok{  }\KeywordTok{add_header_above}\NormalTok{(}\KeywordTok{c}\NormalTok{(}\StringTok{" "}\NormalTok{ =}\StringTok{ }\DecValTok{1}\NormalTok{, }\StringTok{"Score"}\NormalTok{ =}\StringTok{ }\DecValTok{2}\NormalTok{))}
\end{Highlighting}
\end{Shaded}

\begin{table}

\caption{\label{tab:q3}Question 3: Model 2}
\centering
\begin{tabular}[t]{l>{\raggedright\arraybackslash}p{2cm}>{\raggedright\arraybackslash}p{2cm}}
\toprule
\multicolumn{1}{c}{ } & \multicolumn{2}{c}{Score} \\
\cmidrule(l{2pt}r{2pt}){2-3}
Term & b & CI\\
\midrule
\addlinespace[0.3em]
\multicolumn{3}{l}{\textbf{Fixed Parts}}\\
\hspace{1em}verbal & 30.30 & [29.07, 30.92]\\
\hspace{1em}spatial & 30.27 & [29.50, 31.03]\\
\hspace{1em}verbal:order & -0.93 & [-1.31, 0.86]\\
\hspace{1em}spatial:order & -0.40 & [-1.29, 0.68]\\
\addlinespace[0.3em]
\multicolumn{3}{l}{\textbf{Random Parts}}\\
\hspace{1em}$\tau_{00}$ & 9.05 & [7.43, 10.12]\\
\hspace{1em}$\tau_{11}$ & 8.66 & [6.96, 8.97]\\
$R^2_m$ & 0.00 & \\
$R^2_c$ & 0.72 & \\
\bottomrule
\end{tabular}
\end{table}

\begin{Shaded}
\begin{Highlighting}[]
\KeywordTok{anova}\NormalTok{(Model_}\DecValTok{1}\NormalTok{, Model_}\DecValTok{2}\NormalTok{)}
\end{Highlighting}
\end{Shaded}

\begin{verbatim}
## Data: dat
## Models:
## Model_2: score ~ -1 + verbal + spatial + order:verbal + order:spatial + 
## Model_2:     (-1 + verbal + spatial | child_unique) + (-1 + verbal + spatial + 
## Model_2:     order:verbal + order:spatial | family)
## Model_1: score ~ -1 + verbal + spatial + order:verbal + order:spatial + 
## Model_1:     fam_size_GMC:verbal + fam_size_GMC:spatial + fam_size_GMC:order:verbal + 
## Model_1:     fam_size_GMC:order:spatial + (-1 + verbal + spatial | child_unique) + 
## Model_1:     (-1 + verbal + spatial + order:verbal + order:spatial | family)
##         Df   AIC   BIC  logLik deviance  Chisq Chi Df Pr(>Chisq)
## Model_2 18 12976 13080 -6470.0    12940                         
## Model_1 22 12984 13111 -6469.8    12940 0.3025      4     0.9897
\end{verbatim}

There appear to be no family effects in this sample.

\subsection{Question 4}\label{question-4}

Now modify Model 2 to remove all birth order effects (call the new model
Model 3) and conduct a model comparison to Model 2.

\begin{Shaded}
\begin{Highlighting}[]
\NormalTok{Model_}\DecValTok{3}\NormalTok{ <-}\StringTok{ }\KeywordTok{lmer}\NormalTok{(score }\OperatorTok{~}\StringTok{ }
\StringTok{                }\OperatorTok{-}\DecValTok{1} \OperatorTok{+}\StringTok{ }\NormalTok{verbal }\OperatorTok{+}\StringTok{ }\NormalTok{spatial }\OperatorTok{+}\StringTok{ }\CommentTok{# level 1}
\StringTok{                }\NormalTok{(}\OperatorTok{-}\DecValTok{1} \OperatorTok{+}\StringTok{ }\NormalTok{verbal }\OperatorTok{+}\StringTok{ }\NormalTok{spatial }\OperatorTok{|}\StringTok{ }\NormalTok{child_unique) }\OperatorTok{+}
\StringTok{                }\NormalTok{(}\OperatorTok{-}\DecValTok{1} \OperatorTok{+}\StringTok{ }\NormalTok{verbal }\OperatorTok{+}\StringTok{ }\NormalTok{spatial }\OperatorTok{|}\StringTok{ }\NormalTok{family),}
                \DataTypeTok{data=}\NormalTok{dat) }
\NormalTok{tab3    <-}\StringTok{ }\KeywordTok{table_fun}\NormalTok{(Model_}\DecValTok{3}\NormalTok{)}

\NormalTok{tab3 }\OperatorTok\StringTok{ }\KeywordTok{select}\NormalTok{(}\OperatorTok{-}\NormalTok{type) }\OperatorTok
\StringTok{  }\CommentTok{# mutate(term = str_replace_all(term, "\textbackslash{}\textbackslash{}_", "\textbackslash{}\textbackslash{}\textbackslash{}\textbackslash{}_")) %>%}
\StringTok{  }\KeywordTok{kable}\NormalTok{(., }\StringTok{"latex"}\NormalTok{, }\DataTypeTok{booktabs =}\NormalTok{ T, }\DataTypeTok{escape =}\NormalTok{ F,}
        \DataTypeTok{col.names =} \KeywordTok{c}\NormalTok{(}\StringTok{"Term"}\NormalTok{, }\StringTok{"b"}\NormalTok{, }\StringTok{"CI"}\NormalTok{),}
        \DataTypeTok{caption =} \StringTok{"Question 4: Model 3"}\NormalTok{) }\OperatorTok
\StringTok{  }\KeywordTok{kable_styling}\NormalTok{(}\DataTypeTok{full_width =}\NormalTok{ F) }\OperatorTok
\StringTok{  }\KeywordTok{column_spec}\NormalTok{(}\DecValTok{2}\OperatorTok{:}\DecValTok{3}\NormalTok{, }\DataTypeTok{width =} \StringTok{"2cm"}\NormalTok{) }\OperatorTok
\StringTok{  }\KeywordTok{group_rows}\NormalTok{(}\StringTok{"Fixed Parts"}\NormalTok{,}\DecValTok{1}\NormalTok{,}\DecValTok{2}\NormalTok{) }\OperatorTok
\StringTok{  }\KeywordTok{group_rows}\NormalTok{(}\StringTok{"Random Parts"}\NormalTok{,}\DecValTok{3}\NormalTok{,}\DecValTok{4}\NormalTok{) }\OperatorTok
\StringTok{  }\KeywordTok{group_rows}\NormalTok{(}\StringTok{"Model Terms"}\NormalTok{,}\DecValTok{5}\NormalTok{,}\DecValTok{6}\NormalTok{) }\OperatorTok
\StringTok{  }\KeywordTok{add_header_above}\NormalTok{(}\KeywordTok{c}\NormalTok{(}\StringTok{" "}\NormalTok{ =}\StringTok{ }\DecValTok{1}\NormalTok{, }\StringTok{"Score"}\NormalTok{ =}\StringTok{ }\DecValTok{2}\NormalTok{))}
\end{Highlighting}
\end{Shaded}

\begin{table}

\caption{\label{tab:q4}Question 4: Model 3}
\centering
\begin{tabular}[t]{l>{\raggedright\arraybackslash}p{2cm}>{\raggedright\arraybackslash}p{2cm}}
\toprule
\multicolumn{1}{c}{ } & \multicolumn{2}{c}{Score} \\
\cmidrule(l{2pt}r{2pt}){2-3}
Term & b & CI\\
\midrule
\addlinespace[0.3em]
\multicolumn{3}{l}{\textbf{Fixed Parts}}\\
\hspace{1em}verbal & 29.84 & [29.11, 30.26]\\
\hspace{1em}spatial & 30.07 & [29.69, 30.31]\\
\addlinespace[0.3em]
\multicolumn{3}{l}{\textbf{Random Parts}}\\
\hspace{1em}$\tau_{00}$ & 9.22 & [7.86, 11.91]\\
\hspace{1em}$\tau_{11}$ & 8.71 & [7.19, 10.76]\\
$R^2_m$ & 0.00 & \\
$R^2_c$ & 0.72 & \\
\bottomrule
\end{tabular}
\end{table}

\begin{Shaded}
\begin{Highlighting}[]
\KeywordTok{anova}\NormalTok{(Model_}\DecValTok{3}\NormalTok{, Model_}\DecValTok{2}\NormalTok{)}
\end{Highlighting}
\end{Shaded}

\begin{verbatim}
## Data: dat
## Models:
## Model_3: score ~ -1 + verbal + spatial + (-1 + verbal + spatial | child_unique) + 
## Model_3:     (-1 + verbal + spatial | family)
## Model_2: score ~ -1 + verbal + spatial + order:verbal + order:spatial + 
## Model_2:     (-1 + verbal + spatial | child_unique) + (-1 + verbal + spatial + 
## Model_2:     order:verbal + order:spatial | family)
##         Df   AIC   BIC  logLik deviance Chisq Chi Df Pr(>Chisq)
## Model_3  9 12964 13016 -6472.9    12946                        
## Model_2 18 12976 13080 -6470.0    12940   5.9      9     0.7499
\end{verbatim}

\subsection{Part A}\label{part-a}

What can you conclude from this comparison? There do not appear to be
any birth order effects.

\subsection{Part B}\label{part-b}

Compare Model 3 to Model 1. What is being tested and what does it offer
beyond the previous two model comparisons?

\begin{Shaded}
\begin{Highlighting}[]
\KeywordTok{anova}\NormalTok{(Model_}\DecValTok{3}\NormalTok{, Model_}\DecValTok{1}\NormalTok{)}
\end{Highlighting}
\end{Shaded}

\begin{verbatim}
## Data: dat
## Models:
## Model_3: score ~ -1 + verbal + spatial + (-1 + verbal + spatial | child_unique) + 
## Model_3:     (-1 + verbal + spatial | family)
## Model_1: score ~ -1 + verbal + spatial + order:verbal + order:spatial + 
## Model_1:     fam_size_GMC:verbal + fam_size_GMC:spatial + fam_size_GMC:order:verbal + 
## Model_1:     fam_size_GMC:order:spatial + (-1 + verbal + spatial | child_unique) + 
## Model_1:     (-1 + verbal + spatial + order:verbal + order:spatial | family)
##         Df   AIC   BIC  logLik deviance  Chisq Chi Df Pr(>Chisq)
## Model_3  9 12964 13016 -6472.9    12946                         
## Model_1 22 12984 13111 -6469.8    12940 6.2025     13     0.9385
\end{verbatim}

This is testing whether a model that only estimates mean verbal and
spatial scores, as well as unique means for each family and child within
a family is a better model than one that adjusts the estimated means
based on birth order and the number of children in the family. Neither
of these seem to impact mean verbal or spatial ability.

\section{Question 5}\label{question-5}

Using the most parsimonious of the three previous models, modify it
(call it Model 4) to provide a model comparison that tests the
significance of the true score correlation between verbal and spatial
abilities at the level of the child.

\begin{Shaded}
\begin{Highlighting}[]
\NormalTok{Model_}\DecValTok{4}\NormalTok{ <-}\StringTok{ }\KeywordTok{lmer}\NormalTok{(score }\OperatorTok{~}\StringTok{ }
\StringTok{                }\OperatorTok{-}\DecValTok{1} \OperatorTok{+}\StringTok{ }\NormalTok{verbal }\OperatorTok{+}\StringTok{ }\NormalTok{spatial }\OperatorTok{+}\StringTok{ }\CommentTok{# level 1}
\StringTok{                }\NormalTok{(}\OperatorTok{-}\DecValTok{1} \OperatorTok{+}\StringTok{ }\NormalTok{verbal }\OperatorTok{+}\StringTok{ }\NormalTok{spatial }\OperatorTok{||}\StringTok{ }\NormalTok{child_unique) }\OperatorTok{+}
\StringTok{                }\NormalTok{(}\OperatorTok{-}\DecValTok{1} \OperatorTok{+}\StringTok{ }\NormalTok{verbal }\OperatorTok{+}\StringTok{ }\NormalTok{spatial }\OperatorTok{|}\StringTok{ }\NormalTok{family),}
                \DataTypeTok{data=}\NormalTok{dat) }

\NormalTok{tab4    <-}\StringTok{ }\KeywordTok{table_fun}\NormalTok{(Model_}\DecValTok{4}\NormalTok{)}
\NormalTok{tab4 }\OperatorTok\StringTok{ }\KeywordTok{select}\NormalTok{(}\OperatorTok{-}\NormalTok{type) }\OperatorTok
\StringTok{  }\CommentTok{# mutate(term = str_replace_all(term, "\textbackslash{}\textbackslash{}_", "\textbackslash{}\textbackslash{}\textbackslash{}\textbackslash{}_")) %>%}
\StringTok{  }\KeywordTok{kable}\NormalTok{(., }\StringTok{"latex"}\NormalTok{, }\DataTypeTok{booktabs =}\NormalTok{ T, }\DataTypeTok{escape =}\NormalTok{ F,}
        \DataTypeTok{col.names =} \KeywordTok{c}\NormalTok{(}\StringTok{"Term"}\NormalTok{, }\StringTok{"b"}\NormalTok{, }\StringTok{"CI"}\NormalTok{),}
        \DataTypeTok{caption =} \StringTok{"Question 4: Model 3"}\NormalTok{) }\OperatorTok
\StringTok{  }\KeywordTok{kable_styling}\NormalTok{(}\DataTypeTok{full_width =}\NormalTok{ F) }\OperatorTok
\StringTok{  }\KeywordTok{column_spec}\NormalTok{(}\DecValTok{2}\OperatorTok{:}\DecValTok{3}\NormalTok{, }\DataTypeTok{width =} \StringTok{"2cm"}\NormalTok{) }\OperatorTok
\StringTok{  }\KeywordTok{group_rows}\NormalTok{(}\StringTok{"Fixed Parts"}\NormalTok{,}\DecValTok{1}\NormalTok{,}\DecValTok{2}\NormalTok{) }\OperatorTok
\StringTok{  }\KeywordTok{group_rows}\NormalTok{(}\StringTok{"Random Parts"}\NormalTok{,}\DecValTok{3}\NormalTok{,}\DecValTok{3}\NormalTok{) }\OperatorTok
\StringTok{  }\KeywordTok{group_rows}\NormalTok{(}\StringTok{"Model Terms"}\NormalTok{,}\DecValTok{4}\NormalTok{,}\DecValTok{5}\NormalTok{) }\OperatorTok
\StringTok{  }\KeywordTok{add_header_above}\NormalTok{(}\KeywordTok{c}\NormalTok{(}\StringTok{" "}\NormalTok{ =}\StringTok{ }\DecValTok{1}\NormalTok{, }\StringTok{"Score"}\NormalTok{ =}\StringTok{ }\DecValTok{2}\NormalTok{))}
\end{Highlighting}
\end{Shaded}

\begin{table}

\caption{\label{tab:unnamed-chunk-4}Question 4: Model 3}
\centering
\begin{tabular}[t]{l>{\raggedright\arraybackslash}p{2cm}>{\raggedright\arraybackslash}p{2cm}}
\toprule
\multicolumn{1}{c}{ } & \multicolumn{2}{c}{Score} \\
\cmidrule(l{2pt}r{2pt}){2-3}
Term & b & CI\\
\midrule
\addlinespace[0.3em]
\multicolumn{3}{l}{\textbf{Fixed Parts}}\\
\hspace{1em}verbal & 29.83 & [29.31, 30.48]\\
\hspace{1em}spatial & 30.07 & [29.58, 30.70]\\
\addlinespace[0.3em]
\multicolumn{3}{l}{\textbf{Random Parts}}\\
\hspace{1em}$\tau_{00}$ & 9.21 & [8.42, 10.09]\\
$R^2_m$ & 0.00 & \\
$R^2_c$ & 0.72 & \\
\bottomrule
\end{tabular}
\end{table}

\begin{Shaded}
\begin{Highlighting}[]
\NormalTok{t_cor <-}\StringTok{ }\NormalTok{(}\KeywordTok{sigma}\NormalTok{(Model_}\DecValTok{4}\NormalTok{)}\OperatorTok{^}\DecValTok{2} \OperatorTok{-}\StringTok{ }\KeywordTok{sigma}\NormalTok{(Model_}\DecValTok{3}\NormalTok{)}\OperatorTok{^}\DecValTok{2}\NormalTok{)}\OperatorTok{/}\KeywordTok{sigma}\NormalTok{(Model_}\DecValTok{4}\NormalTok{)}\OperatorTok{^}\DecValTok{2}
\end{Highlighting}
\end{Shaded}

\begin{Shaded}
\begin{Highlighting}[]
\KeywordTok{anova}\NormalTok{(Model_}\DecValTok{4}\NormalTok{, Model_}\DecValTok{3}\NormalTok{)}
\end{Highlighting}
\end{Shaded}

\begin{verbatim}
## Data: dat
## Models:
## Model_4: score ~ -1 + verbal + spatial + ((0 + verbal | child_unique) + 
## Model_4:     (0 + spatial | child_unique)) + (-1 + verbal + spatial | 
## Model_4:     family)
## Model_3: score ~ -1 + verbal + spatial + (-1 + verbal + spatial | child_unique) + 
## Model_3:     (-1 + verbal + spatial | family)
##         Df   AIC   BIC  logLik deviance Chisq Chi Df Pr(>Chisq)    
## Model_4  8 12996 13042 -6490.0    12980                            
## Model_3  9 12964 13016 -6472.9    12946 34.12      1  5.183e-09 ***
## ---
## Signif. codes:  0 '***' 0.001 '**' 0.01 '*' 0.05 '.' 0.1 ' ' 1
\end{verbatim}

The true score correlation is 0.


\end{document}
