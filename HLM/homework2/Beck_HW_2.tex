\documentclass[]{article}
\usepackage{lmodern}
\usepackage{amssymb,amsmath}
\usepackage{ifxetex,ifluatex}
\usepackage{fixltx2e} % provides \textsubscript
\ifnum 0\ifxetex 1\fi\ifluatex 1\fi=0 % if pdftex
  \usepackage[T1]{fontenc}
  \usepackage[utf8]{inputenc}
\else % if luatex or xelatex
  \ifxetex
    \usepackage{mathspec}
  \else
    \usepackage{fontspec}
  \fi
  \defaultfontfeatures{Ligatures=TeX,Scale=MatchLowercase}
\fi
% use upquote if available, for straight quotes in verbatim environments
\IfFileExists{upquote.sty}{\usepackage{upquote}}{}
% use microtype if available
\IfFileExists{microtype.sty}{%
\usepackage{microtype}
\UseMicrotypeSet[protrusion]{basicmath} % disable protrusion for tt fonts
}{}
\usepackage[margin=1in]{geometry}
\usepackage{hyperref}
\hypersetup{unicode=true,
            pdftitle={Homework 2},
            pdfauthor={Emorie Beck},
            pdfborder={0 0 0},
            breaklinks=true}
\urlstyle{same}  % don't use monospace font for urls
\usepackage{color}
\usepackage{fancyvrb}
\newcommand{\VerbBar}{|}
\newcommand{\VERB}{\Verb[commandchars=\\\{\}]}
\DefineVerbatimEnvironment{Highlighting}{Verbatim}{commandchars=\\\{\}}
% Add ',fontsize=\small' for more characters per line
\usepackage{framed}
\definecolor{shadecolor}{RGB}{248,248,248}
\newenvironment{Shaded}{\begin{snugshade}}{\end{snugshade}}
\newcommand{\KeywordTok}[1]{\textcolor[rgb]{0.13,0.29,0.53}{\textbf{#1}}}
\newcommand{\DataTypeTok}[1]{\textcolor[rgb]{0.13,0.29,0.53}{#1}}
\newcommand{\DecValTok}[1]{\textcolor[rgb]{0.00,0.00,0.81}{#1}}
\newcommand{\BaseNTok}[1]{\textcolor[rgb]{0.00,0.00,0.81}{#1}}
\newcommand{\FloatTok}[1]{\textcolor[rgb]{0.00,0.00,0.81}{#1}}
\newcommand{\ConstantTok}[1]{\textcolor[rgb]{0.00,0.00,0.00}{#1}}
\newcommand{\CharTok}[1]{\textcolor[rgb]{0.31,0.60,0.02}{#1}}
\newcommand{\SpecialCharTok}[1]{\textcolor[rgb]{0.00,0.00,0.00}{#1}}
\newcommand{\StringTok}[1]{\textcolor[rgb]{0.31,0.60,0.02}{#1}}
\newcommand{\VerbatimStringTok}[1]{\textcolor[rgb]{0.31,0.60,0.02}{#1}}
\newcommand{\SpecialStringTok}[1]{\textcolor[rgb]{0.31,0.60,0.02}{#1}}
\newcommand{\ImportTok}[1]{#1}
\newcommand{\CommentTok}[1]{\textcolor[rgb]{0.56,0.35,0.01}{\textit{#1}}}
\newcommand{\DocumentationTok}[1]{\textcolor[rgb]{0.56,0.35,0.01}{\textbf{\textit{#1}}}}
\newcommand{\AnnotationTok}[1]{\textcolor[rgb]{0.56,0.35,0.01}{\textbf{\textit{#1}}}}
\newcommand{\CommentVarTok}[1]{\textcolor[rgb]{0.56,0.35,0.01}{\textbf{\textit{#1}}}}
\newcommand{\OtherTok}[1]{\textcolor[rgb]{0.56,0.35,0.01}{#1}}
\newcommand{\FunctionTok}[1]{\textcolor[rgb]{0.00,0.00,0.00}{#1}}
\newcommand{\VariableTok}[1]{\textcolor[rgb]{0.00,0.00,0.00}{#1}}
\newcommand{\ControlFlowTok}[1]{\textcolor[rgb]{0.13,0.29,0.53}{\textbf{#1}}}
\newcommand{\OperatorTok}[1]{\textcolor[rgb]{0.81,0.36,0.00}{\textbf{#1}}}
\newcommand{\BuiltInTok}[1]{#1}
\newcommand{\ExtensionTok}[1]{#1}
\newcommand{\PreprocessorTok}[1]{\textcolor[rgb]{0.56,0.35,0.01}{\textit{#1}}}
\newcommand{\AttributeTok}[1]{\textcolor[rgb]{0.77,0.63,0.00}{#1}}
\newcommand{\RegionMarkerTok}[1]{#1}
\newcommand{\InformationTok}[1]{\textcolor[rgb]{0.56,0.35,0.01}{\textbf{\textit{#1}}}}
\newcommand{\WarningTok}[1]{\textcolor[rgb]{0.56,0.35,0.01}{\textbf{\textit{#1}}}}
\newcommand{\AlertTok}[1]{\textcolor[rgb]{0.94,0.16,0.16}{#1}}
\newcommand{\ErrorTok}[1]{\textcolor[rgb]{0.64,0.00,0.00}{\textbf{#1}}}
\newcommand{\NormalTok}[1]{#1}
\usepackage{graphicx,grffile}
\makeatletter
\def\maxwidth{\ifdim\Gin@nat@width>\linewidth\linewidth\else\Gin@nat@width\fi}
\def\maxheight{\ifdim\Gin@nat@height>\textheight\textheight\else\Gin@nat@height\fi}
\makeatother
% Scale images if necessary, so that they will not overflow the page
% margins by default, and it is still possible to overwrite the defaults
% using explicit options in \includegraphics[width, height, ...]{}
\setkeys{Gin}{width=\maxwidth,height=\maxheight,keepaspectratio}
\IfFileExists{parskip.sty}{%
\usepackage{parskip}
}{% else
\setlength{\parindent}{0pt}
\setlength{\parskip}{6pt plus 2pt minus 1pt}
}
\setlength{\emergencystretch}{3em}  % prevent overfull lines
\providecommand{\tightlist}{%
  \setlength{\itemsep}{0pt}\setlength{\parskip}{0pt}}
\setcounter{secnumdepth}{0}
% Redefines (sub)paragraphs to behave more like sections
\ifx\paragraph\undefined\else
\let\oldparagraph\paragraph
\renewcommand{\paragraph}[1]{\oldparagraph{#1}\mbox{}}
\fi
\ifx\subparagraph\undefined\else
\let\oldsubparagraph\subparagraph
\renewcommand{\subparagraph}[1]{\oldsubparagraph{#1}\mbox{}}
\fi

%%% Use protect on footnotes to avoid problems with footnotes in titles
\let\rmarkdownfootnote\footnote%
\def\footnote{\protect\rmarkdownfootnote}

%%% Change title format to be more compact
\usepackage{titling}

% Create subtitle command for use in maketitle
\newcommand{\subtitle}[1]{
  \posttitle{
    \begin{center}\large#1\end{center}
    }
}

\setlength{\droptitle}{-2em}
  \title{Homework 2}
  \pretitle{\vspace{\droptitle}\centering\huge}
  \posttitle{\par}
  \author{Emorie Beck}
  \preauthor{\centering\large\emph}
  \postauthor{\par}
  \predate{\centering\large\emph}
  \postdate{\par}
  \date{\today}

\usepackage{fancyhdr}
\usepackage{array}
\usepackage{longtable}
\usepackage{lscape}
\newcommand{\blandscape}{\begin{landscape}}
\newcommand{\elandscape}{\end{landscape}}
\usepackage{dcolumn}
\usepackage{bbm}
\usepackage{threeparttable}
\usepackage{booktabs}
\usepackage{expex}
\usepackage{rotating, graphicx}
\usepackage{tabulary}
\usepackage{algorithm}
\usepackage{multirow}
\usepackage{colortbl}
\usepackage{longtable}
\usepackage{array}
\usepackage{multirow}
\usepackage[table]{xcolor}
\usepackage{wrapfig}
\usepackage{float}
\usepackage{pdflscape}
\usepackage{tabu}
\usepackage{threeparttable}
\usepackage{booktabs}
\usepackage{longtable}
\usepackage{array}
\usepackage{multirow}
\usepackage[table]{xcolor}
\usepackage{wrapfig}
\usepackage{float}
\usepackage{colortbl}
\usepackage{pdflscape}
\usepackage{tabu}
\usepackage{threeparttable}
\usepackage[normalem]{ulem}

\begin{document}
\maketitle

{
\setcounter{tocdepth}{2}
\tableofcontents
}
\section{Workspace}\label{workspace}

\subsection{Packages}\label{packages}

\begin{Shaded}
\begin{Highlighting}[]
\KeywordTok{library}\NormalTok{(psych)}
\KeywordTok{library}\NormalTok{(lme4)}
\KeywordTok{library}\NormalTok{(knitr)}
\KeywordTok{library}\NormalTok{(kableExtra)}
\KeywordTok{library}\NormalTok{(plyr)}
\KeywordTok{library}\NormalTok{(tidyverse)}
\end{Highlighting}
\end{Shaded}

\subsection{Data}\label{data}

\begin{Shaded}
\begin{Highlighting}[]
\NormalTok{data_url <-}\StringTok{ "https://raw.githubusercontent.com/emoriebeck/homeworks/master/homeowrk2/HSB.csv"}
\NormalTok{dat      <-}\StringTok{ }\KeywordTok{read.csv}\NormalTok{(}\KeywordTok{url}\NormalTok{(data_url)) }\OperatorTok\StringTok{ }\NormalTok{tbl_df }\OperatorTok
\StringTok{  }\KeywordTok{mutate}\NormalTok{(}\DataTypeTok{female =} \KeywordTok{factor}\NormalTok{(female, }\DataTypeTok{levels =} \DecValTok{0}\OperatorTok{:}\DecValTok{1}\NormalTok{, }\DataTypeTok{labels =} \KeywordTok{c}\NormalTok{(}\StringTok{"Male"}\NormalTok{, }\StringTok{"Female"}\NormalTok{)),}
         \DataTypeTok{minority =} \KeywordTok{factor}\NormalTok{(minority, }\DataTypeTok{levels =} \DecValTok{0}\OperatorTok{:}\DecValTok{1}\NormalTok{, }\DataTypeTok{labels =} \KeywordTok{c}\NormalTok{(}\StringTok{"Non-Minority"}\NormalTok{,}\StringTok{"Minority"}\NormalTok{)),}
         \DataTypeTok{sector =} \KeywordTok{factor}\NormalTok{(sector, }\DataTypeTok{levels =} \DecValTok{0}\OperatorTok{:}\DecValTok{1}\NormalTok{, }\DataTypeTok{labels =} \KeywordTok{c}\NormalTok{(}\StringTok{"Public"}\NormalTok{, }\StringTok{"Catholic"}\NormalTok{)))}
\end{Highlighting}
\end{Shaded}

\subsection{Local Functions}\label{local-functions}

I don't like stargazer or the other packages that are supposed to create
tables from merMod objects, so I wrote my own. The only feature I
haven't implemented yet is including the ICC becuase I would need to
refit the unconditional model.

\begin{Shaded}
\begin{Highlighting}[]
\NormalTok{table_fun <-}\StringTok{ }\ControlFlowTok{function}\NormalTok{(model)\{}
\NormalTok{  fixed <-}\StringTok{ }\NormalTok{broom}\OperatorTok{::}\KeywordTok{tidy}\NormalTok{(model) }\OperatorTok\StringTok{ }\KeywordTok{filter}\NormalTok{(group }\OperatorTok{==}\StringTok{ "fixed"}\NormalTok{) }\OperatorTok
\StringTok{    }\KeywordTok{select}\NormalTok{(term, estimate) }
\NormalTok{  ## add random effects ##}
\NormalTok{  rand <-}\StringTok{ }\KeywordTok{VarCorr}\NormalTok{(model)[[}\DecValTok{1}\NormalTok{]]}
\NormalTok{  rand <-}\StringTok{ }\NormalTok{rand[}\DecValTok{1}\OperatorTok{:}\KeywordTok{nrow}\NormalTok{(rand), }\DecValTok{1}\OperatorTok{:}\KeywordTok{nrow}\NormalTok{(rand)]}
  \KeywordTok{colnames}\NormalTok{(rand)[}\KeywordTok{colnames}\NormalTok{(rand) }\OperatorTok{==}\StringTok{ "(Intercept)"}\NormalTok{] <-}\StringTok{ "Intercept"}
  \KeywordTok{rownames}\NormalTok{(rand)[}\KeywordTok{rownames}\NormalTok{(rand) }\OperatorTok{==}\StringTok{ "(Intercept)"}\NormalTok{] <-}\StringTok{ "Intercept"}
\NormalTok{  vars <-}\StringTok{ }\KeywordTok{rownames}\NormalTok{(rand)}
\NormalTok{  rand[}\KeywordTok{upper.tri}\NormalTok{(rand)] <-}\StringTok{ }\OtherTok{NA}
\NormalTok{  rand <-}\StringTok{ }\KeywordTok{data.frame}\NormalTok{(rand) }\OperatorTok\StringTok{ }\KeywordTok{mutate}\NormalTok{(}\DataTypeTok{var1 =} \KeywordTok{rownames}\NormalTok{(.)) }\OperatorTok
\StringTok{    }\KeywordTok{gather}\NormalTok{(}\DataTypeTok{key =}\NormalTok{ var2, }\DataTypeTok{value =}\NormalTok{ estimate, }\OperatorTok{-}\NormalTok{var1, }\DataTypeTok{na.rm =}\NormalTok{ T) }\OperatorTok
\StringTok{    }\KeywordTok{mutate}\NormalTok{(}\DataTypeTok{var1 =} \KeywordTok{mapvalues}\NormalTok{(var1, vars, }\DecValTok{0}\OperatorTok{:}\NormalTok{(}\KeywordTok{length}\NormalTok{(vars)}\OperatorTok{-}\DecValTok{1}\NormalTok{)),}
           \DataTypeTok{var2 =} \KeywordTok{mapvalues}\NormalTok{(var2, vars, }\DecValTok{0}\OperatorTok{:}\NormalTok{(}\KeywordTok{length}\NormalTok{(vars)}\OperatorTok{-}\DecValTok{1}\NormalTok{))) }\OperatorTok
\StringTok{    }\KeywordTok{filter}\NormalTok{(var1 }\OperatorTok{==}\StringTok{ }\NormalTok{var2) }\OperatorTok
\StringTok{    }\KeywordTok{unite}\NormalTok{(var, var1, var2, }\DataTypeTok{sep =} \StringTok{""}\NormalTok{) }\OperatorTok
\StringTok{    }\KeywordTok{mutate}\NormalTok{(}\DataTypeTok{var =} \KeywordTok{sprintf}\NormalTok{(}\StringTok{"$}\CharTok{\textbackslash{}\textbackslash{}}\StringTok{tau_\{%s\}$"}\NormalTok{, var))}
\NormalTok{  ## get confidence intervals ##}
\NormalTok{  CI <-}\StringTok{ }\KeywordTok{data.frame}\NormalTok{(}\KeywordTok{confint.merMod}\NormalTok{(model, }\DataTypeTok{method =} \StringTok{"boot"}\NormalTok{, }\DataTypeTok{nsim =} \DecValTok{10}\NormalTok{, }\DataTypeTok{oldNames =}\NormalTok{ F)) }\OperatorTok
\StringTok{    }\KeywordTok{mutate}\NormalTok{(}\DataTypeTok{term =} \KeywordTok{rownames}\NormalTok{(.)) }\OperatorTok\StringTok{ }\KeywordTok{setNames}\NormalTok{(}\KeywordTok{c}\NormalTok{(}\StringTok{"lower"}\NormalTok{, }\StringTok{"upper"}\NormalTok{, }\StringTok{"term"}\NormalTok{))}
  
\NormalTok{  CI }\OperatorTok\StringTok{ }\KeywordTok{filter}\NormalTok{(term }\OperatorTok{==}\StringTok{ "sigma"}\NormalTok{) }\OperatorTok
\StringTok{    }\KeywordTok{mutate}\NormalTok{(}\DataTypeTok{estimate =} \KeywordTok{sigma}\NormalTok{(model),}
           \DataTypeTok{term =} \StringTok{"$}\CharTok{\textbackslash{}\textbackslash{}}\StringTok{sigma^2$"}\NormalTok{,}
           \DataTypeTok{type =} \StringTok{"Residuals"}\NormalTok{)}
  
\NormalTok{  ## Get ICC & R2 values ##}
\NormalTok{  ICC <-}\StringTok{ }\NormalTok{reghelper}\OperatorTok{::}\KeywordTok{ICC}\NormalTok{(model)}
\NormalTok{  R2 <-}\StringTok{ }\NormalTok{MuMIn}\OperatorTok{::}\KeywordTok{r.squaredGLMM}\NormalTok{(model)}
  
\NormalTok{  ## format the fixed effects}
\NormalTok{  fixed <-}\StringTok{ }\NormalTok{fixed }\OperatorTok\StringTok{ }\KeywordTok{left_join}\NormalTok{(CI }\OperatorTok\StringTok{ }\KeywordTok{filter}\NormalTok{(}\OperatorTok{!}\KeywordTok{grepl}\NormalTok{(}\StringTok{".sig"}\NormalTok{, term))) }\OperatorTok
\StringTok{    }\KeywordTok{mutate}\NormalTok{(}\DataTypeTok{type =} \StringTok{"Fixed Parts"}\NormalTok{)}
  
\NormalTok{  rand <-}\StringTok{ }\NormalTok{rand }\OperatorTok
\StringTok{    }\KeywordTok{left_join}\NormalTok{(}
\NormalTok{      CI }\OperatorTok\StringTok{ }\KeywordTok{filter}\NormalTok{(}\KeywordTok{grepl}\NormalTok{(}\StringTok{"sd"}\NormalTok{, term)) }\OperatorTok
\StringTok{        }\KeywordTok{mutate}\NormalTok{(}\DataTypeTok{lower =}\NormalTok{ lower}\OperatorTok{^}\DecValTok{2}\NormalTok{, }\DataTypeTok{upper =}\NormalTok{ upper}\OperatorTok{^}\DecValTok{2}\NormalTok{,}
               \DataTypeTok{var =} \KeywordTok{mapvalues}\NormalTok{(term, }\KeywordTok{unique}\NormalTok{(term), }\DecValTok{0}\OperatorTok{:}\NormalTok{(}\KeywordTok{length}\NormalTok{(}\KeywordTok{unique}\NormalTok{(term))}\OperatorTok{-}\DecValTok{1}\NormalTok{)),}
               \DataTypeTok{var =} \KeywordTok{sprintf}\NormalTok{(}\StringTok{"$}\CharTok{\textbackslash{}\textbackslash{}}\StringTok{tau_\{%s%s\}$"}\NormalTok{, var, var)) }\OperatorTok\StringTok{ }\KeywordTok{select}\NormalTok{(}\OperatorTok{-}\NormalTok{term)) }\OperatorTok
\StringTok{    }\KeywordTok{mutate}\NormalTok{(}\DataTypeTok{type =} \StringTok{"Random Parts"}\NormalTok{) }\OperatorTok\StringTok{ }\KeywordTok{rename}\NormalTok{(}\DataTypeTok{term =}\NormalTok{ var)}
  
\NormalTok{  mod_terms <-}\StringTok{ }\KeywordTok{tribble}\NormalTok{(}
    \OperatorTok{~}\NormalTok{term, }\OperatorTok{~}\NormalTok{estimate, }\OperatorTok{~}\NormalTok{type,}
    \CommentTok{# "ICC", ICC, "Model Terms",}
    \StringTok{"$R^2_m$"}\NormalTok{, R2[}\DecValTok{1}\NormalTok{], }\StringTok{"Model Terms"}\NormalTok{,}
    \StringTok{"$R^2_c$"}\NormalTok{, R2[}\DecValTok{2}\NormalTok{], }\StringTok{"Model Terms"}
\NormalTok{  )}
  
\NormalTok{  tab <-}\StringTok{ }\NormalTok{fixed }\OperatorTok
\StringTok{    }\KeywordTok{full_join}\NormalTok{(rand) }\OperatorTok
\StringTok{    }\KeywordTok{mutate}\NormalTok{(}\DataTypeTok{CI =} \KeywordTok{sprintf}\NormalTok{(}\StringTok{"[%.2f, %.2f]"}\NormalTok{, lower, upper)) }\OperatorTok
\StringTok{    }\KeywordTok{select}\NormalTok{(}\OperatorTok{-}\NormalTok{lower, }\OperatorTok{-}\NormalTok{upper) }\OperatorTok
\StringTok{    }\KeywordTok{full_join}\NormalTok{(mod_terms) }\OperatorTok
\StringTok{    }\KeywordTok{mutate}\NormalTok{(}\DataTypeTok{estimate =} \KeywordTok{sprintf}\NormalTok{(}\StringTok{"%.2f"}\NormalTok{, estimate)) }\OperatorTok
\StringTok{    }\NormalTok{dplyr}\OperatorTok{::}\KeywordTok{rename}\NormalTok{(}\DataTypeTok{b =}\NormalTok{ estimate) }\OperatorTok
\StringTok{    }\KeywordTok{select}\NormalTok{(type, }\KeywordTok{everything}\NormalTok{())}
  \KeywordTok{return}\NormalTok{(tab)}
\NormalTok{\}}
\end{Highlighting}
\end{Shaded}

\section{Question 1}\label{question-1}

Begin by testing the fully unconditional model:
\[mathach_{ij} = \beta_0 + r_{ij}\] \[\beta_0 = \gamma_{00} + u_{0j}\]
Calculate the intraclass correlation to determine how much of the
variance in math achievement resides at Level 2 (the school level).

\begin{Shaded}
\begin{Highlighting}[]
\NormalTok{lmerICC <-}\StringTok{ }\ControlFlowTok{function}\NormalTok{(obj) \{}
\NormalTok{  v <-}\StringTok{ }\KeywordTok{as.data.frame}\NormalTok{(}\KeywordTok{VarCorr}\NormalTok{(obj))}
\NormalTok{  v}\OperatorTok{$}\NormalTok{vcov[}\DecValTok{1}\NormalTok{]}\OperatorTok{/}\KeywordTok{sum}\NormalTok{(v}\OperatorTok{$}\NormalTok{vcov)}
\NormalTok{\}}

\NormalTok{mod0 <-}\StringTok{ }\KeywordTok{lmer}\NormalTok{(mathach }\OperatorTok{~}\StringTok{ }\DecValTok{1} \OperatorTok{+}\StringTok{ }\NormalTok{(}\DecValTok{1} \OperatorTok{|}\StringTok{ }\NormalTok{School), }\DataTypeTok{data =}\NormalTok{ dat)}
\NormalTok{tidy0 <-}\StringTok{ }\NormalTok{broom}\OperatorTok{::}\KeywordTok{tidy}\NormalTok{(mod0)}
\KeywordTok{lmerICC}\NormalTok{(mod0)}
\end{Highlighting}
\end{Shaded}

\begin{verbatim}
## [1] 0.180353
\end{verbatim}

\section{Question 2}\label{question-2}

Modify the model to include student minority status (minority:
1=minority, 0=other):
\[mathach_{ij} = \beta_{0j} + \beta_{1j}minority_{ij} + r_{ij}\]
\[\beta_{0j} = \gamma_{00} + u_{0j}\]
\[\beta_{1j} = \gamma_{10} + u_{1j}\]

\begin{Shaded}
\begin{Highlighting}[]
\NormalTok{mod1  <-}\StringTok{ }\KeywordTok{lmer}\NormalTok{(mathach }\OperatorTok{~}\StringTok{ }\NormalTok{minority }\OperatorTok{+}\StringTok{ }\NormalTok{(minority }\OperatorTok{|}\StringTok{ }\NormalTok{School), }\DataTypeTok{data =}\NormalTok{ dat)}
\NormalTok{tidy1 <-}\StringTok{ }\NormalTok{broom}\OperatorTok{::}\KeywordTok{tidy}\NormalTok{(mod1)}
\NormalTok{tab   <-}\StringTok{ }\KeywordTok{table_fun}\NormalTok{(mod1)}

\KeywordTok{options}\NormalTok{(}\DataTypeTok{knitr.kable.NA =} \StringTok{''}\NormalTok{)}
\NormalTok{tab }\OperatorTok\StringTok{ }\KeywordTok{select}\NormalTok{(}\OperatorTok{-}\NormalTok{type) }\OperatorTok
\StringTok{  }\KeywordTok{kable}\NormalTok{(., }\StringTok{"latex"}\NormalTok{, }\DataTypeTok{escape =}\NormalTok{ F, }\DataTypeTok{booktabs =}\NormalTok{ T,}
        \DataTypeTok{col.names =} \KeywordTok{c}\NormalTok{( }\StringTok{"Term"}\NormalTok{, }\StringTok{"b"}\NormalTok{, }\StringTok{"CI"}\NormalTok{)) }\OperatorTok
\StringTok{  }\KeywordTok{group_rows}\NormalTok{(}\StringTok{"Fixed"}\NormalTok{, }\DecValTok{1}\NormalTok{,}\DecValTok{2}\NormalTok{) }\OperatorTok
\StringTok{  }\KeywordTok{group_rows}\NormalTok{(}\StringTok{"Random"}\NormalTok{, }\DecValTok{3}\NormalTok{,}\DecValTok{4}\NormalTok{) }\OperatorTok
\StringTok{  }\KeywordTok{group_rows}\NormalTok{(}\StringTok{"Model Terms"}\NormalTok{, }\DecValTok{5}\NormalTok{,}\DecValTok{6}\NormalTok{) }\OperatorTok
\StringTok{  }\KeywordTok{kable_styling}\NormalTok{(}\DataTypeTok{latex_options =} \KeywordTok{c}\NormalTok{(}\StringTok{"repeat_header"}\NormalTok{),}\DataTypeTok{full_width =}\NormalTok{ F)}
\end{Highlighting}
\end{Shaded}

\begin{table}[H]
\centering
\begin{tabular}{lll}
\toprule
Term & b & CI\\
\midrule
\addlinespace[0.3em]
\multicolumn{3}{l}{\textbf{Fixed}}\\
\hspace{1em}(Intercept) & 13.68 & [13.28, 14.00]\\
\hspace{1em}minorityMinority & -3.75 & [-4.09, -3.42]\\
\addlinespace[0.3em]
\multicolumn{3}{l}{\textbf{Random}}\\
\hspace{1em}$\tau_{00}$ & 5.62 & [4.45, 6.64]\\
\hspace{1em}$\tau_{11}$ & 3.24 & [2.08, 4.76]\\
$R^2_m$ & 0.06 & \\
$R^2_c$ & 0.21 & \\
\bottomrule
\end{tabular}
\end{table}

\subsection{Part A}\label{part-a}

Is math achievement significantly related to minority status?\\
Yes, match minority status predicts math achievement,
\(\gamma_{10} = -3.75\), 95\% CI \(= [-4.09, -3.42]\).

\subsection{Part B}\label{part-b}

What is the expected (mean) level of math achievement for non-minority
students?\\
The expected mean level of math achievement for non-minority students is
\(\gamma_{00} = 13.68\), 95\% CI \(= [13.28, 14.00]\).

\subsection{Part C}\label{part-c}

What is the expected (mean) level of math achievement for minority
students?\\
The expected mean level of math achievement for non-minority students is
\(9.93\)

\subsection{Part D}\label{part-d}

How much Level 1 variance is accounted for by this model compared to the
fully unconditional model?

\begin{Shaded}
\begin{Highlighting}[]
\NormalTok{sig.d <-}\StringTok{ }\NormalTok{(}\KeywordTok{sigma}\NormalTok{(mod0) }\OperatorTok{-}\StringTok{ }\KeywordTok{sigma}\NormalTok{(mod1))}\OperatorTok{/}\KeywordTok{sigma}\NormalTok{(mod0)}
\end{Highlighting}
\end{Shaded}

This model explains \(2\%`\) of more of the variance in math achievement
than the fully unconditional model.

\section{Question 3}\label{question-3}

Now add student sex (female: male=0, female=1) and group-centered SES to
the Level 1 model:\\
\[mathach_{ij} = \beta_{0j} + \beta_{1j}minority_{ij} + \beta_{2j}female_{ij} + \beta_{3j}(ses_{ij}-meanses_{j}) + r_{ij}\]
\[\beta_{0j} = \gamma_{00} + u_{0j}\]
\[\beta_{1j} = \gamma_{10} + u_{1j}\]
\[\beta_{2j} = \gamma_{20} + u_{2j}\]
\[\beta_{3j} = \gamma_{30} + u_{3j}\]

\begin{Shaded}
\begin{Highlighting}[]
\NormalTok{dat <-}\StringTok{ }\NormalTok{dat }\OperatorTok
\StringTok{  }\KeywordTok{mutate}\NormalTok{(}\DataTypeTok{GC.ses =}\NormalTok{ ses }\OperatorTok{-}\StringTok{ }\NormalTok{meanses)}

\NormalTok{mod2  <-}\StringTok{ }\KeywordTok{lmer}\NormalTok{(mathach }\OperatorTok{~}\StringTok{ }\NormalTok{minority }\OperatorTok{+}\StringTok{ }\NormalTok{female }\OperatorTok{+}\StringTok{ }\NormalTok{GC.ses }\OperatorTok{+}\StringTok{ }\NormalTok{(minority }\OperatorTok{+}\StringTok{ }\NormalTok{female }\OperatorTok{+}\StringTok{ }\NormalTok{GC.ses }\OperatorTok{|}\StringTok{ }\NormalTok{School), }\DataTypeTok{data =}\NormalTok{ dat)}
\NormalTok{tidy2 <-}\StringTok{ }\NormalTok{broom}\OperatorTok{::}\KeywordTok{tidy}\NormalTok{(mod2)}
\NormalTok{tab2  <-}\StringTok{ }\KeywordTok{table_fun}\NormalTok{(mod2)}

\NormalTok{tab2 }\OperatorTok\StringTok{ }\KeywordTok{select}\NormalTok{(}\OperatorTok{-}\NormalTok{type) }\OperatorTok
\StringTok{  }\KeywordTok{kable}\NormalTok{(., }\StringTok{"latex"}\NormalTok{, }\DataTypeTok{escape =}\NormalTok{ F, }\DataTypeTok{booktabs =}\NormalTok{ T,}
        \DataTypeTok{col.names =} \KeywordTok{c}\NormalTok{( }\StringTok{"Term"}\NormalTok{, }\StringTok{"b"}\NormalTok{, }\StringTok{"CI"}\NormalTok{)) }\OperatorTok
\StringTok{  }\KeywordTok{group_rows}\NormalTok{(}\StringTok{"Fixed"}\NormalTok{, }\DecValTok{1}\NormalTok{,}\DecValTok{4}\NormalTok{) }\OperatorTok
\StringTok{  }\KeywordTok{group_rows}\NormalTok{(}\StringTok{"Random"}\NormalTok{, }\DecValTok{5}\NormalTok{,}\DecValTok{8}\NormalTok{) }\OperatorTok
\StringTok{  }\KeywordTok{group_rows}\NormalTok{(}\StringTok{"Model Terms"}\NormalTok{, }\DecValTok{9}\NormalTok{,}\DecValTok{10}\NormalTok{) }\OperatorTok
\StringTok{  }\KeywordTok{kable_styling}\NormalTok{(}\DataTypeTok{latex_options =} \KeywordTok{c}\NormalTok{(}\StringTok{"repeat_header"}\NormalTok{),}\DataTypeTok{full_width =}\NormalTok{ F)}
\end{Highlighting}
\end{Shaded}

\begin{table}[H]
\centering
\begin{tabular}{lll}
\toprule
Term & b & CI\\
\midrule
\addlinespace[0.3em]
\multicolumn{3}{l}{\textbf{Fixed}}\\
\hspace{1em}(Intercept) & 14.15 & [13.87, 14.18]\\
\hspace{1em}minorityMinority & -3.20 & [-3.56, -2.76]\\
\hspace{1em}femaleFemale & -1.26 & [-1.47, -0.79]\\
\hspace{1em}GC.ses & 1.87 & [1.74, 2.01]\\
\addlinespace[0.3em]
\multicolumn{3}{l}{\textbf{Random}}\\
\hspace{1em}$\tau_{00}$ & 6.23 & [5.55, 8.11]\\
\hspace{1em}$\tau_{11}$ & 1.98 & [1.08, 4.28]\\
\hspace{1em}$\tau_{22}$ & 0.88 & [0.21, 1.78]\\
\hspace{1em}$\tau_{33}$ & 0.40 & [0.11, 0.76]\\
$R^2_m$ & 0.10 & \\
$R^2_c$ & 0.25 & \\
\bottomrule
\end{tabular}
\end{table}

\subsection{Part a}\label{part-a-1}

Is there a significant sex difference in math achievement, controlling
for minority status and SES? Yes, females have lower math achievement
than males, controlling for SES and minority status,
\(\gamma_{20} = -1.26, 95\% CI [-1.47, -0.79]\).

\subsection{Part b}\label{part-b-1}

Is the effect of student-level SES significant? Explain how the
coefficient for this effect (\(\beta_{3j}\)) should be interpreted.\\
Controlling for gender and minority status, a one unit increase in SES
is associated with a \(1.87\) (95\% CI \([1.74, 2.01]\)) increase in
math achievement.

\subsection{Part c}\label{part-c-1}

What is the expected (mean) level of math achievement for minority male
students with SES equal to their school average?\\
\(Y_{ij} = \gamma_{00} + \gamma_{10}*1 + \gamma_{20}*0 + \gamma_{30}*0\)\\
\(Y_{ij} = \gamma_{00} + \gamma_{10}\)\\
\(Y_{ij} = 14.15 + -3.20\)\\
\(Y_{ij} = 10.95\)

\subsection{Part d}\label{part-d-1}

How much Level 1 variance is accounted for by this model compared to the
fully unconditional model?\\
5.94\% of the variance is accounted for by the fully conditional model.

\subsection{Part e}\label{part-e}

\begin{enumerate}
\def\labelenumi{(\alph{enumi})}
\setcounter{enumi}{4}
\tightlist
\item
  Does this model provide a significantly better fit than the previous
  model?
\end{enumerate}

\begin{Shaded}
\begin{Highlighting}[]
\NormalTok{(c1 <-}\StringTok{ }\KeywordTok{anova}\NormalTok{(mod1, mod2)) }
\end{Highlighting}
\end{Shaded}

\begin{verbatim}
## Data: dat
## Models:
## mod1: mathach ~ minority + (minority | School)
## mod2: mathach ~ minority + female + GC.ses + (minority + female + GC.ses | 
## mod2:     School)
##      Df   AIC   BIC logLik deviance  Chisq Chi Df Pr(>Chisq)    
## mod1  6 46811 46852 -23400    46799                             
## mod2 15 46460 46563 -23215    46430 369.33      9  < 2.2e-16 ***
## ---
## Signif. codes:  0 '***' 0.001 '**' 0.01 '*' 0.05 '.' 0.1 ' ' 1
\end{verbatim}

Yes, the deviance of the model that includes gender and SES have smaller
deviance than the model that does not, \(\chi^2(9) = 369.33\),
\(p<.001\).

\subsection{Part f}\label{part-f}

Explain why there are 9 degrees of freedom for the \(\chi^2\) test in
the previous question.\\
The smaller model (that did not include gender and SES) had 6 degrees of
freedom, while the larger model had 15 degrees of freedom. The deviance
test we conducted is \(\chi^2\) distributed with degrees of freedom
equal to the degrees of freedom of the larger minus the smaller model.

\section{Question 4}\label{question-4}

Now add sector (1=Catholic, 0=Public) to the model:\\
\[mathach_{ij} = \beta_{0j} + \beta_{1j}minority_{ij} + \beta_{2j}female_{ij} + \beta_{3j}(ses_{ij}-meanses_{j}) + r_{ij}\]\\
\[\beta_{0j} = \gamma_{00} + \gamma{01}sector_j + u_{0j}\]\\
\[\beta_{1j} = \gamma_{10} + \gamma{11}sector_j + u_{1j}\]\\
\[\beta_{2j} = \gamma_{20} + \gamma{21}sector_j + u_{2j}\]\\
\[\beta_{3j} = \gamma_{30} + \gamma{31}sector_j + u_{3j}\]

\begin{Shaded}
\begin{Highlighting}[]
\NormalTok{mod3  <-}\StringTok{ }\KeywordTok{lmer}\NormalTok{(mathach }\OperatorTok{~}\StringTok{ }\NormalTok{minority}\OperatorTok{*}\NormalTok{sector }\OperatorTok{+}\StringTok{ }\NormalTok{female}\OperatorTok{*}\NormalTok{sector }\OperatorTok{+}\StringTok{ }\NormalTok{GC.ses}\OperatorTok{*}\NormalTok{sector }\OperatorTok{+}\StringTok{ }
\StringTok{                }\NormalTok{(minority }\OperatorTok{+}\StringTok{ }\NormalTok{female }\OperatorTok{+}\StringTok{ }\NormalTok{GC.ses }\OperatorTok{|}\StringTok{ }\NormalTok{School), }\DataTypeTok{data =}\NormalTok{ dat)}
\NormalTok{tidy3 <-}\StringTok{ }\NormalTok{broom}\OperatorTok{::}\KeywordTok{tidy}\NormalTok{(mod3)}
\NormalTok{tab3  <-}\StringTok{ }\KeywordTok{table_fun}\NormalTok{(mod3)}

\NormalTok{tab3 }\OperatorTok\StringTok{ }\KeywordTok{select}\NormalTok{(}\OperatorTok{-}\NormalTok{type) }\OperatorTok
\StringTok{  }\KeywordTok{kable}\NormalTok{(., }\StringTok{"latex"}\NormalTok{, }\DataTypeTok{escape =}\NormalTok{ F, }\DataTypeTok{booktabs =}\NormalTok{ T,}
        \DataTypeTok{col.names =} \KeywordTok{c}\NormalTok{( }\StringTok{"Term"}\NormalTok{, }\StringTok{"b"}\NormalTok{, }\StringTok{"CI"}\NormalTok{)) }\OperatorTok
\StringTok{  }\KeywordTok{group_rows}\NormalTok{(}\StringTok{"Fixed"}\NormalTok{, }\DecValTok{1}\NormalTok{,}\DecValTok{8}\NormalTok{) }\OperatorTok
\StringTok{  }\KeywordTok{group_rows}\NormalTok{(}\StringTok{"Random"}\NormalTok{, }\DecValTok{9}\NormalTok{,}\DecValTok{12}\NormalTok{) }\OperatorTok
\StringTok{  }\KeywordTok{group_rows}\NormalTok{(}\StringTok{"Model Terms"}\NormalTok{, }\DecValTok{13}\NormalTok{,}\DecValTok{14}\NormalTok{) }\OperatorTok
\StringTok{  }\KeywordTok{kable_styling}\NormalTok{(}\DataTypeTok{latex_options =} \KeywordTok{c}\NormalTok{(}\StringTok{"repeat_header"}\NormalTok{),}\DataTypeTok{full_width =}\NormalTok{ F)}
\end{Highlighting}
\end{Shaded}

\begin{table}[H]
\centering
\begin{tabular}{lll}
\toprule
Term & b & CI\\
\midrule
\addlinespace[0.3em]
\multicolumn{3}{l}{\textbf{Fixed}}\\
\hspace{1em}(Intercept) & 13.15 & [12.76, 13.50]\\
\hspace{1em}minorityMinority & -4.14 & [-4.66, -3.73]\\
\hspace{1em}sectorCatholic & 2.31 & [1.98, 3.17]\\
\hspace{1em}femaleFemale & -1.30 & [-1.70, -1.02]\\
\hspace{1em}GC.ses & 2.30 & [1.93, 2.50]\\
\hspace{1em}minorityMinority:sectorCatholic & 1.98 & [1.28, 2.73]\\
\hspace{1em}sectorCatholic:femaleFemale & 0.05 & [-0.50, 0.19]\\
\hspace{1em}sectorCatholic:GC.ses & -0.98 & [-1.31, -0.78]\\
\addlinespace[0.3em]
\multicolumn{3}{l}{\textbf{Random}}\\
\hspace{1em}$\tau_{00}$ & 4.91 & [3.57, 5.61]\\
\hspace{1em}$\tau_{11}$ & 0.76 & [0.01, 2.06]\\
\hspace{1em}$\tau_{22}$ & 1.00 & [0.40, 1.23]\\
\hspace{1em}$\tau_{33}$ & 0.18 & [0.01, 0.67]\\
$R^2_m$ & 0.14 & \\
$R^2_c$ & 0.24 & \\
\bottomrule
\end{tabular}
\end{table}

\subsection{Part a}\label{part-a-2}

Does sector significantly predict the Level 1 intercepts
(\(\beta_{0j}\))? If so, provide an interpretation of the relationship.

Sector predicts the level 1 intercepts, \(\gamma01 = 2.31\),
bootstrapped 95\% CI \$\$. In other words, Catholic schools have higher
average math achievement than public schools by \(2.31\).

\subsection{Part b}\label{part-b-2}

Does sector significantly predict the Level 1 slope for minority
(\(\beta_{1j}\))? If so, provide an interpretation of the relationship.

Sector is also a significant predictor of the the level 1 slope for
minority -- that is, math achievement varies as a function of both
minority status and school sector, \(\gamma01 = 1.98\), bootstrapped
95\% CI \([1.28, 2.73]\). In other words, Minority students at Catholic
high schools have math achievement that is on average \(1.98\) than
non-minority students at public high schools \(13.15\).

\subsection{Part c}\label{part-c-2}

Does sector significantly predict the Level 1 slope for ses
(\(\beta_{3j}\))? If so, provide an interpretation of the relationship.

Sector is also a significant predictor of the the level 1 slope for ses
-- that is, math achievement varies as a function of both minority
status and school sector, \(\gamma01 = -0.98\), bootstrapped 95\% CI
\([-1.31, -0.78]\). In other words, Minority students at Catholic high
schools have math achievement that is on average \(1.98\) than
non-minority students at public high schools \(13.15\).

\subsection{Part d}\label{part-d-2}

How much Level 2 variance is accounted for by this model compared to the
model (Question 3) that does not contain sector? Note that this will
require calculating four values, one for each of the four Level 2
equations. You will find something unusual when you do this; comment on
why you think the odd result is occurring.

\begin{Shaded}
\begin{Highlighting}[]
\NormalTok{tidy2 }\OperatorTok\StringTok{ }\KeywordTok{filter}\NormalTok{(}\KeywordTok{grepl}\NormalTok{(}\StringTok{"sd_"}\NormalTok{, term)) }\OperatorTok\StringTok{ }\KeywordTok{select}\NormalTok{(term, estimate, group) }\OperatorTok\StringTok{ }\KeywordTok{mutate}\NormalTok{(}\DataTypeTok{Model =} \StringTok{"Q3 Model"}\NormalTok{) }\OperatorTok\StringTok{ }
\KeywordTok{full_join}\NormalTok{(tidy3 }\OperatorTok\StringTok{ }\KeywordTok{filter}\NormalTok{(}\KeywordTok{grepl}\NormalTok{(}\StringTok{"sd_"}\NormalTok{, term)) }\OperatorTok\StringTok{ }\KeywordTok{select}\NormalTok{(term, estimate, group) }\OperatorTok\StringTok{ }\KeywordTok{mutate}\NormalTok{(}\DataTypeTok{Model =} \StringTok{"Q4 Model"}\NormalTok{)) }\OperatorTok
\StringTok{  }\KeywordTok{mutate}\NormalTok{(}\DataTypeTok{estimate =}\NormalTok{ estimate}\OperatorTok{^}\DecValTok{2}\NormalTok{) }\OperatorTok
\StringTok{  }\KeywordTok{group_by}\NormalTok{(Model) }\OperatorTok\StringTok{ }\KeywordTok{mutate}\NormalTok{(}\DataTypeTok{estimate =}\NormalTok{ estimate}\OperatorTok{/}\KeywordTok{sum}\NormalTok{(estimate)) }\OperatorTok
\StringTok{  }\KeywordTok{spread}\NormalTok{(}\DataTypeTok{key =}\NormalTok{ Model, }\DataTypeTok{value =}\NormalTok{ estimate) }\OperatorTok
\StringTok{  }\KeywordTok{mutate}\NormalTok{(}\DataTypeTok{diff =} \StringTok{`}\DataTypeTok{Q4 Model}\StringTok{`}\OperatorTok{-}\StringTok{`}\DataTypeTok{Q3 Model}\StringTok{`}\NormalTok{) }\OperatorTok
\StringTok{  }\KeywordTok{kable}\NormalTok{(., }\StringTok{"latex"}\NormalTok{, }\DataTypeTok{booktabs =}\NormalTok{ T, }\DataTypeTok{digits =} \DecValTok{2}\NormalTok{,}
        \DataTypeTok{col.names =} \KeywordTok{c}\NormalTok{(}\StringTok{"Term"}\NormalTok{, }\StringTok{"Source"}\NormalTok{, }\StringTok{"Q3 Model"}\NormalTok{, }\StringTok{"Q4 Model"}\NormalTok{, }\StringTok{"Difference in Variance"}\NormalTok{)) }
\end{Highlighting}
\end{Shaded}

\begin{tabular}{llrrr}
\toprule
Term & Source & Q3 Model & Q4 Model & Difference in Variance\\
\midrule
sd\_(Intercept).School & School & 0.14 & 0.12 & -0.02\\
sd\_femaleFemale.School & School & 0.02 & 0.02 & 0.00\\
sd\_GC.ses.School & School & 0.01 & 0.00 & 0.00\\
sd\_minorityMinority.School & School & 0.04 & 0.02 & -0.03\\
sd\_Observation.Residual & Residual & 0.79 & 0.84 & 0.05\\
\bottomrule
\end{tabular}

Across the models, we see that some sources of variance increase and
some decrease when sector is included at Level 2. The proportion of
variance of the intercepts as well minority and group-centered SES
slopes decreased, while the variances of the gender slopes and Level 1
Residuals actually decreased when sector was added. In other words, we
are explaining \emph{less} variance at Level 2 than we were previously.
This makes sense when you interpret the \(\tau\) matrix as conditional
variances (conditional on the new Level 2 predictor sector). What this
means is that when we include sector in the model, we reduce differences
across schools, at least for mean math achievement, SES, and minority
status.

\subsection{Part e}\label{part-e-1}

Does this model provide a significantly better fit than the previous
model?

\begin{Shaded}
\begin{Highlighting}[]
\NormalTok{(c2 <-}\StringTok{ }\KeywordTok{anova}\NormalTok{(mod2, mod3))}
\end{Highlighting}
\end{Shaded}

\begin{verbatim}
## Data: dat
## Models:
## mod2: mathach ~ minority + female + GC.ses + (minority + female + GC.ses | 
## mod2:     School)
## mod3: mathach ~ minority * sector + female * sector + GC.ses * sector + 
## mod3:     (minority + female + GC.ses | School)
##      Df   AIC   BIC logLik deviance  Chisq Chi Df Pr(>Chisq)    
## mod2 15 46460 46563 -23215    46430                             
## mod3 19 46375 46506 -23168    46337 92.723      4  < 2.2e-16 ***
## ---
## Signif. codes:  0 '***' 0.001 '**' 0.01 '*' 0.05 '.' 0.1 ' ' 1
\end{verbatim}

Yes, the deviance of the model that includes sector in Level 2 is
smaller deviance than the model that does not, \(\chi^2(4) = 92.72\),
\(p<.001\).


\end{document}
