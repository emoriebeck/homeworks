\documentclass{article}\usepackage[]{graphicx}\usepackage[]{color}
%% maxwidth is the original width if it is less than linewidth
%% otherwise use linewidth (to make sure the graphics do not exceed the margin)
\makeatletter
\def\maxwidth{ %
  \ifdim\Gin@nat@width>\linewidth
    \linewidth
  \else
    \Gin@nat@width
  \fi
}
\makeatother

\definecolor{fgcolor}{rgb}{0.345, 0.345, 0.345}
\newcommand{\hlnum}[1]{\textcolor[rgb]{0.686,0.059,0.569}{#1}}%
\newcommand{\hlstr}[1]{\textcolor[rgb]{0.192,0.494,0.8}{#1}}%
\newcommand{\hlcom}[1]{\textcolor[rgb]{0.678,0.584,0.686}{\textit{#1}}}%
\newcommand{\hlopt}[1]{\textcolor[rgb]{0,0,0}{#1}}%
\newcommand{\hlstd}[1]{\textcolor[rgb]{0.345,0.345,0.345}{#1}}%
\newcommand{\hlkwa}[1]{\textcolor[rgb]{0.161,0.373,0.58}{\textbf{#1}}}%
\newcommand{\hlkwb}[1]{\textcolor[rgb]{0.69,0.353,0.396}{#1}}%
\newcommand{\hlkwc}[1]{\textcolor[rgb]{0.333,0.667,0.333}{#1}}%
\newcommand{\hlkwd}[1]{\textcolor[rgb]{0.737,0.353,0.396}{\textbf{#1}}}%
\let\hlipl\hlkwb

\usepackage{framed}
\makeatletter
\newenvironment{kframe}{%
 \def\at@end@of@kframe{}%
 \ifinner\ifhmode%
  \def\at@end@of@kframe{\end{minipage}}%
  \begin{minipage}{\columnwidth}%
 \fi\fi%
 \def\FrameCommand##1{\hskip\@totalleftmargin \hskip-\fboxsep
 \colorbox{shadecolor}{##1}\hskip-\fboxsep
     % There is no \\@totalrightmargin, so:
     \hskip-\linewidth \hskip-\@totalleftmargin \hskip\columnwidth}%
 \MakeFramed {\advance\hsize-\width
   \@totalleftmargin\z@ \linewidth\hsize
   \@setminipage}}%
 {\par\unskip\endMakeFramed%
 \at@end@of@kframe}
\makeatother

\definecolor{shadecolor}{rgb}{.97, .97, .97}
\definecolor{messagecolor}{rgb}{0, 0, 0}
\definecolor{warningcolor}{rgb}{1, 0, 1}
\definecolor{errorcolor}{rgb}{1, 0, 0}
\newenvironment{knitrout}{}{} % an empty environment to be redefined in TeX

\usepackage{alltt}

% \usepackage[utf8]{inputenc}
\usepackage{amsmath}
\usepackage{fancyhdr}
\usepackage{array}
\usepackage{longtable}
\usepackage{graphicx}
\usepackage{color}
\usepackage[letterpaper, margin=1in]{geometry}
\usepackage{lscape}
\newcommand{\blandscape}{\begin{landscape}}
\newcommand{\elandscape}{\end{landscape}}
\usepackage{dcolumn}
\usepackage{bbm}
\usepackage{threeparttable}
\usepackage{booktabs}
\usepackage{expex}
\usepackage{pdflscape}
\usepackage{rotating, graphicx}
\usepackage{tabulary}
\usepackage{lscape}
\usepackage{makecell}
\usepackage{algorithm}
\usepackage{multirow}
\usepackage{colortbl}
\usepackage{longtable}
\usepackage{array}
\usepackage{multirow}
\usepackage{wrapfig}
\usepackage{float}
\usepackage{pdflscape}
\usepackage{tabu}
\usepackage{threeparttable}

\title{%
Homework 12\\
\large Applied Mutlivariate Analysis}
\date{November 27, 2018}
\author{Emorie Beck}
\IfFileExists{upquote.sty}{\usepackage{upquote}}{}
\begin{document}
\maketitle
% \SweaveOpts{concordance=TRUE}

\section{Workspace}
\subsection{Packages}



\begin{knitrout}
\definecolor{shadecolor}{rgb}{0.969, 0.969, 0.969}\color{fgcolor}\begin{kframe}
\begin{alltt}
\hlkwd{library}\hlstd{(car)}
\hlkwd{library}\hlstd{(knitr)}
\hlkwd{library}\hlstd{(kableExtra)}
\hlkwd{library}\hlstd{(psych)}
\hlkwd{library}\hlstd{(MASS)}
\hlkwd{library}\hlstd{(Rmisc)}
\hlkwd{library}\hlstd{(mlogit)}
\hlkwd{library}\hlstd{(broom)}
\hlkwd{library}\hlstd{(plyr)}
\hlkwd{library}\hlstd{(tidyverse)}
\end{alltt}
\end{kframe}
\end{knitrout}



\subsection{data}
The file, Set\_10.csv, contains the following data from the job search study: number of publications while in graduate school, length of time to complete the Ph.D. (in years), sex of candidate (1 = men, 2 = women), quality of the degree-granting institution (1 = top-tier research institution, 2 = middle-tier research institution, and 3 = lower-tier research institution), and the outcome of the job search (1 = no interviews, 2 = interviewed but not hired, 3 = hired).

Conduct a multinomial logistic regression on these data, predicting job search outcome from the other variables. Use the "no interviews" outcome as the reference for the dependent variable. Use the lower-tier category as the reference for the quality of the degree-granting institution predictor. Use women as the reference for the sex of candidate predictor.

\begin{knitrout}
\definecolor{shadecolor}{rgb}{0.969, 0.969, 0.969}\color{fgcolor}\begin{kframe}
\begin{alltt}
\hlstd{wd} \hlkwb{<-} \hlstr{"https://github.com/emoriebeck/homeworks/raw/master/multivariate/homeworks/homework12"}

\hlstd{dat} \hlkwb{<-} \hlkwd{sprintf}\hlstd{(}\hlstr{"%s/Set_10.csv"}\hlstd{, wd)} \hlopt \hlkwd{read.csv}\hlstd{(.,} \hlkwc{stringsAsFactors} \hlstd{= F)} \hlopt \hlstd{tbl_df} \hlopt
  \hlkwd{mutate}\hlstd{(}\hlkwc{outcome} \hlstd{=} \hlkwd{factor}\hlstd{(outcome,} \hlkwc{levels} \hlstd{=} \hlkwd{c}\hlstd{(}\hlnum{1}\hlstd{,}\hlnum{2}\hlstd{,}\hlnum{3}\hlstd{),} \hlkwc{labels} \hlstd{=} \hlkwd{c}\hlstd{(}\hlstr{"no interview"}\hlstd{,} \hlstr{"interview"}\hlstd{,} \hlstr{"hired"}\hlstd{)),}
         \hlkwc{sex} \hlstd{=} \hlkwd{as.numeric}\hlstd{(}\hlkwd{mapvalues}\hlstd{(sex,} \hlnum{1}\hlopt{:}\hlnum{2}\hlstd{,} \hlkwd{c}\hlstd{(}\hlnum{1}\hlstd{,}\hlnum{0}\hlstd{))),}
         \hlkwc{Institution} \hlstd{=} \hlkwd{factor}\hlstd{(}\hlkwd{mapvalues}\hlstd{(Institution,} \hlnum{1}\hlopt{:}\hlnum{3}\hlstd{,} \hlnum{0}\hlopt{:}\hlnum{2}\hlstd{),} \hlkwc{levels} \hlstd{=} \hlnum{0}\hlopt{:}\hlnum{2}\hlstd{,} \hlkwc{labels} \hlstd{=} \hlkwd{c}\hlstd{(}\hlstr{"Tier 1"}\hlstd{,} \hlstr{"Tier 2"}\hlstd{,} \hlstr{"Tier 3"}\hlstd{)))} \hlopt
  \hlkwd{mutate_at}\hlstd{(}\hlkwd{vars}\hlstd{(pubs, years),} \hlkwd{funs}\hlstd{(}\hlkwc{c} \hlstd{=} \hlkwd{as.numeric}\hlstd{(}\hlkwd{scale}\hlstd{(.,} \hlkwc{scale} \hlstd{= F))))}

\hlkwd{head}\hlstd{(dat)}
\end{alltt}
\begin{verbatim}
## # A tibble: 6 x 8
##      ID Institution   sex years  pubs outcome      pubs_c years_c
##   <int> <fct>       <dbl> <int> <int> <fct>         <dbl>   <dbl>
## 1   122 Tier 3          1     5     0 interview     -4.30 -1.09  
## 2     1 Tier 3          1     6     0 no interview  -4.30 -0.0900
## 3   191 Tier 3          0     6     0 no interview  -4.30 -0.0900
## 4   194 Tier 2          0     6     0 no interview  -4.30 -0.0900
## 5     4 Tier 3          1     7     0 no interview  -4.30  0.91  
## 6     6 Tier 2          1     7     0 no interview  -4.30  0.91
\end{verbatim}
\end{kframe}
\end{knitrout}

\section{Question 1}
1. When the "interviewed but not hired" outcome is compared to the reference outcome: 

\begin{knitrout}
\definecolor{shadecolor}{rgb}{0.969, 0.969, 0.969}\color{fgcolor}\begin{kframe}
\begin{alltt}
\hlstd{jobs} \hlkwb{<-} \hlstd{dat} \hlopt
  \hlkwd{mutate}\hlstd{(}\hlkwc{NI} \hlstd{=} \hlkwd{ifelse}\hlstd{(outcome} \hlopt{==} \hlstr{"no interview"}\hlstd{,} \hlnum{1}\hlstd{,} \hlnum{0}\hlstd{),}
         \hlkwc{I} \hlstd{=} \hlkwd{ifelse}\hlstd{(outcome} \hlopt{==} \hlstr{"interview"}\hlstd{,} \hlnum{1}\hlstd{,} \hlnum{0}\hlstd{),}
         \hlkwc{H} \hlstd{=} \hlkwd{ifelse}\hlstd{(outcome} \hlopt{==} \hlstr{"hired"}\hlstd{,} \hlnum{1}\hlstd{,} \hlnum{0}\hlstd{))} \hlopt
  \hlkwd{gather}\hlstd{(}\hlkwc{key} \hlstd{= outcome.ids,} \hlkwc{value} \hlstd{= outcome, NI}\hlopt{:}\hlstd{H)} \hlopt
  \hlkwd{mutate}\hlstd{(}\hlkwc{T1vT2} \hlstd{=} \hlkwd{ifelse}\hlstd{(Institution} \hlopt{==} \hlstr{"Tier 2"}\hlstd{,} \hlnum{1}\hlstd{,} \hlnum{0}\hlstd{),}
         \hlkwc{T1vT3} \hlstd{=} \hlkwd{ifelse}\hlstd{(Institution} \hlopt{==} \hlstr{"Tier 3"}\hlstd{,} \hlnum{1}\hlstd{,} \hlnum{0}\hlstd{))} \hlopt
  \hlkwd{select}\hlstd{(ID, sex}\hlopt{:}\hlstd{pubs, outcome.ids}\hlopt{:}\hlstd{T1vT3)} \hlopt
  \hlkwd{mutate}\hlstd{(}\hlkwc{outcome.ids} \hlstd{=} \hlkwd{factor}\hlstd{(outcome.ids,} \hlkwc{levels} \hlstd{=} \hlkwd{c}\hlstd{(}\hlstr{"NI"}\hlstd{,} \hlstr{"I"}\hlstd{,} \hlstr{"H"}\hlstd{)))} \hlopt
  \hlkwd{arrange}\hlstd{(ID, outcome.ids)} \hlopt \hlstd{data.frame}

\hlstd{J} \hlkwb{<-} \hlkwd{mlogit.data}\hlstd{(jobs,}\hlkwc{shape}\hlstd{=}\hlstr{"long"}\hlstd{,}\hlkwc{choice}\hlstd{=}\hlstr{"outcome"}\hlstd{,}\hlkwc{alt.var}\hlstd{=}\hlstr{"outcome.ids"}\hlstd{)}

\hlstd{Ref_Level} \hlkwb{<-} \hlstr{"NI"}
\hlstd{fit_1} \hlkwb{<-} \hlkwd{mlogit}\hlstd{(outcome} \hlopt{~} \hlnum{0} \hlopt{|} \hlnum{1} \hlopt{+} \hlstd{sex} \hlopt{+} \hlstd{T1vT2} \hlopt{+} \hlstd{T1vT3} \hlopt{+} \hlstd{years} \hlopt{+} \hlstd{pubs,} \hlkwc{data} \hlstd{= J,} \hlkwc{reflevel} \hlstd{= Ref_Level)}
\end{alltt}
\end{kframe}
\end{knitrout}

\begin{knitrout}
\definecolor{shadecolor}{rgb}{0.969, 0.969, 0.969}\color{fgcolor}\begin{kframe}
\begin{alltt}
\hlkwd{cbind}\hlstd{(}\hlkwd{data.frame}\hlstd{(}\hlkwc{b} \hlstd{=} \hlkwd{coef}\hlstd{(fit_1)),} \hlkwd{confint}\hlstd{(fit_1))} \hlopt \hlkwd{data.frame}\hlstd{()} \hlopt
  \hlkwd{mutate}\hlstd{(}\hlkwc{term} \hlstd{=} \hlkwd{rownames}\hlstd{(.))} \hlopt
  \hlstd{tbl_df} \hlopt
  \hlkwd{select}\hlstd{(term,} \hlkwd{everything}\hlstd{())} \hlopt
  \hlkwd{setNames}\hlstd{(}\hlkwd{c}\hlstd{(}\hlstr{"term"}\hlstd{,} \hlstr{"b"}\hlstd{,} \hlstr{"lower"}\hlstd{,} \hlstr{"upper"}\hlstd{))} \hlopt
  \hlkwd{mutate}\hlstd{(}\hlkwc{sig} \hlstd{=} \hlkwd{ifelse}\hlstd{(}\hlkwd{sign}\hlstd{(lower)} \hlopt{==} \hlkwd{sign}\hlstd{(upper),} \hlstr{"sig"}\hlstd{,} \hlstr{"ns"}\hlstd{))} \hlopt
  \hlkwd{mutate_at}\hlstd{(}\hlkwd{vars}\hlstd{(b, lower, upper),} \hlkwd{funs}\hlstd{(exp))} \hlopt
  \hlkwd{mutate}\hlstd{(}\hlkwc{CI} \hlstd{=} \hlkwd{sprintf}\hlstd{(}\hlstr{"[%.2f, %.2f]"}\hlstd{, lower, upper),} \hlkwc{b} \hlstd{=} \hlkwd{sprintf}\hlstd{(}\hlstr{"%.2f"}\hlstd{, b))} \hlopt
  \hlkwd{mutate_at}\hlstd{(}\hlkwd{vars}\hlstd{(b, CI),} \hlkwd{funs}\hlstd{(}\hlkwd{ifelse}\hlstd{(sig} \hlopt{==} \hlstr{"sig"}\hlstd{,} \hlkwd{sprintf}\hlstd{(}\hlstr{"\textbackslash{}\textbackslash{}textbf\{%s\}"}\hlstd{, .), .)))} \hlopt
  \hlkwd{select}\hlstd{(term, b, CI)} \hlopt
  \hlkwd{kable}\hlstd{(.,} \hlstr{"latex"}\hlstd{,} \hlkwc{booktabs} \hlstd{= T,} \hlkwc{escape} \hlstd{= F)} \hlopt
  \hlkwd{kable_styling}\hlstd{(}\hlkwc{full_width} \hlstd{= F)}
\end{alltt}
\end{kframe}\begin{table}[H]
\centering
\begin{tabular}{lll}
\toprule
term & b & CI\\
\midrule
I:(intercept) & \textbf{368880.06} & \textbf{[1299.26, 104730785.79]}\\
H:(intercept) & \textbf{21544.17} & \textbf{[57.57, 8062497.58]}\\
I:sex & 1.82 & [0.46, 7.25]\\
H:sex & 2.86 & [0.63, 12.90]\\
I:T1vT2 & 0.26 & [0.03, 1.98]\\
\addlinespace
H:T1vT2 & 0.17 & [0.02, 1.55]\\
I:T1vT3 & 0.50 & [0.05, 4.50]\\
H:T1vT3 & 0.58 & [0.05, 6.28]\\
I:years & \textbf{0.07} & \textbf{[0.03, 0.18]}\\
H:years & \textbf{0.03} & \textbf{[0.01, 0.07]}\\
\addlinespace
I:pubs & \textbf{9.90} & \textbf{[4.47, 21.92]}\\
H:pubs & \textbf{36.95} & \textbf{[15.89, 85.89]}\\
\bottomrule
\end{tabular}
\end{table}


\end{knitrout}

\subsection{Part A}
What are the significant predictors?

Both years and publications are significant predictors of the outcome. 

\subsection{Part B}
How should the significant predictors be interpreted?

\textbf{Years}: An additional year in graduate school multiplies the odds associated with being interviewed by .07.

\textbf{Publications}: Each additional publication multiplies the odds of being interviewed by 9.90.

\section{Question 2}
When the "hired" outcome is compared to the reference outcome: 

\subsection{Part A}
What are the significant predictors?

Both years and publications are significant predictors of the outcome. 

\subsection{Part B}
How should the significant predictors be interpreted?

\textbf{Years}: An additional year in graduate school multiplies the odds associated with being hired by .03. 

An additional year in graduate school is associated with a .03 increase in odds of being hired.

\textbf{Publications}: Each additional publication multiplies the odds of being hired by 36.95. 

\section{Question 3}
What is the probability that a man will be hired if he completes his degree in 5 years at a third-tier institution and enters the job market with 5 publications?

$Y_H = b_{0H} + b_{1H}*sex + b_{2H}*years + b_{3H}*pubs$

\begin{knitrout}
\definecolor{shadecolor}{rgb}{0.969, 0.969, 0.969}\color{fgcolor}\begin{kframe}
\begin{alltt}
\hlcom{# get cases that match this because I'm too lazy to  create a data frame}
\hlstd{dat} \hlopt \hlkwd{filter}\hlstd{(years} \hlopt{==} \hlnum{5} \hlopt{&} \hlstd{pubs} \hlopt{==} \hlnum{5} \hlopt{&} \hlstd{sex} \hlopt{==} \hlnum{0} \hlopt{&} \hlstd{Institution} \hlopt{==} \hlstr{"Tier 3"} \hlopt{&} \hlstd{outcome} \hlopt{==} \hlstr{"hired"}\hlstd{)}
\end{alltt}
\begin{verbatim}
## # A tibble: 0 x 8
## # ... with 8 variables: ID <int>, Institution <fct>, sex <dbl>,
## #   years <int>, pubs <int>, outcome <fct>, pubs_c <dbl>, years_c <dbl>
\end{verbatim}
\begin{alltt}
\hlstd{nd} \hlkwb{<-} \hlkwd{crossing}\hlstd{(}\hlkwc{sex} \hlstd{=} \hlnum{1}\hlstd{,}
         \hlkwc{years} \hlstd{=} \hlnum{5}\hlstd{,}
         \hlkwc{pubs} \hlstd{=} \hlnum{5}\hlstd{,}
         \hlkwc{outcome.ids} \hlstd{=} \hlkwd{c}\hlstd{(}\hlstr{"NI"}\hlstd{,} \hlstr{"I"}\hlstd{,} \hlstr{"H"}\hlstd{),}
         \hlkwc{T1vT2} \hlstd{=} \hlnum{0}\hlstd{,}
         \hlkwc{T1vT3} \hlstd{=} \hlnum{1}
         \hlstd{)} \hlopt
  \hlkwd{mutate}\hlstd{(}\hlkwc{outcome} \hlstd{=} \hlkwd{c}\hlstd{(}\hlnum{0}\hlstd{,} \hlnum{0}\hlstd{,} \hlnum{1}\hlstd{))}

\hlstd{O.Q3} \hlkwb{<-} \hlkwd{predict}\hlstd{(fit_1,} \hlkwc{newdata} \hlstd{= nd)}
\hlcom{# e^L_M / (1 + sigma_\{j=2\}^\{M\}\{e^L_j\})}
\hlstd{(p.Q3} \hlkwb{<-} \hlkwd{exp}\hlstd{(O.Q3[}\hlstr{"H"}\hlstd{])} \hlopt{/} \hlstd{(}\hlnum{1} \hlopt{+} \hlkwd{sum}\hlstd{(}\hlkwd{exp}\hlstd{(O.Q3[}\hlkwd{c}\hlstd{(}\hlstr{"I"}\hlstd{,} \hlstr{"H"}\hlstd{)]))))}
\end{alltt}
\begin{verbatim}
##         H 
## 0.3492526
\end{verbatim}
\end{kframe}
\end{knitrout}

The probability would be 34.93\%.

\section{Question 4}
4. How do his odds of getting hired change if he gets 2 more publications but takes a year longer to finish?
\begin{knitrout}
\definecolor{shadecolor}{rgb}{0.969, 0.969, 0.969}\color{fgcolor}\begin{kframe}
\begin{alltt}
\hlcom{# get cases that match this because I'm too lazy to  create a data frame}
\hlstd{nd} \hlkwb{<-} \hlkwd{crossing}\hlstd{(}\hlkwc{sex} \hlstd{=} \hlnum{1}\hlstd{,}
         \hlkwc{years} \hlstd{=} \hlnum{6}\hlstd{,}
         \hlkwc{pubs} \hlstd{=} \hlnum{7}\hlstd{,}
         \hlkwc{outcome.ids} \hlstd{=} \hlkwd{c}\hlstd{(}\hlstr{"NI"}\hlstd{,} \hlstr{"I"}\hlstd{,} \hlstr{"H"}\hlstd{),}
         \hlkwc{T1vT2} \hlstd{=} \hlnum{0}\hlstd{,}
         \hlkwc{T1vT3} \hlstd{=} \hlnum{1}
         \hlstd{)} \hlopt
  \hlkwd{mutate}\hlstd{(}\hlkwc{outcome} \hlstd{=} \hlkwd{c}\hlstd{(}\hlnum{0}\hlstd{,} \hlnum{0}\hlstd{,} \hlnum{1}\hlstd{))}

\hlstd{O.Q4} \hlkwb{<-} \hlkwd{predict}\hlstd{(fit_1,} \hlkwc{newdata} \hlstd{= nd)}
\hlstd{(O.diff} \hlkwb{<-} \hlkwd{exp}\hlstd{(O.Q4[}\hlstr{"H"}\hlstd{]))} \hlopt{-} \hlkwd{exp}\hlstd{(O.Q3[}\hlstr{"H"}\hlstd{])}
\end{alltt}
\begin{verbatim}
##         H 
## 0.6969054
\end{verbatim}
\end{kframe}
\end{knitrout}

The difference in odds will be .70 higher odds of being hired.

\end{document}
