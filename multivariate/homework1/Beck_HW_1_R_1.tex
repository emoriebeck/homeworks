\documentclass{article}\usepackage[]{graphicx}\usepackage[]{color}
%% maxwidth is the original width if it is less than linewidth
%% otherwise use linewidth (to make sure the graphics do not exceed the margin)
\makeatletter
\def\maxwidth{ %
  \ifdim\Gin@nat@width>\linewidth
    \linewidth
  \else
    \Gin@nat@width
  \fi
}
\makeatother

\definecolor{fgcolor}{rgb}{0.345, 0.345, 0.345}
\newcommand{\hlnum}[1]{\textcolor[rgb]{0.686,0.059,0.569}{#1}}%
\newcommand{\hlstr}[1]{\textcolor[rgb]{0.192,0.494,0.8}{#1}}%
\newcommand{\hlcom}[1]{\textcolor[rgb]{0.678,0.584,0.686}{\textit{#1}}}%
\newcommand{\hlopt}[1]{\textcolor[rgb]{0,0,0}{#1}}%
\newcommand{\hlstd}[1]{\textcolor[rgb]{0.345,0.345,0.345}{#1}}%
\newcommand{\hlkwa}[1]{\textcolor[rgb]{0.161,0.373,0.58}{\textbf{#1}}}%
\newcommand{\hlkwb}[1]{\textcolor[rgb]{0.69,0.353,0.396}{#1}}%
\newcommand{\hlkwc}[1]{\textcolor[rgb]{0.333,0.667,0.333}{#1}}%
\newcommand{\hlkwd}[1]{\textcolor[rgb]{0.737,0.353,0.396}{\textbf{#1}}}%
\let\hlipl\hlkwb

\usepackage{framed}
\makeatletter
\newenvironment{kframe}{%
 \def\at@end@of@kframe{}%
 \ifinner\ifhmode%
  \def\at@end@of@kframe{\end{minipage}}%
  \begin{minipage}{\columnwidth}%
 \fi\fi%
 \def\FrameCommand##1{\hskip\@totalleftmargin \hskip-\fboxsep
 \colorbox{shadecolor}{##1}\hskip-\fboxsep
     % There is no \\@totalrightmargin, so:
     \hskip-\linewidth \hskip-\@totalleftmargin \hskip\columnwidth}%
 \MakeFramed {\advance\hsize-\width
   \@totalleftmargin\z@ \linewidth\hsize
   \@setminipage}}%
 {\par\unskip\endMakeFramed%
 \at@end@of@kframe}
\makeatother

\definecolor{shadecolor}{rgb}{.97, .97, .97}
\definecolor{messagecolor}{rgb}{0, 0, 0}
\definecolor{warningcolor}{rgb}{1, 0, 1}
\definecolor{errorcolor}{rgb}{1, 0, 0}
\newenvironment{knitrout}{}{} % an empty environment to be redefined in TeX

\usepackage{alltt}

% \usepackage[utf8]{inputenc}
\usepackage{amsmath}
\usepackage{fancyhdr}
\usepackage{array}
\usepackage{longtable}
\usepackage{graphicx}
\usepackage{color}
\usepackage[letterpaper, margin=1in]{geometry}
\usepackage{lscape}
\newcommand{\blandscape}{\begin{landscape}}
\newcommand{\elandscape}{\end{landscape}}
\usepackage{dcolumn}
\usepackage{bbm}
\usepackage{threeparttable}
\usepackage{booktabs}
\usepackage{expex}
\usepackage{pdflscape}
\usepackage{rotating, graphicx}
\usepackage{tabulary}
\usepackage{lscape}
\usepackage{makecell}
\usepackage{algorithm}
\usepackage{multirow}
\usepackage{colortbl}
\usepackage{longtable}
\usepackage{array}
\usepackage{multirow}
\usepackage{wrapfig}
\usepackage{float}
\usepackage{pdflscape}
\usepackage{tabu}
\usepackage{threeparttable}

\title{%
Homework 1\\
\large Applied Mutlivariate Analysis}
\date{September 7, 2018}
\author{Emorie Beck}
\IfFileExists{upquote.sty}{\usepackage{upquote}}{}
\begin{document}
\maketitle
% \SweaveOpts{concordance=TRUE}

\section{Workspace}
\subsection{Packages}
\begin{knitrout}
\definecolor{shadecolor}{rgb}{0.969, 0.969, 0.969}\color{fgcolor}\begin{kframe}
\begin{alltt}
\hlkwd{library}\hlstd{(car)}
\hlkwd{library}\hlstd{(knitr)}
\hlkwd{library}\hlstd{(kableExtra)}
\hlkwd{library}\hlstd{(plyr)}
\hlkwd{library}\hlstd{(tidyverse)}
\end{alltt}


{\ttfamily\noindent\itshape\color{messagecolor}{\#\# -- Attaching packages ------------------------------------------------------------------------------- tidyverse 1.2.1 --}}

{\ttfamily\noindent\itshape\color{messagecolor}{\#\# √ ggplot2 2.2.1\ \ \ \  √ purrr\ \  0.2.4\\\#\# √ tibble\ \ 1.4.2\ \ \ \  √ dplyr\ \  0.7.5\\\#\# √ tidyr\ \  0.8.0\ \ \ \  √ stringr 1.3.0\\\#\# √ readr\ \  1.1.1\ \ \ \  √ forcats 0.3.0}}

{\ttfamily\noindent\itshape\color{messagecolor}{\#\# -- Conflicts ---------------------------------------------------------------------------------- tidyverse\_conflicts() --\\\#\# x dplyr::arrange()\ \  masks plyr::arrange()\\\#\# x purrr::compact()\ \  masks plyr::compact()\\\#\# x dplyr::count()\ \ \ \  masks plyr::count()\\\#\# x dplyr::failwith()\ \ masks plyr::failwith()\\\#\# x dplyr::filter()\ \ \ \ masks stats::filter()\\\#\# x dplyr::id()\ \ \ \ \ \ \ \ masks plyr::id()\\\#\# x dplyr::lag()\ \ \ \ \ \  masks stats::lag()\\\#\# x dplyr::mutate()\ \ \ \ masks plyr::mutate()\\\#\# x dplyr::recode()\ \ \ \ masks car::recode()\\\#\# x dplyr::rename()\ \ \ \ masks plyr::rename()\\\#\# x purrr::some()\ \ \ \ \ \ masks car::some()\\\#\# x dplyr::summarise() masks plyr::summarise()\\\#\# x dplyr::summarize() masks plyr::summarize()}}\end{kframe}
\end{knitrout}

\subsection{data}
\begin{knitrout}
\definecolor{shadecolor}{rgb}{0.969, 0.969, 0.969}\color{fgcolor}\begin{kframe}
\begin{alltt}
\hlstd{dat} \hlkwb{<-} \hlkwd{tribble}\hlstd{(}
  \hlopt{~}\hlstd{Group,} \hlopt{~}\hlstd{V1,} \hlopt{~}\hlstd{V2,} \hlopt{~}\hlstd{V3,} \hlopt{~}\hlstd{V4,}
  \hlnum{1}\hlstd{,} \hlnum{4}\hlstd{,} \hlnum{9}\hlstd{,} \hlnum{3}\hlstd{,} \hlnum{8}\hlstd{,}
  \hlnum{2}\hlstd{,} \hlnum{6}\hlstd{,} \hlnum{7}\hlstd{,} \hlnum{2}\hlstd{,} \hlnum{1}\hlstd{,}
  \hlnum{3}\hlstd{,} \hlnum{1}\hlstd{,} \hlnum{6}\hlstd{,} \hlnum{6}\hlstd{,} \hlnum{2}\hlstd{,}
  \hlnum{4}\hlstd{,} \hlnum{3}\hlstd{,} \hlnum{8}\hlstd{,} \hlnum{7}\hlstd{,} \hlnum{4}
\hlstd{)}
\hlstd{X} \hlkwb{<-} \hlstd{dat} \hlopt \hlkwd{select}\hlstd{(}\hlopt{-}\hlstd{Group)} \hlopt \hlstd{as.matrix}
\end{alltt}
\end{kframe}
\end{knitrout}


\begin{kframe}
\begin{alltt}
\hlstd{dat} \hlopt
  \hlkwd{kable}\hlstd{(.,} \hlstr{"latex"}\hlstd{,} \hlkwc{escape} \hlstd{= F)} \hlopt
  \hlkwd{column_spec}\hlstd{(}\hlnum{1}\hlstd{,} \hlkwc{bold} \hlstd{= T)}
\end{alltt}
\end{kframe}
\begin{tabular}{>{\bfseries}r|r|r|r|r}
\hline
Group & V1 & V2 & V3 & V4\\
\hline
1 & 4 & 9 & 3 & 8\\
\hline
2 & 6 & 7 & 2 & 1\\
\hline
3 & 1 & 6 & 6 & 2\\
\hline
4 & 3 & 8 & 7 & 4\\
\hline
\end{tabular}




The general matrix equation, LXM, describes how to create linear combinations of the groups (the L matrix) and variables (the M matrix) to test different hypotheses.

\section{Part 1}
In words, describe what each of the following L vectors is trying to accomplish: 
\begin{knitrout}
\definecolor{shadecolor}{rgb}{0.969, 0.969, 0.969}\color{fgcolor}\begin{kframe}
\begin{alltt}
\hlstd{L1} \hlkwb{<-} \hlkwd{c}\hlstd{(}\hlnum{1}\hlstd{,} \hlnum{1}\hlstd{,} \hlnum{1}\hlstd{,} \hlnum{1}\hlstd{)}
\hlstd{L2} \hlkwb{<-} \hlkwd{c}\hlstd{(}\hlnum{1}\hlstd{,} \hlnum{0}\hlstd{,} \hlnum{0}\hlstd{,}\hlopt{-}\hlnum{1}\hlstd{)}
\hlstd{L3} \hlkwb{<-} \hlkwd{c}\hlstd{(}\hlnum{1}\hlstd{,} \hlnum{0}\hlstd{,} \hlnum{0}\hlstd{,} \hlnum{0}\hlstd{)}
\hlstd{L4} \hlkwb{<-} \hlkwd{c}\hlstd{(}\hlnum{1}\hlstd{,} \hlnum{1}\hlstd{,}\hlopt{-}\hlnum{2}\hlstd{,} \hlnum{0}\hlstd{)}
\hlstd{L5} \hlkwb{<-} \hlkwd{c}\hlstd{(}\hlnum{1}\hlstd{,} \hlnum{1}\hlstd{,}\hlopt{-}\hlnum{1}\hlstd{,}\hlopt{-}\hlnum{1}\hlstd{)}
\end{alltt}
\end{kframe}
\end{knitrout}


\subsection{Question 1}
[1 1 1 1] = L1\\
The grand mean across groups.  

\subsection{Question 2}
[1 0 0 -1] = L2 \\
The difference in means between the first and fourth groups.

\subsection{Question 3}
[1 0 0 0] = L3 \\
The mean of the first group.  

\subsection{Question 4}
[1 1 -2 0] = L4 \\
The differnce in means between groups 1+2 and 3. 

\subsection{Question 5}
5. [1 1 -1 -1] = L5\\
The difference in means between groups 1+2 and 3+4.

\section{Part 2}
Assume that you want to answer the following "variable" questions. Give the appropriate M vector or matrix.  
\subsection{Question 6}
6. The linear combination given by L should be performed separately on each of the variables (= M1).

\begin{knitrout}
\definecolor{shadecolor}{rgb}{0.969, 0.969, 0.969}\color{fgcolor}\begin{kframe}
\begin{alltt}
\hlstd{(M1} \hlkwb{<-} \hlkwd{diag}\hlstd{(}\hlnum{1}\hlstd{,} \hlnum{4}\hlstd{,} \hlnum{4}\hlstd{))}
\end{alltt}
\begin{verbatim}
##      [,1] [,2] [,3] [,4]
## [1,]    1    0    0    0
## [2,]    0    1    0    0
## [3,]    0    0    1    0
## [4,]    0    0    0    1
\end{verbatim}
\end{kframe}
\end{knitrout}

\subsection{Question 7}
7. Variable 3 is the only variable of interest (= M2).  

\begin{knitrout}
\definecolor{shadecolor}{rgb}{0.969, 0.969, 0.969}\color{fgcolor}\begin{kframe}
\begin{alltt}
\hlstd{(M2} \hlkwb{<-} \hlkwd{c}\hlstd{(}\hlnum{0}\hlstd{,}\hlnum{0}\hlstd{,}\hlnum{1}\hlstd{,}\hlnum{0}\hlstd{))}
\end{alltt}
\begin{verbatim}
## [1] 0 0 1 0
\end{verbatim}
\end{kframe}
\end{knitrout}


\subsection{Question 8}
8. The difference between Variables 1 and 4 is of interest (=M3).  
\begin{knitrout}
\definecolor{shadecolor}{rgb}{0.969, 0.969, 0.969}\color{fgcolor}\begin{kframe}
\begin{alltt}
\hlstd{(M3} \hlkwb{<-} \hlkwd{c}\hlstd{(}\hlnum{1}\hlstd{,} \hlnum{0}\hlstd{,} \hlnum{0}\hlstd{,} \hlopt{-}\hlnum{1}\hlstd{))}
\end{alltt}
\begin{verbatim}
## [1]  1  0  0 -1
\end{verbatim}
\end{kframe}
\end{knitrout}

\subsection{Question 9}
9. The sum of all variables is of interest (= M4).
\begin{knitrout}
\definecolor{shadecolor}{rgb}{0.969, 0.969, 0.969}\color{fgcolor}\begin{kframe}
\begin{alltt}
\hlstd{(M4} \hlkwb{<-} \hlkwd{rep}\hlstd{(}\hlnum{1}\hlstd{,}\hlnum{4}\hlstd{))}
\end{alltt}
\begin{verbatim}
## [1] 1 1 1 1
\end{verbatim}
\end{kframe}
\end{knitrout}


\subsection{Question 10}
10.The difference between the first two variables and the difference between the second two variables is to be compared (= M5).
\begin{knitrout}
\definecolor{shadecolor}{rgb}{0.969, 0.969, 0.969}\color{fgcolor}\begin{kframe}
\begin{alltt}
\hlstd{(M5} \hlkwb{<-} \hlkwd{c}\hlstd{(}\hlnum{1}\hlstd{,} \hlopt{-}\hlnum{1}\hlstd{,} \hlopt{-}\hlnum{1}\hlstd{,} \hlnum{1}\hlstd{))}
\end{alltt}
\begin{verbatim}
## [1]  1 -1 -1  1
\end{verbatim}
\end{kframe}
\end{knitrout}


\section{Part 3}
Carry out the following matrix multiplications, in R. 

\subsection{Question 11}


$$
L1XM2 =
\begin{bmatrix}
1\\
1\\
1\\
1
\end{bmatrix}
\begin{bmatrix}
4 & 9 & 3 & 8\\
6 & 7 & 2 & 1\\
1 & 6 & 6 & 2\\
3 & 8 & 7 & 4
\end{bmatrix}
\begin{bmatrix}
0 & 0 & 1 0
\end{bmatrix}
$$ 

\begin{knitrout}
\definecolor{shadecolor}{rgb}{0.969, 0.969, 0.969}\color{fgcolor}\begin{kframe}
\begin{alltt}
\hlstd{L1} \hlopt \hlstd{X} \hlopt \hlstd{M2}
\end{alltt}
\begin{verbatim}
##      [,1]
## [1,]   18
\end{verbatim}
\end{kframe}
\end{knitrout}

\subsection{Question 12}
$$
L2XM5 =
\begin{bmatrix}
1\\
0\\
0\\
-1
\end{bmatrix}
\begin{bmatrix}
4 & 9 & 3 & 8\\
6 & 7 & 2 & 1\\
1 & 6 & 6 & 2\\
3 & 8 & 7 & 4
\end{bmatrix}
\begin{bmatrix}
1 & -1 & -1 1
\end{bmatrix}
$$ 
\begin{knitrout}
\definecolor{shadecolor}{rgb}{0.969, 0.969, 0.969}\color{fgcolor}\begin{kframe}
\begin{alltt}
\hlstd{L2} \hlopt \hlstd{X} \hlopt \hlstd{M5}
\end{alltt}
\begin{verbatim}
##      [,1]
## [1,]    8
\end{verbatim}
\end{kframe}
\end{knitrout}

\subsection{Question 13}

$$
L3XM3 =
\begin{bmatrix}
1\\
0\\
0\\
0
\end{bmatrix}
\begin{bmatrix}
4 & 9 & 3 & 8\\
6 & 7 & 2 & 1\\
1 & 6 & 6 & 2\\
3 & 8 & 7 & 4
\end{bmatrix}
\begin{bmatrix}
1 & 0 & -1 0
\end{bmatrix}
$$ 
\begin{knitrout}
\definecolor{shadecolor}{rgb}{0.969, 0.969, 0.969}\color{fgcolor}\begin{kframe}
\begin{alltt}
\hlstd{L3} \hlopt \hlstd{X} \hlopt \hlstd{M3}
\end{alltt}
\begin{verbatim}
##      [,1]
## [1,]   -4
\end{verbatim}
\end{kframe}
\end{knitrout}

\subsection{Question 14}
$$
L4XM4 =
\begin{bmatrix}
1\\
1\\
-2\\
0
\end{bmatrix}
\begin{bmatrix}
4 & 9 & 3 & 8\\
6 & 7 & 2 & 1\\
1 & 6 & 6 & 2\\
3 & 8 & 7 & 4
\end{bmatrix}
\begin{bmatrix}
1 & 1 & 1 & 1
\end{bmatrix}
$$ 
\begin{knitrout}
\definecolor{shadecolor}{rgb}{0.969, 0.969, 0.969}\color{fgcolor}\begin{kframe}
\begin{alltt}
\hlstd{L4} \hlopt \hlstd{X} \hlopt \hlstd{M4}
\end{alltt}
\begin{verbatim}
##      [,1]
## [1,]   10
\end{verbatim}
\end{kframe}
\end{knitrout}

\subsection{Question 15}
$$
L5XM1 =
\begin{bmatrix}
1\\
1\\
-1\\
-1
\end{bmatrix}
\begin{bmatrix}
4 & 9 & 3 & 8\\
6 & 7 & 2 & 1\\
1 & 6 & 6 & 2\\
3 & 8 & 7 & 4
\end{bmatrix}
\begin{bmatrix}
1 & 0 & 0 & 0 \\
0 & 1 & 0 & 0 \\
0 & 0 & 1 & 0 \\
0 & 0 & 0 & 1
\end{bmatrix}
$$ 
\begin{knitrout}
\definecolor{shadecolor}{rgb}{0.969, 0.969, 0.969}\color{fgcolor}\begin{kframe}
\begin{alltt}
\hlstd{L5} \hlopt \hlstd{X} \hlopt \hlstd{M1}
\end{alltt}
\begin{verbatim}
##      [,1] [,2] [,3] [,4]
## [1,]    6    2   -8    3
\end{verbatim}
\end{kframe}
\end{knitrout}


\end{document}
