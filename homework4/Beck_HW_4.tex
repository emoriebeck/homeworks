\documentclass[]{article}
\usepackage{lmodern}
\usepackage{amssymb,amsmath}
\usepackage{ifxetex,ifluatex}
\usepackage{fixltx2e} % provides \textsubscript
\ifnum 0\ifxetex 1\fi\ifluatex 1\fi=0 % if pdftex
  \usepackage[T1]{fontenc}
  \usepackage[utf8]{inputenc}
\else % if luatex or xelatex
  \ifxetex
    \usepackage{mathspec}
  \else
    \usepackage{fontspec}
  \fi
  \defaultfontfeatures{Ligatures=TeX,Scale=MatchLowercase}
\fi
% use upquote if available, for straight quotes in verbatim environments
\IfFileExists{upquote.sty}{\usepackage{upquote}}{}
% use microtype if available
\IfFileExists{microtype.sty}{%
\usepackage{microtype}
\UseMicrotypeSet[protrusion]{basicmath} % disable protrusion for tt fonts
}{}
\usepackage[margin=1in]{geometry}
\usepackage{hyperref}
\hypersetup{unicode=true,
            pdftitle={Homework 4},
            pdfauthor={Emorie Beck},
            pdfborder={0 0 0},
            breaklinks=true}
\urlstyle{same}  % don't use monospace font for urls
\usepackage{color}
\usepackage{fancyvrb}
\newcommand{\VerbBar}{|}
\newcommand{\VERB}{\Verb[commandchars=\\\{\}]}
\DefineVerbatimEnvironment{Highlighting}{Verbatim}{commandchars=\\\{\}}
% Add ',fontsize=\small' for more characters per line
\usepackage{framed}
\definecolor{shadecolor}{RGB}{248,248,248}
\newenvironment{Shaded}{\begin{snugshade}}{\end{snugshade}}
\newcommand{\KeywordTok}[1]{\textcolor[rgb]{0.13,0.29,0.53}{\textbf{#1}}}
\newcommand{\DataTypeTok}[1]{\textcolor[rgb]{0.13,0.29,0.53}{#1}}
\newcommand{\DecValTok}[1]{\textcolor[rgb]{0.00,0.00,0.81}{#1}}
\newcommand{\BaseNTok}[1]{\textcolor[rgb]{0.00,0.00,0.81}{#1}}
\newcommand{\FloatTok}[1]{\textcolor[rgb]{0.00,0.00,0.81}{#1}}
\newcommand{\ConstantTok}[1]{\textcolor[rgb]{0.00,0.00,0.00}{#1}}
\newcommand{\CharTok}[1]{\textcolor[rgb]{0.31,0.60,0.02}{#1}}
\newcommand{\SpecialCharTok}[1]{\textcolor[rgb]{0.00,0.00,0.00}{#1}}
\newcommand{\StringTok}[1]{\textcolor[rgb]{0.31,0.60,0.02}{#1}}
\newcommand{\VerbatimStringTok}[1]{\textcolor[rgb]{0.31,0.60,0.02}{#1}}
\newcommand{\SpecialStringTok}[1]{\textcolor[rgb]{0.31,0.60,0.02}{#1}}
\newcommand{\ImportTok}[1]{#1}
\newcommand{\CommentTok}[1]{\textcolor[rgb]{0.56,0.35,0.01}{\textit{#1}}}
\newcommand{\DocumentationTok}[1]{\textcolor[rgb]{0.56,0.35,0.01}{\textbf{\textit{#1}}}}
\newcommand{\AnnotationTok}[1]{\textcolor[rgb]{0.56,0.35,0.01}{\textbf{\textit{#1}}}}
\newcommand{\CommentVarTok}[1]{\textcolor[rgb]{0.56,0.35,0.01}{\textbf{\textit{#1}}}}
\newcommand{\OtherTok}[1]{\textcolor[rgb]{0.56,0.35,0.01}{#1}}
\newcommand{\FunctionTok}[1]{\textcolor[rgb]{0.00,0.00,0.00}{#1}}
\newcommand{\VariableTok}[1]{\textcolor[rgb]{0.00,0.00,0.00}{#1}}
\newcommand{\ControlFlowTok}[1]{\textcolor[rgb]{0.13,0.29,0.53}{\textbf{#1}}}
\newcommand{\OperatorTok}[1]{\textcolor[rgb]{0.81,0.36,0.00}{\textbf{#1}}}
\newcommand{\BuiltInTok}[1]{#1}
\newcommand{\ExtensionTok}[1]{#1}
\newcommand{\PreprocessorTok}[1]{\textcolor[rgb]{0.56,0.35,0.01}{\textit{#1}}}
\newcommand{\AttributeTok}[1]{\textcolor[rgb]{0.77,0.63,0.00}{#1}}
\newcommand{\RegionMarkerTok}[1]{#1}
\newcommand{\InformationTok}[1]{\textcolor[rgb]{0.56,0.35,0.01}{\textbf{\textit{#1}}}}
\newcommand{\WarningTok}[1]{\textcolor[rgb]{0.56,0.35,0.01}{\textbf{\textit{#1}}}}
\newcommand{\AlertTok}[1]{\textcolor[rgb]{0.94,0.16,0.16}{#1}}
\newcommand{\ErrorTok}[1]{\textcolor[rgb]{0.64,0.00,0.00}{\textbf{#1}}}
\newcommand{\NormalTok}[1]{#1}
\usepackage{graphicx,grffile}
\makeatletter
\def\maxwidth{\ifdim\Gin@nat@width>\linewidth\linewidth\else\Gin@nat@width\fi}
\def\maxheight{\ifdim\Gin@nat@height>\textheight\textheight\else\Gin@nat@height\fi}
\makeatother
% Scale images if necessary, so that they will not overflow the page
% margins by default, and it is still possible to overwrite the defaults
% using explicit options in \includegraphics[width, height, ...]{}
\setkeys{Gin}{width=\maxwidth,height=\maxheight,keepaspectratio}
\IfFileExists{parskip.sty}{%
\usepackage{parskip}
}{% else
\setlength{\parindent}{0pt}
\setlength{\parskip}{6pt plus 2pt minus 1pt}
}
\setlength{\emergencystretch}{3em}  % prevent overfull lines
\providecommand{\tightlist}{%
  \setlength{\itemsep}{0pt}\setlength{\parskip}{0pt}}
\setcounter{secnumdepth}{0}
% Redefines (sub)paragraphs to behave more like sections
\ifx\paragraph\undefined\else
\let\oldparagraph\paragraph
\renewcommand{\paragraph}[1]{\oldparagraph{#1}\mbox{}}
\fi
\ifx\subparagraph\undefined\else
\let\oldsubparagraph\subparagraph
\renewcommand{\subparagraph}[1]{\oldsubparagraph{#1}\mbox{}}
\fi

%%% Use protect on footnotes to avoid problems with footnotes in titles
\let\rmarkdownfootnote\footnote%
\def\footnote{\protect\rmarkdownfootnote}

%%% Change title format to be more compact
\usepackage{titling}

% Create subtitle command for use in maketitle
\newcommand{\subtitle}[1]{
  \posttitle{
    \begin{center}\large#1\end{center}
    }
}

\setlength{\droptitle}{-2em}
  \title{Homework 4}
  \pretitle{\vspace{\droptitle}\centering\huge}
  \posttitle{\par}
  \author{Emorie Beck}
  \preauthor{\centering\large\emph}
  \postauthor{\par}
  \predate{\centering\large\emph}
  \postdate{\par}
  \date{\today}

\usepackage{fancyhdr}
\usepackage{array}
\usepackage{longtable}
\usepackage{lscape}
\newcommand{\blandscape}{\begin{landscape}}
\newcommand{\elandscape}{\end{landscape}}
\usepackage{dcolumn}
\usepackage{bbm}
\usepackage{threeparttable}
\usepackage{booktabs}
\usepackage{expex}
\usepackage{rotating, graphicx}
\usepackage{tabulary}
\usepackage{algorithm}
\usepackage{multirow}
\usepackage{colortbl}
\usepackage{longtable}
\usepackage{array}
\usepackage{multirow}
\usepackage[table]{xcolor}
\usepackage{wrapfig}
\usepackage{float}
\usepackage{pdflscape}
\usepackage{tabu}
\usepackage{threeparttable}
\usepackage{booktabs}
\usepackage{longtable}
\usepackage{array}
\usepackage{multirow}
\usepackage[table]{xcolor}
\usepackage{wrapfig}
\usepackage{float}
\usepackage{colortbl}
\usepackage{pdflscape}
\usepackage{tabu}
\usepackage{threeparttable}
\usepackage[normalem]{ulem}

\begin{document}
\maketitle

{
\setcounter{tocdepth}{2}
\tableofcontents
}
\section{Workspace}\label{workspace}

\subsection{Packages}\label{packages}

\begin{Shaded}
\begin{Highlighting}[]
\KeywordTok{library}\NormalTok{(psych)}
\KeywordTok{library}\NormalTok{(lme4)}
\KeywordTok{library}\NormalTok{(knitr)}
\KeywordTok{library}\NormalTok{(kableExtra)}
\KeywordTok{library}\NormalTok{(multcomp)}
\KeywordTok{library}\NormalTok{(plyr)}
\KeywordTok{library}\NormalTok{(tidyverse)}
\end{Highlighting}
\end{Shaded}

For this assignment, you will use the High School and Beyond data
(HSB.csv) that were used for Homework 1.

\subsection{Data}\label{data}

\begin{Shaded}
\begin{Highlighting}[]
\NormalTok{data_url <-}\StringTok{ "https://raw.githubusercontent.com/emoriebeck/homeworks/master/homework4/HSB(18).csv"}
\NormalTok{dat      <-}\StringTok{ }\KeywordTok{read.csv}\NormalTok{(}\KeywordTok{url}\NormalTok{(data_url)) }\OperatorTok\StringTok{ }\NormalTok{tbl_df}
\end{Highlighting}
\end{Shaded}

\section{Question 1}\label{question-1}

Create four dummy codes to represent the four possible combinations of
the variables, minority and female:

\begin{itemize}
\item minority girls (MG): minority = 1, female = 1  
\item minority boys (MB) minority = 1 and female = 0 
\item nonminority girls (NMG): minority = 0 and female = 1 
\item nonminority boys (NMB): minority = 0 and female = 0  
\end{itemize}

\begin{Shaded}
\begin{Highlighting}[]
\NormalTok{dat <-}\StringTok{ }\NormalTok{dat }\OperatorTok\StringTok{ }
\StringTok{  }\KeywordTok{mutate}\NormalTok{(}
    \DataTypeTok{MG  =} \KeywordTok{ifelse}\NormalTok{(female }\OperatorTok{==}\StringTok{ }\DecValTok{1} \OperatorTok{&}\StringTok{ }\NormalTok{minority }\OperatorTok{==}\StringTok{ }\DecValTok{1}\NormalTok{, }\DecValTok{1}\NormalTok{, }\DecValTok{0}\NormalTok{),}
    \DataTypeTok{MB  =} \KeywordTok{ifelse}\NormalTok{(female }\OperatorTok{==}\StringTok{ }\DecValTok{0} \OperatorTok{&}\StringTok{ }\NormalTok{minority }\OperatorTok{==}\StringTok{ }\DecValTok{1}\NormalTok{, }\DecValTok{1}\NormalTok{, }\DecValTok{0}\NormalTok{),}
    \DataTypeTok{NMG =} \KeywordTok{ifelse}\NormalTok{(female }\OperatorTok{==}\StringTok{ }\DecValTok{1} \OperatorTok{&}\StringTok{ }\NormalTok{minority }\OperatorTok{==}\StringTok{ }\DecValTok{0}\NormalTok{, }\DecValTok{1}\NormalTok{, }\DecValTok{0}\NormalTok{),}
    \DataTypeTok{NMB =} \KeywordTok{ifelse}\NormalTok{(female }\OperatorTok{==}\StringTok{ }\DecValTok{0} \OperatorTok{&}\StringTok{ }\NormalTok{minority }\OperatorTok{==}\StringTok{ }\DecValTok{0}\NormalTok{, }\DecValTok{1}\NormalTok{, }\DecValTok{0}\NormalTok{)}
\NormalTok{  )}
\end{Highlighting}
\end{Shaded}

\section{Question 2}\label{question-2}

Fit the following no-intercept model:

\subsection{Part A}\label{part-a}

mathach \textasciitilde{}-1 + MG + MB + NMG + NMB + (-1 + MG + MB + NMG
+ NMB\textbar{}School).

\begin{Shaded}
\begin{Highlighting}[]
\KeywordTok{source}\NormalTok{(}\StringTok{"https://raw.githubusercontent.com/emoriebeck/homeworks/master/table_fun.R"}\NormalTok{)}

\NormalTok{mod2 <-}\StringTok{ }\KeywordTok{lmer}\NormalTok{(mathach }\OperatorTok{~}\StringTok{ }\OperatorTok{-}\DecValTok{1} \OperatorTok{+}\StringTok{ }\NormalTok{MG }\OperatorTok{+}\StringTok{ }\NormalTok{MB }\OperatorTok{+}\StringTok{ }\NormalTok{NMG }\OperatorTok{+}\StringTok{ }\NormalTok{NMB }\OperatorTok{+}\StringTok{ }\NormalTok{(}\OperatorTok{-}\DecValTok{1} \OperatorTok{+}\StringTok{ }\NormalTok{MG }\OperatorTok{+}\StringTok{ }\NormalTok{MB }\OperatorTok{+}\StringTok{ }\NormalTok{NMG }\OperatorTok{+}\StringTok{ }\NormalTok{NMB}\OperatorTok{|}\NormalTok{School), }\DataTypeTok{data =}\NormalTok{ dat)}
\NormalTok{tab2 <-}\StringTok{ }\KeywordTok{table_fun}\NormalTok{(mod2)}
\end{Highlighting}
\end{Shaded}

\subsection{Part B}\label{part-b}

Construct weights for combining the fixed effect parameters from (a)
that will test the main effect of Student Sex.

\begin{Shaded}
\begin{Highlighting}[]
\NormalTok{contra2b            <-}\StringTok{ }\KeywordTok{matrix}\NormalTok{(}\KeywordTok{c}\NormalTok{(}\OperatorTok{-}\DecValTok{1}\NormalTok{,}\DecValTok{1}\NormalTok{,}\OperatorTok{-}\DecValTok{1}\NormalTok{,}\DecValTok{1}\NormalTok{),}\DataTypeTok{nrow=}\DecValTok{1}\NormalTok{,}\DataTypeTok{ncol=}\DecValTok{4}\NormalTok{,}\DataTypeTok{byrow=}\OtherTok{TRUE}\NormalTok{)}
\KeywordTok{row.names}\NormalTok{(contra2b) <-}\StringTok{ "female"}

\NormalTok{glht_mod2_b   <-}\StringTok{ }\KeywordTok{glht}\NormalTok{(mod2, }\DataTypeTok{linfct =}\NormalTok{ contra2b, }\DataTypeTok{alternative =} \StringTok{"two.sided"}\NormalTok{, }\DataTypeTok{rhs=}\DecValTok{0}\NormalTok{)}
\NormalTok{res_mod2_b    <-}\StringTok{ }\KeywordTok{confint}\NormalTok{(glht_mod2_b, }\DataTypeTok{calpha =} \KeywordTok{univariate_calpha}\NormalTok{()) }
\NormalTok{res_mod2_b_df <-}\StringTok{ }\NormalTok{res_mod2_b}\OperatorTok{$}\NormalTok{confint }\OperatorTok\StringTok{ }\KeywordTok{data.frame}\NormalTok{()}
\end{Highlighting}
\end{Shaded}

There was a main effect of sex, \(b = 2.61\), 95\% CI \([1.8, 3.43\){]}.

\subsection{Part C}\label{part-c}

Construct weights for combining the fixed effect parameters from (a)
that will test the main effect of Minority Status.

\begin{Shaded}
\begin{Highlighting}[]
\NormalTok{contra2c            <-}\StringTok{ }\KeywordTok{matrix}\NormalTok{(}\KeywordTok{c}\NormalTok{(}\DecValTok{1}\NormalTok{,}\DecValTok{1}\NormalTok{,}\OperatorTok{-}\DecValTok{1}\NormalTok{,}\OperatorTok{-}\DecValTok{1}\NormalTok{),}\DataTypeTok{nrow=}\DecValTok{1}\NormalTok{,}\DataTypeTok{ncol=}\DecValTok{4}\NormalTok{,}\DataTypeTok{byrow=}\OtherTok{TRUE}\NormalTok{)}
\KeywordTok{row.names}\NormalTok{(contra2c) <-}\StringTok{ "minority"}

\NormalTok{glht_mod2_c   <-}\StringTok{ }\KeywordTok{glht}\NormalTok{(mod2, }\DataTypeTok{linfct =}\NormalTok{ contra2c, }\DataTypeTok{alternative =} \StringTok{"two.sided"}\NormalTok{, }\DataTypeTok{rhs=}\DecValTok{0}\NormalTok{)}
\NormalTok{res_mod2_c    <-}\StringTok{ }\KeywordTok{confint}\NormalTok{(glht_mod2_c, }\DataTypeTok{calpha =} \KeywordTok{univariate_calpha}\NormalTok{()) }
\NormalTok{res_mod2_c_df <-}\StringTok{ }\NormalTok{res_mod2_c}\OperatorTok{$}\NormalTok{confint }\OperatorTok\StringTok{ }\KeywordTok{data.frame}\NormalTok{()}
\end{Highlighting}
\end{Shaded}

There was a main effect of sex, \(b = -7.51\), 95\% CI
\([-8.59, -6.43\){]}.

\subsection{Part D}\label{part-d}

Construct weights for combining the fixed effect parameters from (a)
that will test the Student Sex x Minority Status interaction.

\begin{Shaded}
\begin{Highlighting}[]
\NormalTok{contra2d            <-}\StringTok{ }\KeywordTok{matrix}\NormalTok{(}\KeywordTok{c}\NormalTok{(}\DecValTok{1}\NormalTok{,}\OperatorTok{-}\DecValTok{1}\NormalTok{,}\OperatorTok{-}\DecValTok{1}\NormalTok{,}\DecValTok{1}\NormalTok{),}\DataTypeTok{nrow=}\DecValTok{1}\NormalTok{,}\DataTypeTok{ncol=}\DecValTok{4}\NormalTok{,}\DataTypeTok{byrow=}\OtherTok{TRUE}\NormalTok{)}
\KeywordTok{row.names}\NormalTok{(contra2d) <-}\StringTok{ "minority:female"}

\NormalTok{glht_mod2_d   <-}\StringTok{ }\KeywordTok{glht}\NormalTok{(mod2, }\DataTypeTok{linfct =}\NormalTok{ contra2d, }\DataTypeTok{alternative =} \StringTok{"two.sided"}\NormalTok{, }\DataTypeTok{rhs=}\DecValTok{0}\NormalTok{)}
\NormalTok{res_mod2_d    <-}\StringTok{ }\KeywordTok{confint}\NormalTok{(glht_mod2_c, }\DataTypeTok{calpha =} \KeywordTok{univariate_calpha}\NormalTok{()) }
\NormalTok{res_mod2_d_df <-}\StringTok{ }\NormalTok{res_mod2_d}\OperatorTok{$}\NormalTok{confint }\OperatorTok\StringTok{ }\KeywordTok{data.frame}\NormalTok{()}
\end{Highlighting}
\end{Shaded}

There was no interaction between gender and minority status,
\(b = -7.51\), 95\% CI \([-8.59, -6.43\){]}.

\subsection{Part E}\label{part-e}

Construct weights for combining the fixed effect parameters from (a)
that will test the Minority Status effect, but just for males.

\begin{Shaded}
\begin{Highlighting}[]
\NormalTok{contra2e            <-}\StringTok{ }\KeywordTok{matrix}\NormalTok{(}\KeywordTok{c}\NormalTok{(}\DecValTok{0}\NormalTok{,}\DecValTok{1}\NormalTok{,}\DecValTok{0}\NormalTok{,}\OperatorTok{-}\DecValTok{1}\NormalTok{),}\DataTypeTok{nrow=}\DecValTok{1}\NormalTok{,}\DataTypeTok{ncol=}\DecValTok{4}\NormalTok{,}\DataTypeTok{byrow=}\OtherTok{TRUE}\NormalTok{)}
\KeywordTok{row.names}\NormalTok{(contra2e) <-}\StringTok{ "Males: Minority v. Non-Minority"}

\NormalTok{glht_mod2_e   <-}\StringTok{ }\KeywordTok{glht}\NormalTok{(mod2, }\DataTypeTok{linfct =}\NormalTok{ contra2e, }\DataTypeTok{alternative =} \StringTok{"two.sided"}\NormalTok{, }\DataTypeTok{rhs=}\DecValTok{0}\NormalTok{)}
\NormalTok{res_mod2_e    <-}\StringTok{ }\KeywordTok{confint}\NormalTok{(glht_mod2_e, }\DataTypeTok{calpha =} \KeywordTok{univariate_calpha}\NormalTok{()) }
\NormalTok{res_mod2_e_df <-}\StringTok{ }\NormalTok{res_mod2_e}\OperatorTok{$}\NormalTok{confint }\OperatorTok\StringTok{ }\KeywordTok{data.frame}\NormalTok{()}
\end{Highlighting}
\end{Shaded}

Minority and nonminority males differed, \(b = -3.96\), 95\% CI
\([-4.62, -3.3\){]}.

\subsection{Part F}\label{part-f}

Construct weights for combining the fixed effect parameters from (a)
that will test the Student Sex effect, but just for minority students.

\begin{Shaded}
\begin{Highlighting}[]
\NormalTok{contra2f            <-}\StringTok{ }\KeywordTok{matrix}\NormalTok{(}\KeywordTok{c}\NormalTok{(}\DecValTok{1}\NormalTok{,}\OperatorTok{-}\DecValTok{1}\NormalTok{,}\DecValTok{0}\NormalTok{,}\DecValTok{0}\NormalTok{),}\DataTypeTok{nrow=}\DecValTok{1}\NormalTok{,}\DataTypeTok{ncol=}\DecValTok{4}\NormalTok{,}\DataTypeTok{byrow=}\OtherTok{TRUE}\NormalTok{)}
\KeywordTok{row.names}\NormalTok{(contra2f) <-}\StringTok{ "Minorities: Females v. Males"}

\NormalTok{glht_mod2_f   <-}\StringTok{ }\KeywordTok{glht}\NormalTok{(mod2, }\DataTypeTok{linfct =}\NormalTok{ contra2f, }\DataTypeTok{alternative =} \StringTok{"two.sided"}\NormalTok{, }\DataTypeTok{rhs=}\DecValTok{0}\NormalTok{)}
\NormalTok{res_mod2_f    <-}\StringTok{ }\KeywordTok{confint}\NormalTok{(glht_mod2_f, }\DataTypeTok{calpha =} \KeywordTok{univariate_calpha}\NormalTok{()) }
\NormalTok{res_mod2_f_df <-}\StringTok{ }\NormalTok{res_mod2_f}\OperatorTok{$}\NormalTok{confint }\OperatorTok\StringTok{ }\KeywordTok{data.frame}\NormalTok{()}
\end{Highlighting}
\end{Shaded}

Male and female minorities differed in math achievement, \(b = -1.1\),
95\% CI \([-1.78, -0.43\){]}.

\subsection{Part G}\label{part-g}

Use the resulting weight matrix in \texttt{glht()} to obtain the
significance tests. Which effects are significant?

\begin{Shaded}
\begin{Highlighting}[]
\NormalTok{contra2 <-}\StringTok{ }\KeywordTok{rbind}\NormalTok{(contra2b, contra2c, contra2d, contra2e, contra2f)}

\NormalTok{glht_mod2   <-}\StringTok{ }\KeywordTok{glht}\NormalTok{(mod2, }\DataTypeTok{linfct =}\NormalTok{ contra2, }\DataTypeTok{alternative =} \StringTok{"two.sided"}\NormalTok{, }\DataTypeTok{rhs=}\DecValTok{0}\NormalTok{)}
\NormalTok{res_mod2    <-}\StringTok{ }\KeywordTok{confint}\NormalTok{(glht_mod2, }\DataTypeTok{calpha =} \KeywordTok{univariate_calpha}\NormalTok{()) }
\NormalTok{res_mod2_df <-}\StringTok{ }\NormalTok{res_mod2}\OperatorTok{$}\NormalTok{confint }\OperatorTok\StringTok{ }\KeywordTok{data.frame}\NormalTok{() }\OperatorTok\StringTok{ }\KeywordTok{mutate}\NormalTok{(}\DataTypeTok{term =} \KeywordTok{rownames}\NormalTok{(.))}

\KeywordTok{options}\NormalTok{(}\DataTypeTok{knitr.kable.NA =} \StringTok{''}\NormalTok{)}
\NormalTok{res_mod2_df }\OperatorTok\StringTok{ }
\StringTok{  }\KeywordTok{mutate}\NormalTok{(}\DataTypeTok{CI =} \KeywordTok{sprintf}\NormalTok{(}\StringTok{"[%.2f, %.2f]"}\NormalTok{, lwr, upr)) }\OperatorTok
\StringTok{  }\KeywordTok{select}\NormalTok{(term, Estimate, CI) }\OperatorTok
\StringTok{  }\KeywordTok{kable}\NormalTok{(., }\StringTok{"latex"}\NormalTok{, }\DataTypeTok{escape =}\NormalTok{ F, }\DataTypeTok{booktabs =}\NormalTok{ T, }\DataTypeTok{digits =} \DecValTok{2}\NormalTok{,}
        \DataTypeTok{col.names =} \KeywordTok{c}\NormalTok{( }\StringTok{"Hypothesis"}\NormalTok{, }\StringTok{"b"}\NormalTok{, }\StringTok{"CI"}\NormalTok{),}
        \DataTypeTok{caption =} \StringTok{"Question 2G glht Results"}\NormalTok{) }\OperatorTok
\StringTok{  }\KeywordTok{kable_styling}\NormalTok{(}\DataTypeTok{full_width =}\NormalTok{ F)}
\end{Highlighting}
\end{Shaded}

\begin{table}

\caption{\label{tab:unnamed-chunk-8}Question 2G glht Results}
\centering
\begin{tabular}[t]{lrl}
\toprule
Hypothesis & b & CI\\
\midrule
female & 2.61 & [1.80, 3.43]\\
minority & -7.51 & [-8.59, -6.43]\\
minority:female & 0.40 & [-0.35, 1.16]\\
Males: Minority v. Non-Minority & -3.96 & [-4.62, -3.30]\\
Minorities: Females v. Males & -1.10 & [-1.78, -0.43]\\
\bottomrule
\end{tabular}
\end{table}

All the effects but the interaction are significant.

\section{Question 3}\label{question-3}

Now fit this model:

\subsection{Part A}\label{part-a-1}

mathach \textasciitilde{} 1 + minority + female + minority:female + (1 +
minority + female + minority:female\textbar{}School)

\begin{Shaded}
\begin{Highlighting}[]
\NormalTok{mod3 <-}\StringTok{ }\KeywordTok{lmer}\NormalTok{(mathach }\OperatorTok{~}\StringTok{ }\DecValTok{1} \OperatorTok{+}\StringTok{ }\NormalTok{minority}\OperatorTok{*}\NormalTok{female }\OperatorTok{+}\StringTok{ }\NormalTok{(}\DecValTok{1} \OperatorTok{+}\StringTok{ }\NormalTok{minority }\OperatorTok{*}\StringTok{ }\NormalTok{female }\OperatorTok{|}\StringTok{ }\NormalTok{School), }\DataTypeTok{data =}\NormalTok{ dat)}
\NormalTok{tab3 <-}\StringTok{ }\KeywordTok{table_fun}\NormalTok{(mod3)}
\end{Highlighting}
\end{Shaded}

\subsection{Part B}\label{part-b-1}

Do the main effects and interaction tests resemble the results from the
previous analysis?

\begin{Shaded}
\begin{Highlighting}[]
\NormalTok{res_mod2_df }\OperatorTok\StringTok{ }
\StringTok{  }\KeywordTok{mutate}\NormalTok{(}\DataTypeTok{Estimate =} \KeywordTok{sprintf}\NormalTok{(}\StringTok{"%.2f"}\NormalTok{, Estimate),}
         \DataTypeTok{CI =} \KeywordTok{sprintf}\NormalTok{(}\StringTok{"[%.2f, %.2f]"}\NormalTok{, lwr, upr)) }\OperatorTok
\StringTok{  }\KeywordTok{select}\NormalTok{(term, Estimate, CI) }\OperatorTok
\StringTok{  }\KeywordTok{setNames}\NormalTok{(}\KeywordTok{c}\NormalTok{(}\StringTok{"term"}\NormalTok{, }\StringTok{"b"}\NormalTok{, }\StringTok{"CI"}\NormalTok{)) }\OperatorTok
\StringTok{  }\KeywordTok{filter}\NormalTok{(term }\OperatorTok\StringTok{ }\KeywordTok{c}\NormalTok{(}\StringTok{"female"}\NormalTok{, }\StringTok{"minority"}\NormalTok{, }\StringTok{"minority:female"}\NormalTok{)) }\OperatorTok
\StringTok{  }\KeywordTok{mutate}\NormalTok{(}\DataTypeTok{model =} \StringTok{"No Intercept Model"}\NormalTok{) }\OperatorTok
\StringTok{  }\KeywordTok{full_join}\NormalTok{(tab3 }\OperatorTok\StringTok{ }\KeywordTok{filter}\NormalTok{(type }\OperatorTok{==}\StringTok{ "Fixed Parts"}\NormalTok{) }\OperatorTok
\StringTok{              }\KeywordTok{mutate}\NormalTok{(}\DataTypeTok{model =} \StringTok{"Traditional Model"}\NormalTok{) }\OperatorTok
\StringTok{              }\KeywordTok{select}\NormalTok{(}\OperatorTok{-}\NormalTok{type)) }\OperatorTok
\StringTok{  }\KeywordTok{gather}\NormalTok{(}\DataTypeTok{key =}\NormalTok{ est, }\DataTypeTok{value =}\NormalTok{ value, b, CI) }\OperatorTok
\StringTok{  }\KeywordTok{unite}\NormalTok{(tmp, model, est, }\DataTypeTok{sep =} \StringTok{"."}\NormalTok{) }\OperatorTok
\StringTok{  }\KeywordTok{spread}\NormalTok{(}\DataTypeTok{key =}\NormalTok{ tmp, }\DataTypeTok{value =}\NormalTok{ value) }\OperatorTok
\StringTok{  }\KeywordTok{kable}\NormalTok{(., }\StringTok{"latex"}\NormalTok{, }\DataTypeTok{escape =}\NormalTok{ F, }\DataTypeTok{booktabs =}\NormalTok{ T, }\DataTypeTok{digits =} \DecValTok{2}\NormalTok{,}
        \DataTypeTok{col.names =} \KeywordTok{c}\NormalTok{( }\StringTok{"Term"}\NormalTok{, }\KeywordTok{rep}\NormalTok{(}\KeywordTok{c}\NormalTok{(}\StringTok{"b"}\NormalTok{, }\StringTok{"CI"}\NormalTok{), }\DataTypeTok{times =} \DecValTok{2}\NormalTok{)),}
        \DataTypeTok{caption =} \StringTok{"Question 3B Results"}\NormalTok{) }\OperatorTok
\StringTok{  }\KeywordTok{kable_styling}\NormalTok{(}\DataTypeTok{full_width =}\NormalTok{ F) }\OperatorTok\StringTok{ }
\StringTok{  }\KeywordTok{add_header_above}\NormalTok{(}\KeywordTok{c}\NormalTok{(}\StringTok{" "}\NormalTok{ =}\StringTok{ }\DecValTok{1}\NormalTok{, }\StringTok{"No Intercept Model"}\NormalTok{ =}\StringTok{ }\DecValTok{2}\NormalTok{, }\StringTok{"Traditional Model"}\NormalTok{ =}\StringTok{ }\DecValTok{2}\NormalTok{))}
\end{Highlighting}
\end{Shaded}

\begin{table}

\caption{\label{tab:unnamed-chunk-10}Question 3B Results}
\centering
\begin{tabular}[t]{lllll}
\toprule
\multicolumn{1}{c}{ } & \multicolumn{2}{c}{No Intercept Model} & \multicolumn{2}{c}{Traditional Model} \\
\cmidrule(l{2pt}r{2pt}){2-3} \cmidrule(l{2pt}r{2pt}){4-5}
Term & b & CI & b & CI\\
\midrule
(Intercept) &  &  & 14.45 & [14.05, 14.75]\\
female & 2.61 & [1.80, 3.43] & -1.51 & [-2.12, -1.11]\\
minority & -7.51 & [-8.59, -6.43] & -3.96 & [-4.51, -3.66]\\
minority:female & 0.40 & [-0.35, 1.16] & 0.40 & [-0.08, 0.89]\\
\bottomrule
\end{tabular}
\end{table}

\subsection{Part C}\label{part-c-1}

Construct weights for combining the fixed effect parameters from (a)
that will reproduce the means for the four groups (MB, MG, NMB, and
NMG).

\begin{Shaded}
\begin{Highlighting}[]
\NormalTok{contra3 <-}\StringTok{ }\KeywordTok{matrix}\NormalTok{(}\KeywordTok{c}\NormalTok{(}
  \DecValTok{1}\NormalTok{, }\DecValTok{1}\NormalTok{, }\DecValTok{0}\NormalTok{, }\DecValTok{0}\NormalTok{,}
  \DecValTok{1}\NormalTok{, }\DecValTok{1}\NormalTok{, }\DecValTok{1}\NormalTok{, }\DecValTok{1}\NormalTok{,}
  \DecValTok{1}\NormalTok{, }\DecValTok{0}\NormalTok{, }\DecValTok{0}\NormalTok{, }\DecValTok{0}\NormalTok{,}
  \DecValTok{1}\NormalTok{, }\DecValTok{0}\NormalTok{, }\DecValTok{1}\NormalTok{, }\DecValTok{0}
\NormalTok{), }\DataTypeTok{nrow=}\DecValTok{4}\NormalTok{,}\DataTypeTok{ncol=}\DecValTok{4}\NormalTok{,}\DataTypeTok{byrow=}\OtherTok{TRUE}\NormalTok{)}
\KeywordTok{rownames}\NormalTok{(contra3) <-}\StringTok{ }\KeywordTok{c}\NormalTok{(}\StringTok{"MB"}\NormalTok{, }\StringTok{"MG"}\NormalTok{, }\StringTok{"NMB"}\NormalTok{, }\StringTok{"NMG"}\NormalTok{)}
\end{Highlighting}
\end{Shaded}

\subsection{Part D}\label{part-d-1}

Use the weight matrix with \texttt{glht()}. How close are the means to
those that were present in the fixed effect parameters for the model in
Question 2?

\begin{Shaded}
\begin{Highlighting}[]
\NormalTok{glht_mod3   <-}\StringTok{ }\KeywordTok{glht}\NormalTok{(mod3, }\DataTypeTok{linfct =}\NormalTok{ contra3, }\DataTypeTok{alternative =} \StringTok{"two.sided"}\NormalTok{, }\DataTypeTok{rhs=}\DecValTok{0}\NormalTok{)}
\NormalTok{res_mod3    <-}\StringTok{ }\KeywordTok{confint}\NormalTok{(glht_mod3, }\DataTypeTok{calpha =} \KeywordTok{univariate_calpha}\NormalTok{()) }
\NormalTok{res_mod3_df <-}\StringTok{ }\NormalTok{res_mod3}\OperatorTok{$}\NormalTok{confint }\OperatorTok\StringTok{ }\KeywordTok{data.frame}\NormalTok{()}

\NormalTok{res_mod3_df }\OperatorTok\StringTok{ }
\StringTok{  }\KeywordTok{mutate}\NormalTok{(}\DataTypeTok{Estimate =} \KeywordTok{sprintf}\NormalTok{(}\StringTok{"%.2f"}\NormalTok{, Estimate),}
         \DataTypeTok{term =} \KeywordTok{rownames}\NormalTok{(.),}
         \DataTypeTok{CI =} \KeywordTok{sprintf}\NormalTok{(}\StringTok{"[%.2f, %.2f]"}\NormalTok{, lwr, upr)) }\OperatorTok
\StringTok{  }\KeywordTok{select}\NormalTok{(term, Estimate, CI) }\OperatorTok
\StringTok{  }\KeywordTok{setNames}\NormalTok{(}\KeywordTok{c}\NormalTok{(}\StringTok{"term"}\NormalTok{, }\StringTok{"b"}\NormalTok{, }\StringTok{"CI"}\NormalTok{)) }\OperatorTok
\StringTok{  }\KeywordTok{mutate}\NormalTok{(}\DataTypeTok{model =} \StringTok{"ghlt Estimates"}\NormalTok{) }\OperatorTok
\StringTok{  }\KeywordTok{full_join}\NormalTok{(}
\NormalTok{    tab2 }\OperatorTok\StringTok{ }\KeywordTok{filter}\NormalTok{(type }\OperatorTok{==}\StringTok{ "Fixed Parts"}\NormalTok{) }\OperatorTok
\StringTok{              }\KeywordTok{mutate}\NormalTok{(}\DataTypeTok{model =} \StringTok{"No Intercept Model"}\NormalTok{) }\OperatorTok
\StringTok{              }\KeywordTok{select}\NormalTok{(}\OperatorTok{-}\NormalTok{type)}
\NormalTok{  ) }\OperatorTok
\StringTok{  }\KeywordTok{gather}\NormalTok{(}\DataTypeTok{key =}\NormalTok{ est, }\DataTypeTok{value =}\NormalTok{ value, b, CI) }\OperatorTok
\StringTok{  }\KeywordTok{unite}\NormalTok{(tmp, model, est, }\DataTypeTok{sep =} \StringTok{"."}\NormalTok{) }\OperatorTok
\StringTok{  }\KeywordTok{spread}\NormalTok{(}\DataTypeTok{key =}\NormalTok{ tmp, }\DataTypeTok{value =}\NormalTok{ value) }\OperatorTok
\StringTok{  }\KeywordTok{kable}\NormalTok{(., }\StringTok{"latex"}\NormalTok{, }\DataTypeTok{escape =}\NormalTok{ F, }\DataTypeTok{booktabs =}\NormalTok{ T, }\DataTypeTok{digits =} \DecValTok{2}\NormalTok{,}
        \DataTypeTok{col.names =} \KeywordTok{c}\NormalTok{( }\StringTok{"Term"}\NormalTok{, }\KeywordTok{rep}\NormalTok{(}\KeywordTok{c}\NormalTok{(}\StringTok{"b"}\NormalTok{, }\StringTok{"CI"}\NormalTok{), }\DataTypeTok{times =} \DecValTok{2}\NormalTok{)),}
        \DataTypeTok{caption =} \StringTok{"Question 3D Results"}\NormalTok{) }\OperatorTok
\StringTok{  }\KeywordTok{kable_styling}\NormalTok{(}\DataTypeTok{full_width =}\NormalTok{ F) }\OperatorTok\StringTok{ }
\StringTok{  }\KeywordTok{add_header_above}\NormalTok{(}\KeywordTok{c}\NormalTok{(}\StringTok{" "}\NormalTok{ =}\StringTok{ }\DecValTok{1}\NormalTok{, }\StringTok{"glht Estimates"}\NormalTok{ =}\StringTok{ }\DecValTok{2}\NormalTok{, }\StringTok{"No Intercept Model"}\NormalTok{ =}\StringTok{ }\DecValTok{2}\NormalTok{))}
\end{Highlighting}
\end{Shaded}

\begin{table}

\caption{\label{tab:unnamed-chunk-12}Question 3D Results}
\centering
\begin{tabular}[t]{lllll}
\toprule
\multicolumn{1}{c}{ } & \multicolumn{2}{c}{glht Estimates} & \multicolumn{2}{c}{No Intercept Model} \\
\cmidrule(l{2pt}r{2pt}){2-3} \cmidrule(l{2pt}r{2pt}){4-5}
Term & b & CI & b & CI\\
\midrule
MB & 10.49 & [9.78, 11.20] & 10.49 & [9.83, 10.97]\\
MG & 9.39 & [8.66, 10.12] & 9.39 & [8.54, 9.83]\\
NMB & 14.45 & [13.98, 14.92] & 14.45 & [14.14, 14.66]\\
NMG & 12.94 & [12.52, 13.36] & 12.94 & [12.69, 13.23]\\
\bottomrule
\end{tabular}
\end{table}

The means are nearly identical but have different confidence intervals.

\section{Question 4}\label{question-4}

Fit the following model:

\subsection{Part A}\label{part-a-2}

mathach \textasciitilde{} 1 + MB + NMG + NMB + (1 + MB + NMG +
NMB\textbar{}School).

\begin{Shaded}
\begin{Highlighting}[]
\NormalTok{mod4 <-}\StringTok{ }\KeywordTok{lmer}\NormalTok{(mathach }\OperatorTok{~}\StringTok{ }\DecValTok{1} \OperatorTok{+}\StringTok{ }\NormalTok{MB }\OperatorTok{+}\StringTok{ }\NormalTok{NMG }\OperatorTok{+}\StringTok{ }\NormalTok{NMB }\OperatorTok{+}\StringTok{ }\NormalTok{(}\DecValTok{1} \OperatorTok{+}\StringTok{ }\NormalTok{MB }\OperatorTok{+}\StringTok{ }\NormalTok{NMG }\OperatorTok{+}\StringTok{ }\NormalTok{NMB}\OperatorTok{|}\NormalTok{School), }\DataTypeTok{data =}\NormalTok{ dat)}
\NormalTok{tab4 <-}\StringTok{ }\KeywordTok{table_fun}\NormalTok{(mod4)}
\end{Highlighting}
\end{Shaded}

\subsection{Part B}\label{part-b-2}

What do the fixed effect parameters mean in this analysis?

The intercept (\(\gamma_{00}\)) represents the mean of the minority
girls. The MB (\(\gamma_{10}\)) represenets the difference between
minority girls and boys -- minority boys have higher math achivement
than minority girls on average. The NMG term (\(\gamma_{20}\))
represents the difference between minority and non-minority girls --
non-minority girls have higher math achivement than minority girls on
average. The NMB term (\(\gamma_{30}\)) represents the difference
between minority girls and boys -- minority boys have higher math
achivement than minority girls on average.

\subsection{Part C}\label{part-c-2}

Construct the weight matrix that will reproduce the means for the four
groups (MB, MG, NMB, and NMG).

\begin{Shaded}
\begin{Highlighting}[]
\NormalTok{contra4c <-}\StringTok{ }\KeywordTok{matrix}\NormalTok{(}\KeywordTok{c}\NormalTok{(}
   \DecValTok{1}\NormalTok{, }\DecValTok{1}\NormalTok{, }\DecValTok{0}\NormalTok{, }\DecValTok{0}\NormalTok{,}
   \DecValTok{1}\NormalTok{, }\DecValTok{0}\NormalTok{, }\DecValTok{0}\NormalTok{, }\DecValTok{0}\NormalTok{,}
   \DecValTok{1}\NormalTok{, }\DecValTok{0}\NormalTok{, }\DecValTok{1}\NormalTok{, }\DecValTok{0}\NormalTok{,}
   \DecValTok{1}\NormalTok{, }\DecValTok{0}\NormalTok{, }\DecValTok{0}\NormalTok{, }\DecValTok{1}
\NormalTok{), }\DataTypeTok{nrow=}\DecValTok{4}\NormalTok{,}\DataTypeTok{ncol=}\DecValTok{4}\NormalTok{,}\DataTypeTok{byrow=}\OtherTok{TRUE}\NormalTok{)}
\KeywordTok{rownames}\NormalTok{(contra4c) <-}\StringTok{ }\KeywordTok{c}\NormalTok{(}\StringTok{"MB"}\NormalTok{, }\StringTok{"MG"}\NormalTok{, }\StringTok{"NMB"}\NormalTok{, }\StringTok{"NMG"}\NormalTok{)}
\end{Highlighting}
\end{Shaded}

\subsection{Part D}\label{part-d-2}

Construct the weight matrix necessary to reproduce the tests of the two
main effects and the interaction.

\begin{Shaded}
\begin{Highlighting}[]
\NormalTok{contra4d <-}\StringTok{ }\KeywordTok{matrix}\NormalTok{(}\KeywordTok{c}\NormalTok{(}
    \DecValTok{0}\NormalTok{,  }\DecValTok{1}\NormalTok{, }\OperatorTok{-}\DecValTok{1}\NormalTok{, }\DecValTok{1}\NormalTok{, }\CommentTok{# b0 + b0 + b2 - (b0 + b1 + b0 + b3) -- > b2 = b1 + b3}
    \DecValTok{0}\NormalTok{,  }\DecValTok{1}\NormalTok{, }\OperatorTok{-}\DecValTok{1}\NormalTok{,}\OperatorTok{-}\DecValTok{1}\NormalTok{, }\CommentTok{# b0 + b0 + b1 - (b0 + b2 + b0 + b3) -- > b1 = b2 + b3}
    \DecValTok{0}\NormalTok{, }\OperatorTok{-}\DecValTok{1}\NormalTok{, }\OperatorTok{-}\DecValTok{1}\NormalTok{, }\DecValTok{1}  \CommentTok{# b0 + b0 + b3 - (b0 + b1 + b0 + b2) -- > b2 = b1 + b2}
\NormalTok{), }\DataTypeTok{nrow=}\DecValTok{3}\NormalTok{,}\DataTypeTok{ncol=}\DecValTok{4}\NormalTok{,}\DataTypeTok{byrow=}\OtherTok{TRUE}\NormalTok{)}
\KeywordTok{rownames}\NormalTok{(contra4d) <-}\StringTok{ }\KeywordTok{c}\NormalTok{(}\StringTok{"female"}\NormalTok{, }\StringTok{"minority"}\NormalTok{, }\StringTok{"minority:female"}\NormalTok{)}
\end{Highlighting}
\end{Shaded}

\subsection{Part E}\label{part-e-1}

Use the matrix in glht( ) and compare the results to those obtained in
Question 2.

\begin{Shaded}
\begin{Highlighting}[]
\NormalTok{contra4 <-}\StringTok{ }\KeywordTok{rbind}\NormalTok{(contra4c, contra4d)}

\NormalTok{glht_mod4   <-}\StringTok{ }\KeywordTok{glht}\NormalTok{(mod4, }\DataTypeTok{linfct =}\NormalTok{ contra4, }\DataTypeTok{alternative =} \StringTok{"two.sided"}\NormalTok{, }\DataTypeTok{rhs=}\DecValTok{0}\NormalTok{)}
\NormalTok{res_mod4    <-}\StringTok{ }\KeywordTok{confint}\NormalTok{(glht_mod4, }\DataTypeTok{calpha =} \KeywordTok{univariate_calpha}\NormalTok{()) }
\NormalTok{res_mod4_df <-}\StringTok{ }\NormalTok{res_mod4}\OperatorTok{$}\NormalTok{confint }\OperatorTok\StringTok{ }\KeywordTok{data.frame}\NormalTok{() }\OperatorTok\StringTok{ }\KeywordTok{mutate}\NormalTok{(}\DataTypeTok{term =} \KeywordTok{row.names}\NormalTok{(.))}

\NormalTok{res_mod2_df }\OperatorTok\StringTok{ }
\StringTok{  }\KeywordTok{mutate}\NormalTok{(}\DataTypeTok{Estimate =} \KeywordTok{sprintf}\NormalTok{(}\StringTok{"%.2f"}\NormalTok{, Estimate),}
         \DataTypeTok{CI =} \KeywordTok{sprintf}\NormalTok{(}\StringTok{"[%.2f, %.2f]"}\NormalTok{, lwr, upr)) }\OperatorTok
\StringTok{  }\KeywordTok{select}\NormalTok{(term, Estimate, CI) }\OperatorTok
\StringTok{  }\KeywordTok{setNames}\NormalTok{(}\KeywordTok{c}\NormalTok{(}\StringTok{"term"}\NormalTok{, }\StringTok{"b"}\NormalTok{, }\StringTok{"CI"}\NormalTok{)) }\OperatorTok
\StringTok{  }\KeywordTok{filter}\NormalTok{(term }\OperatorTok\StringTok{ }\KeywordTok{c}\NormalTok{(}\StringTok{"female"}\NormalTok{, }\StringTok{"minority"}\NormalTok{, }\StringTok{"minority:female"}\NormalTok{)) }\OperatorTok
\StringTok{  }\KeywordTok{mutate}\NormalTok{(}\DataTypeTok{model =} \StringTok{"Q2 Model"}\NormalTok{) }\OperatorTok\StringTok{ }
\StringTok{  }\KeywordTok{full_join}\NormalTok{(}
\NormalTok{    tab2 }\OperatorTok\StringTok{ }\KeywordTok{filter}\NormalTok{(type }\OperatorTok{==}\StringTok{ "Fixed Parts"}\NormalTok{) }\OperatorTok
\StringTok{              }\KeywordTok{mutate}\NormalTok{(}\DataTypeTok{model =} \StringTok{"Q2 Model"}\NormalTok{) }\OperatorTok
\StringTok{              }\KeywordTok{select}\NormalTok{(}\OperatorTok{-}\NormalTok{type)}
\NormalTok{  ) }\OperatorTok
\StringTok{  }\KeywordTok{full_join}\NormalTok{(}
\NormalTok{    res_mod4_df }\OperatorTok\StringTok{ }
\StringTok{      }\KeywordTok{mutate}\NormalTok{(}\DataTypeTok{model =} \StringTok{"Q4 Model"}\NormalTok{,}
             \DataTypeTok{Estimate =} \KeywordTok{sprintf}\NormalTok{(}\StringTok{"%.2f"}\NormalTok{, Estimate),}
             \DataTypeTok{CI =} \KeywordTok{sprintf}\NormalTok{(}\StringTok{"[%.2f, %.2f]"}\NormalTok{, lwr, upr)) }\OperatorTok
\StringTok{      }\KeywordTok{select}\NormalTok{(term, Estimate, CI, model) }\OperatorTok
\StringTok{      }\KeywordTok{setNames}\NormalTok{(}\KeywordTok{c}\NormalTok{(}\StringTok{"term"}\NormalTok{, }\StringTok{"b"}\NormalTok{, }\StringTok{"CI"}\NormalTok{, }\StringTok{"model"}\NormalTok{))}
\NormalTok{  ) }\OperatorTok
\StringTok{  }\KeywordTok{gather}\NormalTok{(}\DataTypeTok{key =}\NormalTok{ est, }\DataTypeTok{value =}\NormalTok{ value, b, CI) }\OperatorTok
\StringTok{  }\KeywordTok{unite}\NormalTok{(tmp, model, est, }\DataTypeTok{sep =} \StringTok{"."}\NormalTok{) }\OperatorTok
\StringTok{  }\KeywordTok{spread}\NormalTok{(}\DataTypeTok{key =}\NormalTok{ tmp, }\DataTypeTok{value =}\NormalTok{ value) }\OperatorTok
\StringTok{  }\KeywordTok{kable}\NormalTok{(., }\StringTok{"latex"}\NormalTok{, }\DataTypeTok{escape =}\NormalTok{ F, }\DataTypeTok{booktabs =}\NormalTok{ T, }\DataTypeTok{digits =} \DecValTok{2}\NormalTok{,}
        \DataTypeTok{col.names =} \KeywordTok{c}\NormalTok{( }\StringTok{"Term"}\NormalTok{, }\KeywordTok{rep}\NormalTok{(}\KeywordTok{c}\NormalTok{(}\StringTok{"b"}\NormalTok{, }\StringTok{"CI"}\NormalTok{), }\DataTypeTok{times =} \DecValTok{2}\NormalTok{)),}
        \DataTypeTok{caption =} \StringTok{"Question 4D Results"}\NormalTok{) }\OperatorTok
\StringTok{  }\KeywordTok{kable_styling}\NormalTok{(}\DataTypeTok{full_width =}\NormalTok{ F) }\OperatorTok\StringTok{ }
\StringTok{  }\KeywordTok{add_header_above}\NormalTok{(}\KeywordTok{c}\NormalTok{(}\StringTok{" "}\NormalTok{ =}\StringTok{ }\DecValTok{1}\NormalTok{, }\StringTok{"Q2 Model"}\NormalTok{ =}\StringTok{ }\DecValTok{2}\NormalTok{, }\StringTok{"Q4 Model"}\NormalTok{ =}\StringTok{ }\DecValTok{2}\NormalTok{))}
\end{Highlighting}
\end{Shaded}

\begin{table}

\caption{\label{tab:unnamed-chunk-16}Question 4D Results}
\centering
\begin{tabular}[t]{lllll}
\toprule
\multicolumn{1}{c}{ } & \multicolumn{2}{c}{Q2 Model} & \multicolumn{2}{c}{Q4 Model} \\
\cmidrule(l{2pt}r{2pt}){2-3} \cmidrule(l{2pt}r{2pt}){4-5}
Term & b & CI & b & CI\\
\midrule
female & 2.61 & [1.80, 3.43] & 2.61 & [1.80, 3.43]\\
MB & 10.49 & [9.83, 10.97] & 10.49 & [9.78, 11.20]\\
MG & 9.39 & [8.54, 9.83] & 9.39 & [8.66, 10.12]\\
minority & -7.51 & [-8.59, -6.43] & -7.51 & [-8.59, -6.43]\\
minority:female & 0.40 & [-0.35, 1.16] & 0.40 & [-0.35, 1.16]\\
\addlinespace
NMB & 14.45 & [14.14, 14.66] & 12.94 & [12.52, 13.36]\\
NMG & 12.94 & [12.69, 13.23] & 14.45 & [13.98, 14.92]\\
\bottomrule
\end{tabular}
\end{table}

\section{Question 5}\label{question-5}

Compare the fit for the three models using the \texttt{anova(\ )}
function. If you named the fit objects: Fit\_1, Fit\_2, and Fit\_3, then
use \texttt{anova(Fit\_1,\ Fit\_2,\ Fit\_3)}. The result should not
surprise you; why not?

\begin{Shaded}
\begin{Highlighting}[]
\KeywordTok{anova}\NormalTok{(mod2, mod3, mod4)}
\end{Highlighting}
\end{Shaded}

\begin{verbatim}
## Data: dat
## Models:
## mod2: mathach ~ -1 + MG + MB + NMG + NMB + (-1 + MG + MB + NMG + NMB | 
## mod2:     School)
## mod3: mathach ~ 1 + minority * female + (1 + minority * female | School)
## mod4: mathach ~ 1 + MB + NMG + NMB + (1 + MB + NMG + NMB | School)
##      Df   AIC   BIC logLik deviance Chisq Chi Df Pr(>Chisq)    
## mod2 15 46754 46857 -23362    46724                            
## mod3 15 46754 46857 -23362    46724     0      0     <2e-16 ***
## mod4 15 46754 46857 -23362    46724     0      0          1    
## ---
## Signif. codes:  0 '***' 0.001 '**' 0.01 '*' 0.05 '.' 0.1 ' ' 1
\end{verbatim}

In terms of fit statistics, these three models are identical because
they are testing different linear combinations of the same data and
hypotheses.


\end{document}
