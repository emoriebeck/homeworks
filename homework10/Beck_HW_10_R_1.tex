\documentclass[]{article}
\usepackage{lmodern}
\usepackage{amssymb,amsmath}
\usepackage{ifxetex,ifluatex}
\usepackage{fixltx2e} % provides \textsubscript
\ifnum 0\ifxetex 1\fi\ifluatex 1\fi=0 % if pdftex
  \usepackage[T1]{fontenc}
  \usepackage[utf8]{inputenc}
\else % if luatex or xelatex
  \ifxetex
    \usepackage{mathspec}
  \else
    \usepackage{fontspec}
  \fi
  \defaultfontfeatures{Ligatures=TeX,Scale=MatchLowercase}
\fi
% use upquote if available, for straight quotes in verbatim environments
\IfFileExists{upquote.sty}{\usepackage{upquote}}{}
% use microtype if available
\IfFileExists{microtype.sty}{%
\usepackage{microtype}
\UseMicrotypeSet[protrusion]{basicmath} % disable protrusion for tt fonts
}{}
\usepackage[margin=1in]{geometry}
\usepackage{hyperref}
\hypersetup{unicode=true,
            pdftitle={Homework 10},
            pdfauthor={Emorie Beck},
            pdfborder={0 0 0},
            breaklinks=true}
\urlstyle{same}  % don't use monospace font for urls
\usepackage{color}
\usepackage{fancyvrb}
\newcommand{\VerbBar}{|}
\newcommand{\VERB}{\Verb[commandchars=\\\{\}]}
\DefineVerbatimEnvironment{Highlighting}{Verbatim}{commandchars=\\\{\}}
% Add ',fontsize=\small' for more characters per line
\usepackage{framed}
\definecolor{shadecolor}{RGB}{248,248,248}
\newenvironment{Shaded}{\begin{snugshade}}{\end{snugshade}}
\newcommand{\KeywordTok}[1]{\textcolor[rgb]{0.13,0.29,0.53}{\textbf{#1}}}
\newcommand{\DataTypeTok}[1]{\textcolor[rgb]{0.13,0.29,0.53}{#1}}
\newcommand{\DecValTok}[1]{\textcolor[rgb]{0.00,0.00,0.81}{#1}}
\newcommand{\BaseNTok}[1]{\textcolor[rgb]{0.00,0.00,0.81}{#1}}
\newcommand{\FloatTok}[1]{\textcolor[rgb]{0.00,0.00,0.81}{#1}}
\newcommand{\ConstantTok}[1]{\textcolor[rgb]{0.00,0.00,0.00}{#1}}
\newcommand{\CharTok}[1]{\textcolor[rgb]{0.31,0.60,0.02}{#1}}
\newcommand{\SpecialCharTok}[1]{\textcolor[rgb]{0.00,0.00,0.00}{#1}}
\newcommand{\StringTok}[1]{\textcolor[rgb]{0.31,0.60,0.02}{#1}}
\newcommand{\VerbatimStringTok}[1]{\textcolor[rgb]{0.31,0.60,0.02}{#1}}
\newcommand{\SpecialStringTok}[1]{\textcolor[rgb]{0.31,0.60,0.02}{#1}}
\newcommand{\ImportTok}[1]{#1}
\newcommand{\CommentTok}[1]{\textcolor[rgb]{0.56,0.35,0.01}{\textit{#1}}}
\newcommand{\DocumentationTok}[1]{\textcolor[rgb]{0.56,0.35,0.01}{\textbf{\textit{#1}}}}
\newcommand{\AnnotationTok}[1]{\textcolor[rgb]{0.56,0.35,0.01}{\textbf{\textit{#1}}}}
\newcommand{\CommentVarTok}[1]{\textcolor[rgb]{0.56,0.35,0.01}{\textbf{\textit{#1}}}}
\newcommand{\OtherTok}[1]{\textcolor[rgb]{0.56,0.35,0.01}{#1}}
\newcommand{\FunctionTok}[1]{\textcolor[rgb]{0.00,0.00,0.00}{#1}}
\newcommand{\VariableTok}[1]{\textcolor[rgb]{0.00,0.00,0.00}{#1}}
\newcommand{\ControlFlowTok}[1]{\textcolor[rgb]{0.13,0.29,0.53}{\textbf{#1}}}
\newcommand{\OperatorTok}[1]{\textcolor[rgb]{0.81,0.36,0.00}{\textbf{#1}}}
\newcommand{\BuiltInTok}[1]{#1}
\newcommand{\ExtensionTok}[1]{#1}
\newcommand{\PreprocessorTok}[1]{\textcolor[rgb]{0.56,0.35,0.01}{\textit{#1}}}
\newcommand{\AttributeTok}[1]{\textcolor[rgb]{0.77,0.63,0.00}{#1}}
\newcommand{\RegionMarkerTok}[1]{#1}
\newcommand{\InformationTok}[1]{\textcolor[rgb]{0.56,0.35,0.01}{\textbf{\textit{#1}}}}
\newcommand{\WarningTok}[1]{\textcolor[rgb]{0.56,0.35,0.01}{\textbf{\textit{#1}}}}
\newcommand{\AlertTok}[1]{\textcolor[rgb]{0.94,0.16,0.16}{#1}}
\newcommand{\ErrorTok}[1]{\textcolor[rgb]{0.64,0.00,0.00}{\textbf{#1}}}
\newcommand{\NormalTok}[1]{#1}
\usepackage{graphicx,grffile}
\makeatletter
\def\maxwidth{\ifdim\Gin@nat@width>\linewidth\linewidth\else\Gin@nat@width\fi}
\def\maxheight{\ifdim\Gin@nat@height>\textheight\textheight\else\Gin@nat@height\fi}
\makeatother
% Scale images if necessary, so that they will not overflow the page
% margins by default, and it is still possible to overwrite the defaults
% using explicit options in \includegraphics[width, height, ...]{}
\setkeys{Gin}{width=\maxwidth,height=\maxheight,keepaspectratio}
\IfFileExists{parskip.sty}{%
\usepackage{parskip}
}{% else
\setlength{\parindent}{0pt}
\setlength{\parskip}{6pt plus 2pt minus 1pt}
}
\setlength{\emergencystretch}{3em}  % prevent overfull lines
\providecommand{\tightlist}{%
  \setlength{\itemsep}{0pt}\setlength{\parskip}{0pt}}
\setcounter{secnumdepth}{0}
% Redefines (sub)paragraphs to behave more like sections
\ifx\paragraph\undefined\else
\let\oldparagraph\paragraph
\renewcommand{\paragraph}[1]{\oldparagraph{#1}\mbox{}}
\fi
\ifx\subparagraph\undefined\else
\let\oldsubparagraph\subparagraph
\renewcommand{\subparagraph}[1]{\oldsubparagraph{#1}\mbox{}}
\fi

%%% Use protect on footnotes to avoid problems with footnotes in titles
\let\rmarkdownfootnote\footnote%
\def\footnote{\protect\rmarkdownfootnote}

%%% Change title format to be more compact
\usepackage{titling}

% Create subtitle command for use in maketitle
\newcommand{\subtitle}[1]{
  \posttitle{
    \begin{center}\large#1\end{center}
    }
}

\setlength{\droptitle}{-2em}
  \title{Homework 10}
  \pretitle{\vspace{\droptitle}\centering\huge}
  \posttitle{\par}
\subtitle{Psych 5068}
  \author{Emorie Beck}
  \preauthor{\centering\large\emph}
  \postauthor{\par}
  \predate{\centering\large\emph}
  \postdate{\par}
  \date{\today}

\usepackage{fancyhdr}
\usepackage{array}
\usepackage{longtable}
\usepackage{lscape}
\newcommand{\blandscape}{\begin{landscape}}
\newcommand{\elandscape}{\end{landscape}}
\usepackage{dcolumn}
\usepackage{bbm}
\usepackage{threeparttable}
\usepackage{booktabs}
\usepackage{expex}
\usepackage{rotating, graphicx}
\usepackage{tabulary}
\usepackage{algorithm}
\usepackage{multirow}
\usepackage{colortbl}
\usepackage{longtable}
\usepackage{array}
\usepackage{multirow}
\usepackage[table]{xcolor}
\usepackage{wrapfig}
\usepackage{float}
\usepackage{pdflscape}
\usepackage{tabu}
\usepackage{threeparttable}
\usepackage{booktabs}
\usepackage{longtable}
\usepackage{array}
\usepackage{multirow}
\usepackage[table]{xcolor}
\usepackage{wrapfig}
\usepackage{float}
\usepackage{colortbl}
\usepackage{pdflscape}
\usepackage{tabu}
\usepackage{threeparttable}
\usepackage[normalem]{ulem}

\begin{document}
\maketitle

{
\setcounter{tocdepth}{2}
\tableofcontents
}
\section{Workspace}\label{workspace}

\subsection{Packages}\label{packages}

\subsection{Data}\label{data}

\begin{Shaded}
\begin{Highlighting}[]
\KeywordTok{source}\NormalTok{(}\StringTok{"https://raw.githubusercontent.com/emoriebeck/homeworks/master/table_fun.R"}\NormalTok{)}
\NormalTok{data_url <-}\StringTok{ "https://raw.githubusercontent.com/emoriebeck/homeworks/master/homework10/homework_10_long.csv"}
\NormalTok{dat      <-}\StringTok{ }\NormalTok{data_url }\OperatorTok\StringTok{ }\NormalTok{read.csv }\OperatorTok\StringTok{ }\NormalTok{tbl_df }\OperatorTok
\StringTok{  }\KeywordTok{mutate}\NormalTok{(}\DataTypeTok{N_items =} \KeywordTok{ifelse}\NormalTok{(R }\OperatorTok{==}\StringTok{ }\DecValTok{1}\NormalTok{, }\DecValTok{10}\NormalTok{, }\KeywordTok{ifelse}\NormalTok{(DC }\OperatorTok{==}\StringTok{ }\DecValTok{1}\NormalTok{, }\DecValTok{20}\NormalTok{, }\DecValTok{40}\NormalTok{)))}
\end{Highlighting}
\end{Shaded}

The file, homework\_10\_long.csv, contains data from a survey study in
which 245 undergraduates (sex is coded men = 1, women = 2) completed the
10-item Rosenberg Self-Esteem Scale, the 20-item Desire for Control
Scale, and the 40-item Narcissistic Personality Inventory. The items
have been appropriately reverse-coded to all be in a consistent
direction and they have been standardized. Higher item Z scores indicate
higher self-esteem, higher desire for control, and higher narcissism.
Using methods described in class, answer the following questions about
this sample. All of these questions should be answered using the results
from HLM analyses.

\section{Question 1}\label{question-1}

How reliable are these three scales? Using the results from the HLM
analysis, find the internal consistency reliabilities.

\begin{Shaded}
\begin{Highlighting}[]
\NormalTok{fit1 <-}\StringTok{ }\KeywordTok{lmer}\NormalTok{(Score }\OperatorTok{~}\StringTok{ }\OperatorTok{-}\DecValTok{1} \OperatorTok{+}\StringTok{ }\NormalTok{DC }\OperatorTok{+}\StringTok{ }\NormalTok{R }\OperatorTok{+}\StringTok{ }\NormalTok{N }\OperatorTok{+}\StringTok{ }
\StringTok{                 }\NormalTok{(}\OperatorTok{-}\DecValTok{1} \OperatorTok{+}\StringTok{ }\NormalTok{DC }\OperatorTok{+}\StringTok{ }\NormalTok{R }\OperatorTok{+}\StringTok{ }\NormalTok{N }\OperatorTok{|}\NormalTok{ID), }\DataTypeTok{data=}\NormalTok{dat,}
                 \DataTypeTok{REML=}\NormalTok{F)}
\KeywordTok{summary}\NormalTok{(fit1)}
\end{Highlighting}
\end{Shaded}

\begin{verbatim}
## Linear mixed model fit by maximum likelihood  ['lmerMod']
## Formula: Score ~ -1 + DC + R + N + (-1 + DC + R + N | ID)
##    Data: dat
## 
##      AIC      BIC   logLik deviance df.resid 
##  46742.5  46820.0 -23361.2  46722.5    17140 
## 
## Scaled residuals: 
##     Min      1Q  Median      3Q     Max 
## -4.2713 -0.7311 -0.1271  0.7877  2.9545 
## 
## Random effects:
##  Groups   Name Variance Std.Dev. Corr     
##  ID       DC   0.1670   0.4086            
##           R    0.3823   0.6183   0.46     
##           N    0.1001   0.3163   0.67 0.39
##  Residual      0.8364   0.9146            
## Number of obs: 17150, groups:  ID, 245
## 
## Fixed effects:
##      Estimate Std. Error t value
## DC -1.066e-15  2.919e-02       0
## R  -1.053e-15  4.361e-02       0
## N  -6.191e-16  2.222e-02       0
## 
## Correlation of Fixed Effects:
##   DC    R    
## R 0.373      
## N 0.543 0.325
\end{verbatim}

\begin{Shaded}
\begin{Highlighting}[]
\NormalTok{(varc1 <-}\StringTok{ }\KeywordTok{VarCorr}\NormalTok{(fit1) }\OperatorTok\StringTok{ }\KeywordTok{data.frame}\NormalTok{() }\OperatorTok
\StringTok{  }\KeywordTok{filter}\NormalTok{(}\KeywordTok{is.na}\NormalTok{(var2) }\OperatorTok{&}\StringTok{ }\OperatorTok{!}\KeywordTok{is.na}\NormalTok{(var1)) }\OperatorTok
\StringTok{  }\KeywordTok{mutate}\NormalTok{(}\DataTypeTok{N_items =} \KeywordTok{ifelse}\NormalTok{(var1 }\OperatorTok{==}\StringTok{ "R"}\NormalTok{, }\DecValTok{10}\NormalTok{, }\KeywordTok{ifelse}\NormalTok{(var1 }\OperatorTok{==}\StringTok{ "DC"}\NormalTok{, }\DecValTok{20}\NormalTok{, }\DecValTok{40}\NormalTok{)),}
         \DataTypeTok{ICC =}\NormalTok{ sdcor}\OperatorTok{^}\DecValTok{2} \OperatorTok{/}\StringTok{ }\NormalTok{(sdcor}\OperatorTok{^}\DecValTok{2} \OperatorTok{+}\StringTok{ }\KeywordTok{sigma}\NormalTok{(fit1)}\OperatorTok{^}\DecValTok{2}\OperatorTok{/}\NormalTok{N_items)) }\OperatorTok
\StringTok{  }\KeywordTok{select}\NormalTok{(var1, ICC))}
\end{Highlighting}
\end{Shaded}

\begin{verbatim}
##   var1       ICC
## 1   DC 0.7997107
## 2    R 0.8204978
## 3    N 0.8271683
\end{verbatim}

\section{Question 2}\label{question-2}

What are the correlations among the scale means?

\begin{Shaded}
\begin{Highlighting}[]
\NormalTok{vc1 <-}\StringTok{ }\KeywordTok{attr}\NormalTok{(}\KeywordTok{vcov}\NormalTok{(fit1), }\StringTok{"factors"}\NormalTok{)[[}\DecValTok{1}\NormalTok{]] }\OperatorTok\StringTok{ }\NormalTok{as.matrix}
\NormalTok{vc1[}\KeywordTok{upper.tri}\NormalTok{(vc1, }\DataTypeTok{diag =}\NormalTok{ T)] <-}\StringTok{ }\OtherTok{NA}
\NormalTok{vc1 }\OperatorTok\StringTok{ }\NormalTok{data.frame }\OperatorTok
\StringTok{  }\KeywordTok{mutate}\NormalTok{(}\DataTypeTok{v1 =} \KeywordTok{rownames}\NormalTok{(.)) }\OperatorTok
\StringTok{  }\KeywordTok{gather}\NormalTok{(}\DataTypeTok{key =}\NormalTok{ v2, }\DataTypeTok{value =}\NormalTok{ value, }\DataTypeTok{na.rm =}\NormalTok{ T, }\OperatorTok{-}\NormalTok{v1) }\OperatorTok
\StringTok{  }\KeywordTok{unite}\NormalTok{(v, v1, v2, }\DataTypeTok{sep =} \StringTok{"_"}\NormalTok{)}
\end{Highlighting}
\end{Shaded}

\begin{verbatim}
##      v     value
## 2 R_DC 0.3726347
## 3 N_DC 0.5428000
## 6  N_R 0.3247658
\end{verbatim}

\section{Question 3}\label{question-3}

Now find the latent variable correlations for these measures.

\subsection{Part A}\label{part-a}

What are the values of these correlations?

\begin{Shaded}
\begin{Highlighting}[]
\NormalTok{lc1 <-}\StringTok{ }\KeywordTok{cov2cor}\NormalTok{(}\KeywordTok{VarCorr}\NormalTok{(fit1)}\OperatorTok{$}\NormalTok{ID)}
\NormalTok{lc1[}\KeywordTok{upper.tri}\NormalTok{(lc1, }\DataTypeTok{diag =}\NormalTok{ T)] <-}\StringTok{ }\OtherTok{NA}
\NormalTok{lc1 }\OperatorTok\StringTok{ }\NormalTok{data.frame }\OperatorTok
\StringTok{  }\KeywordTok{mutate}\NormalTok{(}\DataTypeTok{v1 =} \KeywordTok{rownames}\NormalTok{(.)) }\OperatorTok
\StringTok{  }\KeywordTok{gather}\NormalTok{(}\DataTypeTok{key =}\NormalTok{ v2, }\DataTypeTok{value =}\NormalTok{ value, }\DataTypeTok{na.rm =}\NormalTok{ T, }\OperatorTok{-}\NormalTok{v1) }\OperatorTok
\StringTok{  }\KeywordTok{unite}\NormalTok{(v, v1, v2, }\DataTypeTok{sep =} \StringTok{"_"}\NormalTok{)}
\end{Highlighting}
\end{Shaded}

\begin{verbatim}
##      v     value
## 2 R_DC 0.4600214
## 3 N_DC 0.6673850
## 6  N_R 0.3942163
\end{verbatim}

\subsection{Part B}\label{part-b}

Is each latent variable correlation significantly different from 0?

\begin{Shaded}
\begin{Highlighting}[]
\NormalTok{CI1 <-}\StringTok{ }\KeywordTok{confint}\NormalTok{(fit1, }\DataTypeTok{oldNames =}\NormalTok{ F)}

\NormalTok{CI1 }\OperatorTok\StringTok{ }\NormalTok{data.frame }\OperatorTok\StringTok{ }
\StringTok{  }\KeywordTok{mutate}\NormalTok{(}\DataTypeTok{var =} \KeywordTok{rownames}\NormalTok{(.)) }\OperatorTok
\StringTok{  }\KeywordTok{filter}\NormalTok{(}\KeywordTok{grepl}\NormalTok{(}\StringTok{"cor"}\NormalTok{, var))}
\end{Highlighting}
\end{Shaded}

\begin{verbatim}
##      X2.5..   X97.5..         var
## 1 0.3230749 0.5798763 cor_R.DC|ID
## 2 0.5572817 0.7577931 cor_N.DC|ID
## 3 0.2543082 0.5191626  cor_N.R|ID
\end{verbatim}

Yes, none of the confidence intervals overlap with 0, so the
correlations are significant.

\subsection{Part C}\label{part-c}

Are the correlations collectively different from 0?

\begin{Shaded}
\begin{Highlighting}[]
\NormalTok{fit3c <-}\StringTok{ }\KeywordTok{lmer}\NormalTok{(Score }\OperatorTok{~}\StringTok{ }\OperatorTok{-}\DecValTok{1} \OperatorTok{+}\StringTok{ }\NormalTok{DC }\OperatorTok{+}\StringTok{ }\NormalTok{R }\OperatorTok{+}\StringTok{ }\NormalTok{N }\OperatorTok{+}\StringTok{ }
\StringTok{                 }\NormalTok{(}\OperatorTok{-}\DecValTok{1} \OperatorTok{+}\StringTok{ }\NormalTok{DC }\OperatorTok{+}\StringTok{ }\NormalTok{R }\OperatorTok{+}\StringTok{ }\NormalTok{N }\OperatorTok{||}\StringTok{ }\NormalTok{ID), }\DataTypeTok{data=}\NormalTok{dat,}
                 \DataTypeTok{REML=}\NormalTok{F)}
\KeywordTok{anova}\NormalTok{(fit1, fit3c)}
\end{Highlighting}
\end{Shaded}

\begin{verbatim}
## Data: dat
## Models:
## fit3c: Score ~ -1 + DC + R + N + ((0 + DC | ID) + (0 + R | ID) + (0 + 
## fit3c:     N | ID))
## fit1: Score ~ -1 + DC + R + N + (-1 + DC + R + N | ID)
##       Df   AIC   BIC logLik deviance  Chisq Chi Df Pr(>Chisq)    
## fit3c  7 46865 46919 -23425    46851                             
## fit1  10 46742 46820 -23361    46722 128.27      3  < 2.2e-16 ***
## ---
## Signif. codes:  0 '***' 0.001 '**' 0.01 '*' 0.05 '.' 0.1 ' ' 1
\end{verbatim}

The model with the correlations is better than one without them, so the
correlations are different than 0.

\section{Question 4}\label{question-4}

Are there sex differences for these scales?

\begin{Shaded}
\begin{Highlighting}[]
\NormalTok{dat <-}\StringTok{ }\NormalTok{dat }\OperatorTok\StringTok{ }\KeywordTok{mutate}\NormalTok{(}\DataTypeTok{Sex =} \KeywordTok{mapvalues}\NormalTok{(Sex, }\KeywordTok{c}\NormalTok{(}\DecValTok{1}\NormalTok{,}\DecValTok{2}\NormalTok{), }\KeywordTok{c}\NormalTok{(}\DecValTok{0}\NormalTok{,}\DecValTok{1}\NormalTok{)))}
\NormalTok{fit4 <-}\StringTok{ }\KeywordTok{lmer}\NormalTok{(Score }\OperatorTok{~}\StringTok{ }\OperatorTok{-}\DecValTok{1} \OperatorTok{+}\StringTok{ }\NormalTok{DC }\OperatorTok{+}\StringTok{ }\NormalTok{R }\OperatorTok{+}\StringTok{ }\NormalTok{N }\OperatorTok{+}\StringTok{ }\NormalTok{DC}\OperatorTok{:}\NormalTok{Sex }\OperatorTok{+}\StringTok{ }\NormalTok{R}\OperatorTok{:}\NormalTok{Sex }\OperatorTok{+}\StringTok{ }\NormalTok{N}\OperatorTok{:}\NormalTok{Sex }\OperatorTok{+}\StringTok{ }
\StringTok{                 }\NormalTok{(}\OperatorTok{-}\DecValTok{1} \OperatorTok{+}\StringTok{ }\NormalTok{DC }\OperatorTok{+}\StringTok{ }\NormalTok{R }\OperatorTok{+}\StringTok{ }\NormalTok{N }\OperatorTok{|}\NormalTok{ID), }\DataTypeTok{data=}\NormalTok{dat,}
                 \DataTypeTok{REML=}\NormalTok{F)}
\KeywordTok{summary}\NormalTok{(fit4)}
\end{Highlighting}
\end{Shaded}

\begin{verbatim}
## Linear mixed model fit by maximum likelihood  ['lmerMod']
## Formula: Score ~ -1 + DC + R + N + DC:Sex + R:Sex + N:Sex + (-1 + DC +  
##     R + N | ID)
##    Data: dat
## 
##      AIC      BIC   logLik deviance df.resid 
##  46740.7  46841.4 -23357.3  46714.7    17137 
## 
## Scaled residuals: 
##     Min      1Q  Median      3Q     Max 
## -4.2674 -0.7313 -0.1255  0.7885  2.9435 
## 
## Random effects:
##  Groups   Name Variance Std.Dev. Corr     
##  ID       DC   0.16301  0.4037            
##           R    0.37670  0.6138   0.45     
##           N    0.09701  0.3115   0.66 0.38
##  Residual      0.83641  0.9146            
## Number of obs: 17150, groups:  ID, 245
## 
## Fixed effects:
##        Estimate Std. Error t value
## DC      0.06756    0.04239   1.594
## R       0.08033    0.06355   1.264
## N       0.05933    0.03216   1.845
## DC:Sex -0.12635    0.05797  -2.180
## R:Sex  -0.15024    0.08690  -1.729
## N:Sex  -0.11095    0.04398  -2.523
## 
## Correlation of Fixed Effects:
##        DC     R      N      DC:Sex R:Sex 
## R       0.363                            
## N       0.533  0.313                     
## DC:Sex -0.731 -0.266 -0.390              
## R:Sex  -0.266 -0.731 -0.229  0.363       
## N:Sex  -0.390 -0.229 -0.731  0.533  0.313
\end{verbatim}

\subsection{Part A}\label{part-a-1}

Are men and women different in their means on each of these scales?

\begin{Shaded}
\begin{Highlighting}[]
\NormalTok{(res <-}\StringTok{ }\KeywordTok{table_fun}\NormalTok{(fit4) }\OperatorTok\StringTok{ }
\StringTok{  }\KeywordTok{filter}\NormalTok{(type }\OperatorTok{==}\StringTok{ "Fixed Parts"}\NormalTok{))}
\end{Highlighting}
\end{Shaded}

\begin{verbatim}
##          type   term     b             CI
## 1 Fixed Parts     DC  0.07   [0.04, 0.16]
## 2 Fixed Parts      R  0.08  [-0.00, 0.22]
## 3 Fixed Parts      N  0.06   [0.03, 0.10]
## 4 Fixed Parts DC:Sex -0.13 [-0.27, -0.10]
## 5 Fixed Parts  R:Sex -0.15 [-0.31, -0.09]
## 6 Fixed Parts  N:Sex -0.11 [-0.19, -0.07]
\end{verbatim}

Men and women differ in desire for control and narcissism. Women have
less desire for control (b = -0.13, 95\% CI = {[}-0.27, -0.10{]}) and
are less narcissitic (b = -0.11, 95\% CI = {[}-0.19, -0.07{]}) than men,
on average.

\subsection{Part B}\label{part-b-1}

Collectively, does participant sex add significantly to the original
model?

\begin{Shaded}
\begin{Highlighting}[]
\KeywordTok{anova}\NormalTok{(fit1, fit4)}
\end{Highlighting}
\end{Shaded}

\begin{verbatim}
## Data: dat
## Models:
## fit1: Score ~ -1 + DC + R + N + (-1 + DC + R + N | ID)
## fit4: Score ~ -1 + DC + R + N + DC:Sex + R:Sex + N:Sex + (-1 + DC + 
## fit4:     R + N | ID)
##      Df   AIC   BIC logLik deviance  Chisq Chi Df Pr(>Chisq)  
## fit1 10 46742 46820 -23361    46722                           
## fit4 13 46741 46841 -23357    46715 7.8074      3    0.05016 .
## ---
## Signif. codes:  0 '***' 0.001 '**' 0.01 '*' 0.05 '.' 0.1 ' ' 1
\end{verbatim}

Although there are sex differences, the model that inlcudes sex is only
marginally better.

\section{Question 5}\label{question-5}

Generally, research finds that narcissists think highly of themselves
and like to control their environments. That said, a narcissist with
high control needs but low self-esteem might be a problem. They might be
particularly likely to manipulate others in an attempt to restore a
grandiose sense of self. Using the latent variable scores fdrom the
initial analysis (Question 3), identify by ID number the person in the
sample who you think is the best candidate for this low self-esteem,
high desire for control, high narcissism label.

\begin{Shaded}
\begin{Highlighting}[]
\NormalTok{Latent_Scores_EB <-}\StringTok{ }\KeywordTok{coef}\NormalTok{(fit1)}\OperatorTok{$}\NormalTok{ID}
\NormalTok{(dark_triad <-}\StringTok{ }\NormalTok{Latent_Scores_EB }\OperatorTok\StringTok{ }\NormalTok{tbl_df }\OperatorTok
\StringTok{  }\KeywordTok{mutate}\NormalTok{(}\DataTypeTok{ID =} \KeywordTok{rownames}\NormalTok{(.)) }\OperatorTok
\StringTok{  }\KeywordTok{filter}\NormalTok{(}\KeywordTok{sign}\NormalTok{(R) }\OperatorTok{==}\StringTok{ }\OperatorTok{-}\DecValTok{1} \OperatorTok{&}\StringTok{ }\KeywordTok{sign}\NormalTok{(DC) }\OperatorTok{==}\StringTok{ }\DecValTok{1} \OperatorTok{&}\StringTok{ }\KeywordTok{sign}\NormalTok{(N) }\OperatorTok{==}\StringTok{ }\DecValTok{1}\NormalTok{) }\OperatorTok
\StringTok{  }\KeywordTok{mutate}\NormalTok{(}\DataTypeTok{dist =} \KeywordTok{rowSums}\NormalTok{(}\KeywordTok{abs}\NormalTok{(}\KeywordTok{cbind}\NormalTok{(DC}\OperatorTok{^}\DecValTok{2}\NormalTok{, R}\OperatorTok{^}\DecValTok{2}\NormalTok{, N}\OperatorTok{^}\DecValTok{2}\NormalTok{)))) }\OperatorTok
\StringTok{  }\KeywordTok{arrange}\NormalTok{(}\KeywordTok{desc}\NormalTok{(dist)))}
\end{Highlighting}
\end{Shaded}

\begin{verbatim}
## # A tibble: 35 x 5
##        DC       R      N ID     dist
##     <dbl>   <dbl>  <dbl> <chr> <dbl>
##  1 0.367  -1.01   0.416  89    1.32 
##  2 0.802  -0.316  0.667  64    1.19 
##  3 0.0634 -0.833  0.234  108   0.753
##  4 0.687  -0.133  0.292  25    0.574
##  5 0.553  -0.372  0.334  233   0.557
##  6 0.0433 -0.691  0.122  193   0.494
##  7 0.341  -0.605  0.0446 136   0.484
##  8 0.387  -0.0334 0.560  11    0.465
##  9 0.420  -0.455  0.0363 134   0.385
## 10 0.303  -0.463  0.278  148   0.383
## # ... with 25 more rows
\end{verbatim}

I looked at the sum of the squared differences from the means (0) for
each score for people above the mean in desire for control and
narcissism and below the mean in self-esteem. Based on this, Person 89
(R = -1.01; DC = 0.37; N = 0.42) appears to have the strongest
configuration of these traits, with individual 64 (R = -0.32; DC = 0.8;
N = 0.67)coming in a close second.


\end{document}
